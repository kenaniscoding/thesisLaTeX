Put an overview of the contents of chapter. Mention here your methodology flow through a figure and provide an overview of it and how your methodology achieves your objectives.  How your methodology achieves each of your specific objectives is what your panelists/examiners will be looking for.  Specify how your methodology achieves your general objective and specific objectives.  A point-by-point comparison how your methodology achieves each of your specific objectives is expected in the final \documentType.

Also make sure that you refer clearly to the chapters on the Literature Review, Theoretical Considerations, and Design Considerations showing how your methodology ties with those that you have discussed in those chapters.

Make an overview of the contents of the chapter. Put here your methodology flow through a figure and provide an overview of it.  


In summative form, Table~\ref{tab:methods_per_objective} indicates the approaches, designs, modes, processes, programs, techniques, and/or ways that the \documentType reaches the objectives. 


\begin{center}
	{\scriptsize
		\begin{tabularx}{\textwidth}{p{0.2\textwidth}|p{0.6\textwidth}|p{0.1\textwidth}}
			\caption{Summary of methods for reaching the objectives} \label{tab:methods_per_objective} \\
			\hline 
			\hline 
			\textbf{Objectives} & 
			\textbf{Methods} &
			\textbf{Locations}\\ 
			\hline 
			\endfirsthead
			\multicolumn{3}{c}%
			{\textit{Continued from previous page}} \\
			\hline
			\hline 
			\textbf{Objectives} & 
			\textbf{Methods} &
			\textbf{Locations}\\ 
			\hline 
			\endhead
			\hline 
			\multicolumn{3}{r}{\textit{Continued on next page}} \\ 
			\endfoot
			\hline 
			\endlastfoot
			\hline
			
			
			\Paste{GO} & \blindlist{enumerate} & Sec.~\ref{sec:implement} on p.~\pageref{sec:implement}\\ \hline
			
			
			\Paste{SO1} & \blindlist{enumerate} & Sec.~\ref{sec:implement} on p.~\pageref{sec:implement} \\ \hline
			
			
			\Paste{SO2} & \blindlist{enumerate} & Sec.~\ref{sec:implement} on p.~\pageref{sec:implement}\\ \hline
			
			
			\Paste{SO3} & \blindlist{enumerate} & Sec.~\ref{sec:implement} on p.~\pageref{sec:implement}\\ \hline
			
			
			\Paste{SO4} & \blindlist{enumerate} & Sec.~\ref{sec:implement} on p.~\pageref{sec:implement} \\ \hline
			
			
			\Paste{SO5} & \blindlist{enumerate} & Sec.~\ref{sec:implement} on p.~\pageref{sec:implement} \\ \hline
			
		\end{tabularx}
	}
\end{center}




\section{Implementation}
\label{sec:implement}

Summarize the process used to create/set-up the work with an explanation of such process, instruments, and materials that you used if any. If the description is lengthy, use condensed bullet points. 

\noindent \textit{Rule of thumb}: Implementation is how you made your  work; (keywords: implemented, created, made, soldered, programmed, etc.).

If you wrote a program or made a simulation, you must state how the program or simulation functions in this section.	An algorithm or a pseudocode as shown in Table~\ref{tab:calcxn} is a good example.


\graytx{\Blindtext}



\section{Evaluation}
\label{sec:evaluate}

Describe the procedures for evaluating the correct behavior and outcome of your  work, including what information you need to gather and how you will obtain or measure it.  

\textit{Rule of thumb}: Evaluation is how you tested your  work; (keywords: measured, tested, compared, simulated, etc.).

\graytx{\Blindtext}



\section{Summary}

Provide the gist of this chapter such that it reflects the contents and the message.
