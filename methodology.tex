
\begin{center}
	{\scriptsize
		\begin{tabularx}{\textwidth}{p{0.2\textwidth}|p{0.6\textwidth}|p{0.1\textwidth}}
			\caption{Summary of methods for reaching the objectives} \label{tab:reaching_objectives} \\
			\hline 
			\hline 
			\textbf{Objectives} & 
			\textbf{Methods} &
			\textbf{Locations}\\ 
			\hline 
			\endfirsthead
			\multicolumn{3}{c}%
			{\textit{Continued from previous page}} \\
			\hline
			\hline 
			\textbf{Objectives} & 
			\textbf{Methods} &
			\textbf{Locations}\\ 
			\hline 
			\endhead
			\hline 
			\multicolumn{3}{r}{\textit{Continued on next page}} \\ 
			\endfoot
			\hline 
			\endlastfoot
			\hline
			
			\Paste{GO} & 
			\begin{enumerate}
				\item Hardware design: Build an image acquisition system with a conveyor belt, LED lights, and Raspberry Pi Camera
				\item Software design: Coded a Raspberry Pi application to grade and sort the Carabao mangoes
			\end{enumerate} 
			& Sec.~\ref{sec:researchApproach} on p.~\pageref{sec:researchApproach} \\ \hline
			
			\Paste{SO1} & \begin{enumerate}
				\item Hardware implementation: Design and build an image acquisition system prototype 
			\end{enumerate} & Sec.~\ref{sec:hardwareDesign} on p.~\pageref{sec:hardwareDesign} \\ \hline
			
			\Paste{SO2} & \begin{enumerate}
				\item Performance testing: Train and test the machine learning algorithm for classifying bruises and ripeness
				\item Data collection: Gather our own Carabao mango dataset together with an online dataset
			\end{enumerate} & Sec.~\ref{sec:dataset} on p.~\pageref{sec:trainandtest} \\ \hline
			
			\Paste{SO3} & \begin{enumerate}
				\item Algorithm development: To develop a code for the image acquisition system
				\item Hardware design: To design a schematic for the microcontroller based system 
			\end{enumerate} & Sec.~\ref{sec:hardwareDesign} on p.~\pageref{sec:hardwareDesign} \\ \hline
			
			\Paste{SO4} & \begin{enumerate}
				\item Formula development: Formulated an equation based on the inputted user priority and the predicted mango classification
			\end{enumerate} & Sec.~\ref{sec:formula} on p.~\pageref{sec:formula} \\ \hline
			
			\Paste{SO5} & \begin{enumerate}
				\item Performance testing: Train and test the machine learning algorithm for classifying bruises
			\end{enumerate} & Sec.~\ref{sec:bruisestraining} on p.~\pageref{sec:bruisestraining} \\ \hline
			
			\Paste{SO6} & \begin{enumerate}
				\item Performance testing: Train and test the machine learning algorithm for classifying ripeness
			\end{enumerate} & Sec.~\ref{sec:ripenesstraining} on p.~\pageref{sec:ripenesstraining} \\ \hline
			
			\Paste{SO7} & \begin{enumerate}
				\item Accuracy testing: Get the percent accuracy testing for getting the length and width of the Carabao mango
			\end{enumerate} & Sec.~\ref{sec:sizeDetermination} on p.~\pageref{sec:sizeDetermination} \\ \hline
			
		\end{tabularx}
	}
\end{center}

\section{Introduction}
The methodology for this research outlines the development of the Carabao Mango sorter using machine learning and computer vision. The sorting system uses a conveyor belt system which delivers the mangoes into the image acquisition system. This system captures the image of the mangoes which will then be going through the various stages of image processing and classification into grades which will depend on the priority of the user. This methodology ensures that the grading of the mangoes will be accurate while being non-destructive.

\section{Research Approach} \label{sec:researchApproach}
This study applies the experimental approach for research in order to develop and properly test the proposed system. The experimental approach of the methodology will allow the researchers to fine-tune the parameters and other factors in the classification of mangoes in order to get optimal results with high accuracy scores while maintaining the quality of the mangoes. This approach will also allow for real-time data processing and classification which will improve the previous static grading systems.

\section{Hardware Design} \label{sec:hardwareDesign}
The prototype consists of hardware and software components for automated mango sorting and grading purposes. The hardware includes the conveyor belt system used to transfer mangoes from scanning to sorting smoothly. A camera and lighting system are able to collect high-resolution images for analysis. The DC motors and stepper motors are responsible for driving the conveyor belt and sorting actuators. The entire system is controlled by a microcontroller (Raspberry Pi 4b), coordinating actions of all components. Sorting actuators then direct mangoes into selected bins based on their classification to make sorting efficient. 

\section{Software Design} 
For the programming language used for the prototype and training and testing the CNN model, Python was used for training and testing the CNN model and it was also used in the microcontroller to run the application containing the UI and CNN model. PyTorch was the main library used in using the EfficientNet model that is used in classifying the ripeness and bruises of the mango. Likewise, tkinter is the used library when designing the UI in Python.

Furthermore, the rest of the software components are of utmost importance to mango classification. Image processing algorithms in OpenCV and CNN models extract features such as color, size, and bruises that are known to determine quality parameters of mangoes. Mangoes are classified based on ripeness and defects by using machine learning algorithms, which further enhances accuracy using deep learning techniques. A user interface (UI) is designed for users to control and observe the system in real time. Finally, the interface programming of the microcontroller provides the necessary synchronization between sensors, actuators, and motors throughout the sorting operation scenario.

\section{Data Collection Methods} \label{sec:dataset}
% The system acquires high-resolution images of mangoes under pre-specified lighting conditions through systematic acquisition. 
% Apart from that, this corpus of data is based on the real-time images acquired from the camera system, where classification 
% operations are carried out based on real-time data. Pre-processing image operations such as flipping, rotating, resizing, 
% normalization, and Gaussian blur are also carried out in order to enhance image clarity and feature detection. Then, 
% the feature extraction process is carried out, where the intensity of color, shape, and texture are analyzed for the 
% detection of characteristic features in terms of the mango. All these aspects lead to the creation of a reliable dataset 
% for the machine learning algorithm that will allow the system to classify and grade mangoes more accurately.

For the data collection, online available image datasets with Carabao mangoes were used together with the captured Carabao mango
images. For the setup of the captured Carabao mangoes, the height of the camera to the white flat surface is 26 cm which can 
be seen on Figure~\ref{fig:mangoImageCollection}. Furthermore,
the S24's camera is used for capturing both cheeks of the Carabao mango. Initially, the Carabao mangoes would be unripe and green
and each day the Carabao mangoes would be pictured until they are ripe. 

\begin{figure}[!htbp]
	\centering
	\includegraphics[width=0.5\textwidth, angle=-90]{camera_setup}
	\caption{Carabao Mango Image Data Collection}
	\label{fig:mangoImageCollection}
\end{figure}


\section{Testing and Evaluation Methods} \label{sec:trainandtest}
In a bid to ensure the mango sorting and grading system is accurate and reliable, there is 
intensive testing conducted at different levels. Unit testing is initially conducted 
on each component separately, for instance, the conveyor belt, sensors, and cameras, to ensure 
that each of the components works as expected when operating separately. After component testing 
on an individual basis, integration testing is conducted to ensure communication between hardware
 and software is correct to ensure the image processing system, motors, and sorting actuators work
  in concert as required. System testing is conducted to conduct overall system performance
  testing in real-world conditions to ensure mangoes are accurately and efficiently sorted and graded.

  
\subsection{Classification Report}
% describe classification report and how it is used in the system

\subsubsection{Confusion Matrix}

\begin{table}[h]
	\centering
	\begin{tabular}{c|c|c}
	\hline
	& \textbf{Predicted Positive} & \textbf{Predicted Negative} \\
	\hline
	\textbf{Actual Positive} & TP & FN \\
	\hline
	\textbf{Actual Negative} & FP & TN \\
	\hline
	\end{tabular}
	\caption{Confusion Matrix Example}
	\label{tab:confusion_matrix}
\end{table}

A confusion matrix is a table that visualizes the performance 
of a classification model. For a binary classification problem, it has four
components: \\

\begin{itemize}
	\item True Positives (TP): Cases correctly predicted as positive
	\item True Negatives (TN): Cases correctly predicted as negative
	\item False Positives (FP): Cases incorrectly predicted as positive. (Type I error)
	\item False Negatives (FN): Cases incorrectly predicted as negative (Type II error)
\end{itemize}


\subsubsection{Precision}
\begin{eqnarray}
	\text{Precision} = \frac{TP}{TP + FP}
	\label{eq:precision}
\end{eqnarray}

Precision measures how many of the predicted positives are actually positive. It answers the question: 
"When the model predicts the positive class, how often is it correct?" High precision means low false positives.

\subsubsection{Recall}
\begin{eqnarray}
	\text{Recall} = \frac{TP}{TP + FN}
	\label{eq:recall}
\end{eqnarray}

Recall, which is also called sensitivity, measures how many of the actual positives were correctly identified. 
It answers the question: 
"Of all the actual positive cases, how many did the model catch?" High recall means low false negatives.

\subsubsection{F1 Score}
\begin{eqnarray}
	F_1 = 2\times \frac{\text{Precision} \times \text{Recall}}{\text{Precision} + \text{Recall}}
	\label{eq:f1_score}
\end{eqnarray}

The F1 score is the harmonic mean of precision and recall. It provides a single metric that balances 
both concerns. This is particularly useful when you need to 
find a balance between precision and recall, as optimizing for one often decreases the other.

\subsubsection{Accuracy}
\begin{eqnarray}
	\text{Accuracy} = \frac{TP + TN}{TP + TN + FP + FN}
	\label{eq:accuracy}
\end{eqnarray}

Accuracy measures the proportion of correct predictions (both true positives and true negatives)
 among the total cases. While intuitive, accuracy can be misleading with imbalanced datasets.


To test system performance, various measures of performance are used to evaluate. 
As seen on equation~\ref{eq:accuracy}, \gls{accuracy score} is used to measure the percentage of 
correctly classified mangoes to ensure the system maintains high precision levels. 
\gls{Precision} as seen on equation~\ref{eq:precision} and \gls{recall} as seen on equation~\ref{eq:recall} 
are used to measure consistency of classification to determine if the system classifies different ripeness 
levels and defects correctly. Furthermore, the F1 score formula 
as seen on equation~\ref{eq:f1_score} is used to evaluate the performance of the model's classification. 

A \gls{confusion matrix} is used to measure correct and incorrect classification to ensure the machine learning model 
is optimized and that minimum errors are achieved. 
Throughput analysis is also used to determine the rate and efficiency of sorting to 
ensure that the system maintains high capacity without bottlenecks to sort mangoes. 
Using these methods of testing, the system is constantly optimized to ensure high-quality and reliable mango classification.

\subsection{Ripeness Training and Testing} \label{sec:ripenesstraining}

For the testing of the ripeness classification, the Carabao mangoes are classified into three ripeness stages which are Green, green yellow, and yellow.
Likewise, The green would represent the ripe mangoes while the green yellow would represent the semi ripe while the yellow would
represent the ripe mangoes. As reference, Figure~\ref{fig:mangoDefect} shows the different ripeness stages for Carabao/Pico mangoes.

\begin{figure}[!htbp]
	\centering
	\includegraphics[width=0.7\textwidth]{ripeness_stages}
	\caption{Carabao Mango Ripeness Stages}
	\label{fig:mangoRipeness}
\end{figure}

\subsection{Bruises Training and Testing} \label{sec:bruisestraining}

For the testing of the bruise classification of the Carabao mangoes, it would classified into two categories which are bruised and not bruised. 
To define what bruise and not bruise mangoes looked like Figure~\ref{fig:mangoDefect} is used as reference to categorize which mangoes are bruised and not bruised.

\begin{figure}[!htbp]
	\centering
	\includegraphics[width=0.7\textwidth]{bruise_guide}
	\caption{Different Kinds of Mango Defects}
	\label{fig:mangoDefect}
\end{figure}

\subsubsection{Stem End Rots}
Stem end rots are characterized by fast-growing, watery, soft rots that penetrate deeply into the flesh. They usually appear as grey-brown or black rots starting from the stem end, often without obvious spores, and can spread rapidly into the mango. 

\subsubsection{Dendritic Spot}
Dendritic spots, on the other hand, are small black spots with irregular edges scattered across the skin. They grow slowly and do not penetrate into the flesh, remaining largely superficial.

\subsubsection{Anthracnose}
Anthracnose can appear in two forms. Body anthracnose presents as black rots on the fruit surface that are usually round, slightly sunken, and located on different parts of the mango. Stem end anthracnose occurs around the stem, also presenting as black rots. While these rots do not penetrate deeply into the flesh, advanced cases may show pink spores.

\subsubsection{Sapburn}
Sapburn appears as dark brown spots or blotches that are often slightly sunken. The damage can occur as runs or streaks down the cheek or as scattered marks around the stem and shoulder, resulting from sap exposure. 

\subsubsection{Skin Browning}
Skin browning may take two forms. Abrasion is recognized as fine brown scratches or rub marks, while sap-related browning appears as light to dark brown flecking, spots, blotches, smears, or rings. These types of browning are generally limited to the skin and do not penetrate deeply.

\subsubsection{Lenticel Spot}
Lenticel spots are another common defect, appearing as round or star-shaped brown spots scattered across the skin surface. These defects are usually cosmetic in nature and do not significantly affect the flesh.

\subsection{Size Determination} \label{sec:sizeDetermination}

To get the size of the mangoes, computer vision techniques such as Gaussian Blur and Thresholding are used to get the length and width of the mangoes.
% will add that the theoretical length and width would be compared to the actual length and width
% also that the predicted size and actual size classification would be shown

\section{Mango Formula with User Priority} \label{sec:formula}
The linear equation used to calculate the Carabao mango grade is shown below. Likewise, the variables \gls{not:bprio}, \gls{not:rprio}, and \gls{not:sprio} represent the user-defined priority weightings for bruising, ripeness, and size characteristics in the \gls{User Priority-Based Grading} system.
Additionally, \gls{not:bpred}, \gls{not:rpred}, and \gls{not:spred} correspond to the machine learning model's predicted values for the bruising, ripeness, and size attributes of the Carabao mango.

\begin{equation}
	\label{eq:userPriorityInput}
	\text{Mango Grade} = \ensuremath{b \left( P \right)  B\left( P \right) + r \left( P \right) R\left( P \right) + s \left( P \right) S\left( P \right)}
\end{equation}

\noindent The machine learning predictions are assigned the following numerical values:

\noindent \textbf{Ripeness Scores:}
\begin{align}
r(\text{yellow}) &= 1.0 \\
r(\text{yellow\_green}) &= 2.0 \\
r(\text{green}) &= 3.0
\end{align}

\noindent \textbf{Bruises Scores:}
\begin{align}
b(\text{bruised}) &= 1.0 \\
b(\text{unbruised}) &= 2.0
\end{align}

\noindent \textbf{Size Scores:}
\begin{align}
s(\text{small}) &= 1.0 \\
s(\text{medium}) &= 2.0 \\
s(\text{large}) &= 3.0
\end{align}


\section{Ethical Considerations}
Ethical considerations ensure that the system is operated safely and responsibly. Data privacy is ensured by securely storing and anonymizing extracted images and classification data so that unauthorized access becomes impossible. The system is also eco-friendly through non-destructive testing, saving mangoes while also ensuring that they are of good quality. Safety in operations is also ensured by protecting moving parts to prevent mechanical harm and incorporating fail-safes to securely stop operation in case of malfunction. Addressing these concerns, the system is not only accurate and efficient but also secure, eco-friendly, and safe for operators, thus a sustainable solution to automated mango sorting and grading.





\section{Summary}
This chapter explained how to create an automatic Carabao mango sorter and grader using machine learning and computer vision. The system integrates hardware and software resources, including a conveyor belt, cameras, sensors, and actuators, to offer accurate, real-time sorting by ripeness, size, and bruises. Various testing and evaluation processes ensure its performance to offer reliability. Ethical issues are data privacy, environmental sustainability, and operation safety. With enhanced efficiency, reduced human error, and enhanced quality, this system provides an affordable, scalable, and non-destructive solution to post-harvest mango classification in agricultural industries.
