
\begin{center}
	{\scriptsize
		\begin{tabularx}{\textwidth}{p{0.2\textwidth}|p{0.6\textwidth}|p{0.1\textwidth}}
			\caption{Summary of methods for reaching the objectives} \label{tab:reaching_objectives} \\
			\hline 
			\hline 
			\textbf{Objectives} & 
			\textbf{Methods} &
			\textbf{Locations}\\ 
			\hline 
			\endfirsthead
			\multicolumn{3}{c}%
			{\textit{Continued from previous page}} \\
			\hline
			\hline 
			\textbf{Objectives} & 
			\textbf{Methods} &
			\textbf{Locations}\\ 
			\hline 
			\endhead
			\hline 
			\multicolumn{3}{r}{\textit{Continued on next page}} \\ 
			\endfoot
			\hline 
			\endlastfoot
			\hline
			
			\Paste{GO} & 
			\begin{enumerate}
        \item Hardware design: Build an image acquisition system with a conveyor belt, \acr{LED} lights, and Raspberry Pi Camera
				\item Software design: Coded a Raspberry Pi application to grade and sort the Carabao mangoes
			\end{enumerate} 
			& Sec.~\ref{sec:researchApproach} on p.~\pageref{sec:researchApproach} \\ \hline
			
			\Paste{SO1} & \begin{enumerate}
				\item Hardware implementation: Design and build an image acquisition system prototype 
			\end{enumerate} & Sec.~\ref{sec:hardwareDesign} on p.~\pageref{sec:hardwareDesign} \\ \hline
			
			\Paste{SO2} & \begin{enumerate}
				\item Performance testing: Train and test the machine learning algorithm for classifying bruises and ripeness
				\item Data collection: Gather our own Carabao mango dataset together with an online dataset
			\end{enumerate} & Sec.~\ref{sec:dataset} on p.~\pageref{sec:trainandtest} \\ \hline
			
			\Paste{SO3} & \begin{enumerate}
				\item Algorithm development: To develop a code for the image acquisition system
				\item Hardware design: To design a schematic for the microcontroller based system 
			\end{enumerate} & Sec.~\ref{sec:hardwareDesign} on p.~\pageref{sec:hardwareDesign} \\ \hline
			
			\Paste{SO4} & \begin{enumerate}
				\item Formula development: Formulated an equation based on the inputted user priority and the predicted mango classification
			\end{enumerate} & Sec.~\ref{sec:formula} on p.~\pageref{sec:formula} \\ \hline
			
			\Paste{SO5} & \begin{enumerate}
				\item Performance testing: Train and test the machine learning algorithm for classifying bruises
			\end{enumerate} & Sec.~\ref{sec:bruisestraining} on p.~\pageref{sec:bruisestraining} \\ \hline
			
			\Paste{SO6} & \begin{enumerate}
				\item Performance testing: Train and test the machine learning algorithm for classifying ripeness
			\end{enumerate} & Sec.~\ref{sec:ripenesstraining} on p.~\pageref{sec:ripenesstraining} \\ \hline
			
			\Paste{SO7} & \begin{enumerate}
				\item Accuracy testing: Get the percent accuracy testing for getting the length and width of the Carabao mango
			\end{enumerate} & Sec.~\ref{sec:sizeDetermination} on p.~\pageref{sec:sizeDetermination} \\ \hline
			
		\end{tabularx}
	}
\end{center}

\section{Introduction}
The methodology for this research outlines the development of the Carabao Mango sorter using machine learning and computer vision. The sorting system uses a conveyor belt system which delivers the mangoes into the image acquisition system. This system captures the image of the mangoes which will then be going through the various stages of image processing and classification into grades which will depend on the priority of the user. This methodology ensures that the grading of the mangoes will be accurate while being non-destructive.

\section{Research Approach} \label{sec:researchApproach}
This study applies the experimental approach for research in order to develop and properly test the proposed system. The experimental approach of the methodology will allow the researchers to fine-tune the parameters and other factors in the classification of mangoes in order to get optimal results with high accuracy scores while maintaining the quality of the mangoes. This approach will also allow for real-time data processing and classification which will improve the previous static grading systems.
To efficiently design and build the prototype, the researchers employed a Scrum agile methodology for managing the two main clusters of the prototype which are the
software and hardware design.

\section{Hardware Design} \label{sec:hardwareDesign}
The prototype consists of hardware and software components for automated mango sorting and grading purposes. The hardware includes the conveyor belt system used to transfer mangoes from scanning to sorting smoothly. A camera and lighting system are able to collect high-resolution images for analysis. The DC motors and stepper motors are responsible for driving the conveyor belt and sorting actuators. The entire system is controlled by a microcontroller \acr{RPi}, coordinating actions of all components. Sorting actuators then direct mangoes into selected bins based on their classification to make sorting efficient. 

\begin{figure}[!htbp]
    \centering
    \subbottom[Left View]{
        \includegraphics[width=0.45\textwidth]{/dimensions/left}
        \label{fig:left-view}
    }
    \hfill
    \subbottom[Top View]{
        \includegraphics[width=0.45\textwidth]{/dimensions/top}
        \label{fig:top-view}
    }
    \vfill
    \subbottom[Right View]{
        \includegraphics[width=0.45\textwidth]{/dimensions/zoom}
        \label{fig:right-view}
    }
    \caption{Image Acquision Dimensions}
    \label{fig:img-acquision-design}
\end{figure}

\section{Software Design} 
For the programming language used for the prototype and training and testing the CNN model, Python was used for training and testing the CNN model and it was also used in the microcontroller to run the application containing the UI and CNN model. PyTorch was the main library used in using the EfficientNet model that is used in classifying the ripeness and bruises of the mango. Likewise, tkinter is the used library when designing the UI in Python.

Furthermore, the rest of the software components are of utmost importance to mango classification. Image processing algorithms in OpenCV and CNN models extract features such as color, size, and bruises that are known to determine quality parameters of mangoes. Mangoes are classified based on ripeness and defects by using machine learning algorithms, which further enhances accuracy using deep learning techniques. A user interface (UI) is designed for users to control and observe the system in real time. Finally, the interface programming of the microcontroller provides the necessary synchronization between sensors, actuators, and motors throughout the sorting operation scenario.
\subsection{Machine Learning Methods}
% [o] todo: Add the explanation that we trained and tested cnn, knn, naive bayes, etc
The processed dataset is to be then used to create models using a variety of
machine learning methods. For a comprehensive evaluation, the processed dataset
was used to train and test a variety of machine learning models. The training
included Convolutional Neural Network (CNN), k-Nearest Neighbors (k-NN), Naive
Bayes, and k-Means clustering and various Efficientnet models. This comparative
analysis was conducted to benchmark the performance of the deep learning
approach against traditional machine learning algorithms.

\subsection{Optimizer}
Choosing the correct optimizer critically impacts both the convergence speed and the
generalization ability of deep neural networks. The widely used \gls{Adam} optimizer
employs adaptive learning rates for each parameter, adjusting them according to the
first- and second-order moments of gradients. However, Adam implements weight decay as
a part of gradient updates, which couples regularization and optimization in a way
that can hamper generalization. AdamW was developed to decouple weight decay from the
adaptive gradient update. Specifically, in AdamW, weight decay is applied directly to the
parameters after the Adam update, leading to improved generalization and often more robust
performance in large-scale tasks. Extensive benchmark comparisons reveal that AdamW outperforms
standard Adam, especially when it comes to image classification or language modeling tasks
with deep architectures \citep{Loshchilov2017Decoupled}.

\subsection{Data Loading Optimization}
Efficient data loading is a vital but often underestimated aspect of deep learning.
In frameworks like PyTorch, the num\_workers parameter of the DataLoader determines how
many subprocesses are used to fetch batches of data in parallel. Setting
num\_workers \textgreater 0 enables multiprocessing, which prefetches batches and keeps the GPU
occupied without idling, especially for large datasets or CPU-intensive augmentations. When
misconfigured, however, the CPU can become a bottleneck, or resource contention may lead
to unexpected slowdowns. The ideal number of workers depends on many factors: CPU and
memory resources, dataset I/O demands, and the complexity of any required preprocessing.
Practically, practitioners start with a low value for num\_workers, gradually increasing while
monitoring CPU utilization and GPU occupancy, always balancing throughput gains against system
constraints \citep{Migacz2020Performance}.
\subsection{Data Transfer Optimization}
Data transfer from host (CPU) to device (GPU) is a significant performance
consideration during training, particularly as model and batch sizes grow. PyTorch
and similar frameworks provide pin\_memory and non\_blocking options to optimize these transfers.
When data is loaded with pin\_memory=True, it is allocated in page-locked (pinned)
memory, which prevents the operating system from swapping it to disk and enables
direct memory access (DMA) from the GPU, reducing latency. Setting non\_blocking=True in
transfer calls further allows these memory copies to be overlapped with computation, eliminating
host-thread blocking and enabling concurrent initiation of multiple transfers. Together, these
settings can cut data transfer times and better exploit GPU concurrency. However, misuse,
such as excessive pinned memory allocation, can reduce overall system stability due to
increased physical memory pressure \citep{Moens2024DataTransfer}.

\subsection{Mixed Precision Training}
Mixed precision training is now a near-standard approach for accelerating deep learning,
especially on modern GPUs equipped with specialized compute units, such as NVIDIA Tensor
Cores, that can handle reduced numerical precision efficiently. By employing 16-bit floating
point (FP16 or BF16) arithmetic for most operations and retaining 32-bit (FP32) precision
for critical accumulations and weight updates, mixed precision training achieves two main
benefits: faster computation throughput and decreased memory footprint. This allows for increased
model or batch sizes and faster experimentation cycles, while, with proper loss scaling,
preserving model convergence and final accuracy \citep{Markidis2018TensorCore}.

\subsection{Adaptive learning Rate Schedulers}
Adaptive learning rate schedules can profoundly affect both convergence speed and the
ability of a model to generalize. The cosine annealing schedule cyclically adjusts the
learning rate from a maximum to a minimum according to a cosine function, periodically
“restarting” back to the initial value. This warm restart strategy prevents the learning
rate from decaying to zero too rapidly and encourages exploration of flatter minima in
the loss surface, thereby enhancing generalization. Cosine annealing with restarts is widely
cited as a simple but effective modification over static or monotonic decay schedules,
giving superior performance across various deep learning domains from computer vision to
language modeling \citep{Loshchilov2016SGDR}.

\subsection{CrossEntropy Loss with Label Smoothing}
Using CrossEntropy loss with label smoothing addressed the issue of overconfidence in
predictions. Standard CrossEntropy encourages the model to assign near‑absolute probability to
the correct class, which can lead to poor generalization, especially when classes are
ambiguous or noisy. Label smoothing redistributes a small fraction of probability mass to
incorrect classes, effectively softening the target distribution. This discourages the model from
becoming overly confident, reduces variance in predictions, and improves robustness against
mislabeled or borderline samples \citep{Guo2024LabelSmoothing, Szegedy2016Rethinking} 

\subsection{Early Stopping and Checkpointing}
Overfitting is a major concern in deep learning, as models with high
capacity can easily memorize the training data without learning to generalize to new
inputs. Early stopping is a widespread technique wherein training is halted when performance
on a held-out validation set ceases to improve, rather than after a fixed number
of epochs. This prevents the model from entering the overfitting regime. Model checkpointing
complements early stopping by routinely saving the model’s parameters and, optionally, optimizer states,
ensuring recoverability in the event of hardware failure and enabling the best-performing model
on validation metrics to be retained, rather than simply the last epoch’s snapshot 
\citep{Hussein2024EarlyStopping, Lee2024CheckQZP}.

\subsection{Input Resolution}
The spatial resolution of input images materially affects both computational cost and
prediction accuracy in deep learning, especially for vision tasks. Higher input resolutions
can theoretically yield better performance, as more visual detail is made available to
the model, but this often comes at the expense of increased memory and higher
training times, sometimes forcing smaller batch sizes and less efficient optimization. Conversely,
reducing input resolution can dramatically decrease resource requirements, permitting faster development
and larger batch sizes, but at a potential loss of accuracy, especially for tasks
that demand fine-grained spatial detail \citep{Richter2020FeatureSpace}.

\subsection{Regularization}
Regularization techniques combat overfitting, and two of the most prominent in
deep learning are dropout and drop path, also called “stochastic depth”. Dropout
randomly deactivates a subset of neurons or weights during each training iteration,
preventing any single unit from becoming indispensable and encouraging redundancy in representation.
Drop path extends this principle by stochastically skipping entire layers or blocks during
training, particularly in architectures with skip connections such as ResNet. This approach
reduces the effective depth of the model during training while maintaining full depth at
inference, acting as an implicit model ensemble and further strengthening generalization 
\citep{Huang2016StochasticDepth}.

\section{Data Collection Methods} \label{sec:dataset}
% The system acquires high-resolution images of mangoes under pre-specified lighting conditions through systematic acquisition. 
% Apart from that, this corpus of data is based on the real-time images acquired from the camera system, where classification 
% operations are carried out based on real-time data. Pre-processing image operations such as flipping, rotating, resizing, 
% normalization, and Gaussian blur are also carried out in order to enhance image clarity and feature detection. Then, 
% the feature extraction process is carried out, where the intensity of color, shape, and texture are analyzed for the 
% detection of characteristic features in terms of the mango. All these aspects lead to the creation of a reliable dataset 
% for the machine learning algorithm that will allow the system to classify and grade mangoes more accurately.

For data collection, publicly available datasets were used along with our own
gathered dataset. to gather the images of the mangoes the setup seen in Figure
~\ref{fig:mangoImageCollection} was used to film the mangoes for about 5 seconds
each side. Using a python script every 20th frame per second was extracted. The
collected images were then sorted into the following directories for use in
training the model: non-bruised, bruised, green, yellow-green, and yellow.


% [o]  todo: ask francis on what software and method did he use to annotate the mangoes
\begin{figure}[!htbp]
    \centering
    \includegraphics[width=0.45\textwidth, angle=-90]{camera_setup}
    \caption{Camera Setup}
    \label{fig:mangoImageCollection}
\end{figure}

\begin{figure}[!htbp]
    \centering
    \subbottom[Boxes of Carabao Mangoes]{
        \includegraphics[width=0.45\textwidth]{/dataset-mangoes/box-mangoes}
        \label{fig:boxes-mangoes}
    }
    \hfill
    \subbottom[Table of Carabao Mangoes]{
        \includegraphics[width=0.45\textwidth]{/dataset-mangoes/table-mangoes}
        \label{fig:table-mangoes}
    }
    \caption{Carabao Mangoes Image Dataset Collection}
    \label{fig:image-collection}
\end{figure}

\begin{figure}[!htbp]
    \centering
    \includegraphics[width=0.45\textwidth]{/dataset-mangoes/sample-frame}
    \caption{Sample Mango Image}
    \label{fig:sample-mango-image}
\end{figure}

For the setup of the captured Carabao mangoes, the height of the camera to the white flat surface is 26 cm which can 
be seen on Figure~\ref{fig:mangoImageCollection}. Furthermore,
the Samsung S24's camera is used for capturing both cheeks of the Carabao mango. Initially, the Carabao mangoes would be unripe and green
and each day the Carabao mangoes would be pictured until they are yellow ripe. Likewise, Figure~\ref{fig:image-collection} shows
the 8 kilogram green Carabao mangoes from the Bicol region. The same mangoes from Bicol are seen on the Figure~\ref{fig:image-collection}.
Note that the mangoes were individually captured one at a time at both cheek sides as a video format which can be seen on Figure~\ref{fig:sample-mango-image}. 

\begin{figure}[!htbp]
    \centering
    \subbottom[Collecting Carabao Mangoes]{
        \includegraphics[width=0.45\textwidth]{farm/me}
        \label{fig:me-collecting}
    }
    \hfill
    \subbottom[Carabao Mango Tree]{
        \includegraphics[width=0.45\textwidth]{farm/tree}
        \label{fig:mango-tree}
    }
    \vfill
    \subbottom[Sack of Carabao Mangoes]{
        \includegraphics[width=0.45\textwidth]{farm/sack}
        \label{fig:sack-mango}
    }

    \caption{Collecting Mango on a Farm}
    \label{fig:farm-of-mine}
\end{figure}

\section{Testing and Evaluation Methods} \label{sec:trainandtest}
In a bid to ensure the mango sorting and grading system is accurate and
reliable, there is intensive testing conducted at different levels. Unit testing
is initially conducted on each component separately, for instance, the conveyor
belt, sensors, and cameras, to ensure that each of the components works as
expected when operating separately. After component testing on an individual
basis, integration testing is conducted to ensure communication between hardware
and software is correct to ensure the image processing system, motors, and
sorting actuators work in concert as required. System testing is conducted to
conduct overall system performance testing in real-world conditions to ensure
mangoes are accurately and efficiently sorted and graded.

% [o] todo put the specs of daniel laptop here for training

For the training everything was done on laptop with the specific model: Ph16-71
acer predator helios 16 2023. The technical specifications of this particular
model I9 13900hx Rtx4070 8gb vram 32gb 5600mhz ddr5 ram

\subsection{Data Augmentation and Splitting}
For the used methods to increase the Carabao mango image dataset, data augmentation techniques such as rotation, flipping, Gaussian blur, brightness adjustment,
noise, crop, and resizing of the images were done.
Note that the split ratio of the dataset is 70-15-15 where it refers to the training, testing, and validation as seen on the Listing~\ref{lst:datasplit_logs}.

The dataset for mango classification was organized into five categories: bruised, not bruised, green, 
yellow-green, and yellow. To ensure robust model training and evaluation, the dataset was initially 
split into training (70\%), validation (15\%), and test (15\%) sets using PyTorch’s automated splitting 
functions. Following standard practice in deep learning (Perez, Wang, 2017), only the training set was 
augmented to increase sample diversity and improve generalization, while 
the validation and test sets remained unaltered to preserve their role as unbiased evaluation benchmarks.

\begin{figure}[!htbp]
    \centering
    \subbottom[Training Dataset]{
        \includegraphics[width=0.45\textwidth]{method/train_ripeness}
        \label{fig:cnn_train_ripe}
    }
    \hfill
    \subbottom[Test Dataset]{
        \includegraphics[width=0.45\textwidth]{method/test_ripeness}
        \label{fig:cnn_test_ripe}
    }
    \vfill
    \subbottom[Validation Dataset]{
        \includegraphics[width=0.45\textwidth]{method/valid_ripeness}
        \label{fig:cnn_valid_ripe}
    }

    \caption{CNN Ripeness 70-15-15 Image Datasplit}
    \label{fig:ripeness_cnn_datasplit}
\end{figure}

\begin{figure}[!htbp]
    \centering
    \subbottom[Training Dataset]{
        \includegraphics[width=0.45\textwidth]{method/train_bruises}
        \label{fig:cnn_train_bruises}
    }
    \hfill
    \subbottom[Test Dataset]{
        \includegraphics[width=0.45\textwidth]{method/test_bruises}
        \label{fig:cnn_test_bruises}
    }
    \vfill
    \subbottom[Validation Dataset]{
        \includegraphics[width=0.45\textwidth]{method/valid_bruises}
        \label{fig:cnn_valid_bruises}
    }

    \caption{CNN Bruises 70-15-15 Image Datasplit}
    \label{fig:bruises_cnn_datasplit}
\end{figure}

The dataset for mango classification was organized into five categories: 
bruised, not bruised, green, yellow-green, and yellow. To ensure robust model
training and evaluation, the dataset was initially split into training (70\%), 
validation (15\%), and test (15\%) sets using PyTorch’s automated splitting 
functions. Following standard practice in deep learning \cite{perez2017effectivenessdataaugmentationimage}
, only the training set was augmented to increase sample diversity and improve 
generalization, while the validation and test sets remained unaltered to 
preserve their role as unbiased evaluation benchmarks.

The validation set contains a balanced representation of the five mango classes. 
In the bruise-based categories shown on Table~\ref{fig:ripeness_cnn_datasplit}, the 
distribution shows slightly more bruised samples ($\sim$260) compared to not bruised 
($\sim$240). On the other hand, for the ripeness-based categories as shown in 
Tables~\ref{fig:bruises_cnn_datasplit}, 
green has the highest count ($\sim$250), followed by yellow-green ($\sim$175), and yellow 
($\sim$125).
This distribution ensures that the validation set provides a fair assessment of the 
model’s performance across both damage-related and ripeness-related classifications.

The test set mirrors the validation set in structure, maintaining proportional 
representation across classes. Approximately 260 bruised samples and ($\sim$240) not 
bruised samples are included as seen in Table~\ref{fig:ripeness_cnn_datasplit}. 
For the ripeness categories seen 
in Table~\ref{fig:bruises_cnn_datasplit}, green ($\sim$225), yellow-green ($\sim$175), and 
yellow ($\sim$125) are represented. This balanced distribution allows for reliable final
evaluation of the trained CNN model, ensuring that results are not biased toward any
single class.


The training set underwent augmentation to artificially expand the dataset and introduce variability. 
Augmentation techniques included transformations such as rotation, flipping, scaling, and brightness 
adjustments. 
After augmentation, the dataset contained approximately 5,100 bruised and 4,900 not
bruised samples as seen in Table~\ref{fig:bruises_cnn_datasplit}. 
For the ripeness categories, green had the highest representation ($\sim$4,200), followed
by yellow-green ($\sim$3,400), and yellow ($\sim$2,600) as seen in Table~\ref{fig:ripeness_cnn_datasplit}. 
This augmentation step increased the training set size substantially, shifting the 
dataset distribution from 70-15-15 to approximately 90-5-5. 
Such a shift is expected, as augmentation only affects the training set, thereby 
increasing its relative proportion.

Augmenting only the training set is a widely accepted best practice in deep learning. 
According to \citet{ShortenKhoshgoftaar2019}, data augmentation enhances model robustness by
simulating real-world variability, but applying it to validation or test sets would artificially
inflate accuracy by exposing the model to transformed versions of already-seen data. 
Similarly, \citet{Goodfellow-et-al-2016} emphasize that evaluation datasets must remain unseen, 
original, and unaltered to provide a true measure of generalization. 
\citet{perez2017effectivenessdataaugmentationimage} further demonstrate that augmentation is
most effective when applied exclusively to training data, as it improves performance without
compromising the integrity of evaluation.

The dataset preparation process therefore ensures that the CNN model is trained on a large,
diverse, and augmented training set, while validation and test sets remain unaltered and
representative. 
This methodology aligns with established best practices in computer vision research,
supporting both robust training and fair evaluation of the mango classification model

\begin{lstlisting}[
float=h,
caption=Datasplit Logs, 
label=lst:datasplit_logs,
language=TeX,
frame=single]
Class Mapping:
------------------------------
green        -> ripeness/green
yellow       -> ripeness/yellow
yellow_green -> ripeness/yellow_green
bruised      -> bruises/bruised
unbruised    -> bruises/not_bruised
Splitting dataset into hierarchical structure...
Processing green -> ripeness/green
  Train: 1225, Val: 262, Test: 263
Processing yellow -> ripeness/yellow
  Train: 616, Val: 132, Test: 132
Processing yellow_green -> ripeness/yellow_green
  Train: 935, Val: 200, Test: 201
Processing bruised -> bruises/bruised
  Train: 1363, Val: 292, Test: 293
Processing unbruised -> bruises/not_bruised
  Train: 1143, Val: 245, Test: 246
Applying massive augmentation to generate 10000 additional images...
Total augmentation combinations available: 309
Original training images: 6832
Total augmented images created: 13664
Target was: 10000

Dataset Statistics:
============================================================

RIPENESS Category:
----------------------------------------
  green        - Train: 7830, Val:  488, Test:  478
  yellow       - Train: 4010, Val:  242, Test:  248
  yellow_green - Train: 6130, Val:  376, Test:  376
  Subtotal     - Train: 17970, Val: 1106, Test: 1102

BRUISES Category:
----------------------------------------
  bruised      - Train: 8820, Val:  526, Test:  538
  not_bruised  - Train: 7370, Val:  446, Test:  450
  Subtotal     - Train: 16190, Val:  972, Test:  988

============================================================
TOTAL        - Train: 34160, Val: 2078, Test: 2090
Ratios       - Train: 89.1%, Val: 5.4%, Test: 5.5%

Dataset processing complete! Output saved to: E:\dir

========================================
\end{lstlisting} 

\subsection{Comparative Test of CNN Models}

To identify the most suitable CNN architecture for 
grading Carabao mangoes, 
multiple CNN models were evaluated under fixed experimental parameters. 
Each model was trained for 15 epochs with an input image size of 224 × 224 pixels, a
batch size of 32, and the \acr{Adam} optimizer set at a learning rate of 0.001.
 Data preprocessing included resizing, normalization using ImageNet mean and standard
 deviation, and augmentation techniques such as random horizontal and vertical flips, 
 random rotations, and Gaussian blur, which were applied exclusively to the 
 training set. 
The validation and test sets remained unaugmented to ensure unbiased evaluation.


The performance of several CNN architectures, including EfficientNetV1, EfficientNetV2,
\acr{VGGNet}, AlexNet, ResNet50, GoogleNet, MobileNetV2, and DenseNet121 was first compared. 
Based on these results, a more detailed comparison was then conducted within the 
EfficientNet family, versions V1 and V2, to determine the most effective variant
for the task.


No advanced optimization techniques such as early stopping, learning rate schedulers,
or mixed precision training were employed. 
This decision was intentional to maintain fairness across all experiments and to ensure
that the only variable factor influencing performance was the network architecture itself. 
Ripeness classification models were trained using a \acr{GPU}, while bruise classification models
were trained on a CPU to compare training times and assess the impact of hardware constraints
on accuracy. 
Model performance was evaluated using precision, recall, F1-score, accuracy, resource
utilization, and elapsed training time.

\subsection{Benchmarking Best CNN Model on +10k Mango Dataset}

As one of the improvements for the final CNN models, the dataset for mango
classification was refined and expanded to improve model robustness and reliability
across both ripeness and bruise detection tasks for the training of the final
CNN model where EfficientNetV2-B3 was used. The data was initially split
into training (70\%), validation (15\%), and test (15\%) sets, with augmentation
applied only to the training set. However, after augmentation, the effective
distribution shifted to 90\% training, 5\% validation, and 5\% test. In addition,
new Carabao mango images were incorporated across all classes to strengthen
representation and improve generalization. As such, to train the final CNN
models, the training set for ripeness category in Table~\ref{fig:cnn_ripeness_info}b
contained 4,900
images of green mangoes, 3,700 images of yellow mangoes, and 5,000 images
of yellow\_green mangoes. For validation in Table~\ref{fig:cnn_ripeness_info}c, 
the set included 200
green mango images, 175 yellow mango images, and 210 yellow\_green images.
The test set in Table~\ref{fig:cnn_ripeness_info}a consisted of 200 green mango images,
160 yellow
mango images, and 220 yellow\_green images. For the bruises category, the
training set Table~\ref{fig:cnn_bruises_info}b contained 6,000 images of
bruised mangoes and 7,000
images of not\_bruised mangoes after augmentation. The validation set 
in Table~\ref{fig:cnn_bruises_info}c
included 200 bruised mango images and 225 not\_bruised mango images,
while the test set as seen in Table~\ref{fig:cnn_bruises_info}a contained 200 bruised 
mango images and 225
not\_bruised mango images. This setup provided a balanced evaluation
framework for the binary classification task, ensuring that both classes were
consistently represented across training, validation, and testing.

The dataset was also cleaned to remove sources of noise and ambiguity.
Images with mixed ripeness features, such as mangoes with both large yellow
and green portions, were placed under yellow\_green instead, while ambiguous
samples, such as yellow mangoes with residual greenish portions, were
excluded to avoid confusing the model. Empty areas present in images were
also removed to ensure that only the fruit itself was used for training.

Augmentation strategies were further refined to preserve class-defining
features. For bruise classification, Gaussian blur was removed since it
obscured critical bruise details. For ripeness classification, brightness and
contrast adjustments were excluded, as these could shift mango colors
between adjacent classes, such as yellow\_green to yellow, introducing
artificial mislabels. Other augmentations, such as rotation, flipping, scaling,
and minor perspective transform, were retained to maintain variability without
compromising class integrity.

Through these improvements, expanded augmentation, inclusion of new
Carabao mango samples, dataset cleaning, and task-specific augmentation
refinements, the final dataset ensured that both CNN models were trained on
high-quality, representative, and diverse data. This preparation supports fair
evaluation on the validation and test sets while maximizing the models’ ability
to generalize to real-world mango classification scenarios.

\begin{figure}[!htbp]
    \centering
    \subbottom[Test Image Size]{
        \includegraphics[width=0.45\textwidth]{/effnetv2-m-bruises/test_class_distribution_bruises}
        \label{fig:bruises_test_cnn}
    }
    \hfill
    \subbottom[Train Image Size]{
        \includegraphics[width=0.45\textwidth]{/effnetv2-m-bruises/train_class_distribution_bruises}
        \label{fig:bruises_train_cnn}
    }
    \vfill
    \subbottom[Validation Image Size]{
        \includegraphics[width=0.45\textwidth]{/effnetv2-m-bruises/valid_class_distribution_bruises}
        \label{fig:bruises_valid_cnn}
    }
    \caption{Bruises Image Datasplit}
    \label{fig:cnn_bruises_info}
\end{figure}

\begin{figure}[!htbp]
    \centering
    \subbottom[Test Image Size]{
        \includegraphics[width=0.45\textwidth]{/effnetv2-m-ripeness/test_class_distribution_ripeness}
        \label{fig:ripeness_test_cnn}
    }
    \hfill
    \subbottom[Train Image Size]{
        \includegraphics[width=0.45\textwidth]{/effnetv2-m-ripeness/train_class_distribution_ripeness}
        \label{fig:ripeness_train_cnn}
    }
    \vfill
    \subbottom[Validation Image Size]{
        \includegraphics[width=0.45\textwidth]{/effnetv2-m-ripeness/valid_class_distribution_ripeness}
        \label{fig:ripeness_valid_cnn}
    }
    \caption{Ripeness Image Datasplit}
    \label{fig:cnn_ripeness_info}
\end{figure}


\subsection{Classification Report}
% describe classification report and how it is used in the system
The classification report provides a detailed summary of the model’s performance
across all output classes by presenting key evaluation metrics such as precision,
recall, F1-score, and support. Precision measures the accuracy of positive predictions,
recall assesses the model’s ability to identify all relevant instances, and
the F1-score represents their harmonic mean, offering a balanced measure of
performance. In this system, the classification report was used to evaluate how
effectively the CNN models identified each mango category—both in ripeness and
bruise detection. By analyzing these metrics, the report helps determine which
class predictions are most accurate and where the model may require further
improvement, ensuring a reliable and interpretable performance assessment for
real-world mango classification.

\subsubsection{Confusion Matrix}

A confusion matrix is a table that visualizes the performance 
of a classification model. For a binary classification problem, it has four
components: \\


\begin{table}[h]
	\centering
	\begin{tabular}{c|c|c}
	\hline
	& \textbf{Predicted Positive} & \textbf{Predicted Negative} \\
	\hline
	\textbf{Actual Positive} & TP & FN \\
	\hline
	\textbf{Actual Negative} & FP & TN \\
	\hline
	\end{tabular}
	\caption{Confusion Matrix Example}
	\label{tab:confusion_matrix}
\end{table}

\begin{itemize}
	\item True Positives (TP): Cases correctly predicted as positive
	\item True Negatives (TN): Cases correctly predicted as negative
	\item False Positives (FP): Cases incorrectly predicted as positive. (Type I error)
	\item False Negatives (FN): Cases incorrectly predicted as negative (Type II error)
\end{itemize}


\subsubsection{Precision}
\begin{eqnarray}
	\text{Precision} = \frac{TP}{TP + FP}
	\label{eq:precision}
\end{eqnarray}

Precision measures how many of the predicted positives are actually positive. It answers the question: 
"When the model predicts the positive class, how often is it correct?" High precision means low false positives.

\subsubsection{Recall}
\begin{eqnarray}
	\text{Recall} = \frac{TP}{TP + FN}
	\label{eq:recall}
\end{eqnarray}

Recall, which is also called sensitivity, measures how many of the actual positives were correctly identified. 
It answers the question: 
"Of all the actual positive cases, how many did the model catch?" High recall means low false negatives.

\subsubsection{F1 Score}
\begin{eqnarray}
	F_1 = 2\times \frac{\text{Precision} \times \text{Recall}}{\text{Precision} + \text{Recall}}
	\label{eq:f1_score}
\end{eqnarray}

The F1 score is the harmonic mean of precision and recall. It provides a single metric that balances 
both concerns. This is particularly useful when you need to 
find a balance between precision and recall, as optimizing for one often decreases the other.

\subsubsection{Accuracy}
\begin{eqnarray}
	\text{Accuracy} = \frac{TP + TN}{TP + TN + FP + FN}
	\label{eq:accuracy}
\end{eqnarray}

Accuracy measures the proportion of correct predictions (both true positives and true negatives)
 among the total cases. While intuitive, accuracy can be misleading with imbalanced datasets.


To test system performance, various measures of performance are used to evaluate. 
As seen on equation~\ref{eq:accuracy}, \gls{accuracy score} is used to measure the percentage of 
correctly classified mangoes to ensure the system maintains high precision levels. 
\gls{Precision} as seen on equation~\ref{eq:precision} and \gls{recall} as seen on equation~\ref{eq:recall} 
are used to measure consistency of classification to determine if the system classifies different ripeness 
levels and defects correctly. Furthermore, the F1 score formula 
as seen on equation~\ref{eq:f1_score} is used to evaluate the performance of the model's classification. 

A \gls{confusion matrix} is used to measure correct and incorrect classification to ensure the machine learning model 
is optimized and that minimum errors are achieved. 
Throughput analysis is also used to determine the rate and efficiency of sorting to 
ensure that the system maintains high capacity without bottlenecks to sort mangoes. 
Using these methods of testing, the system is constantly optimized to ensure high-quality and reliable mango classification.

\subsection{Ripeness Training and Testing} \label{sec:ripenesstraining}

For the testing of the ripeness classification, the Carabao mangoes are classified into three ripeness stages which are Green, green yellow, and yellow.
Likewise, The green would represent the ripe mangoes while the green yellow would represent the semi ripe while the yellow would
represent the ripe mangoes. As reference, Figure~\ref{fig:mangoRipeness} shows the different ripeness stages for Carabao/Pico mangoes \citet{doa2004}.

\begin{figure}[!htbp]
	\centering
	\includegraphics[width=0.7\textwidth]{ripeness_stages}
  \caption{Carabao Mango Ripeness Stages \citep{doa2004}}
	\label{fig:mangoRipeness}
\end{figure}

\subsubsection{Green}
The first classification the researchers selected is the Green stage where the mango's
skin and cheek color is completely light green with no traces of yellow.

\subsubsection{Yellow\_Green}
The second classification is the Yellow\_Green or Green\_Yellow. The main characteristics of this
is that it follows the breaker, turning, and semi-ripe stage of the carabao mango. This means that 
if there is a trace of yellow and green on the skin and cheek of the mango then it is classified as
Yellow\_Green or Green\_Yellow.

\subsubsection{Yellow}
The third and last classification is the Yellow stage where the mango is 80\% to 100\% yellow
on the skin and cheek of the mango. Note that if the mango is overripe then it would be classified 
to be Yellow for ripeness.

\subsection{Bruises Training and Testing} \label{sec:bruisestraining}
For the testing of the bruise classification of the Carabao mangoes, it would classified into two categories which are bruised and not bruised. 
To define what bruise and not bruise mangoes looked like Figure~\ref{fig:mangoDefect} is used as reference to categorize which mangoes are bruised and not bruised.
This means that if the mango has any of these features are shown on the mango then it is considered as bruised.
\begin{figure}[!htbp]
	\centering
	\includegraphics[width=0.7\textwidth]{bruise_guide}
	\caption{Different Kinds of Mango Defects \citep{qld-mango-defect-guide}}
	\label{fig:mangoDefect}
\end{figure}

\subsubsection{Stem End Rots}
They are characterized by fast-growing, watery, soft rots that penetrate deeply into the flesh.  
Likewise, they usually appear as grey-brown or black rots starting from the stem end, often without obvious spores, that can spread rapidly into the mango \citep{Litz2009FruitDiseases,ReviewPostharvestMango2024}.

\subsubsection{Dendritic Spot}
They are small black spots with irregular edges scattered across the skin.  
Furthermore, they grow slowly and do not penetrate into the flesh, remaining largely superficial \citep{HorticultureAustralia2007DendriticSpot}.  

\subsubsection{Anthracnose}
It appears in two forms. First form is body anthracnose.  
Body anthracnose presents as black rots on the fruit surface that are usually round, slightly sunken, and located on different parts of the mango.  
Likewise, the second form is stem end anthracnose, occurring around the stem and also presenting as black rots.  
While these rots do not penetrate deeply into the flesh, advanced cases may show pink spores \citep{Litz2009FruitDiseases,ReviewPostharvestMango2024}.

\subsubsection{Sapburn}
They appear as dark brown spots or blotches that are often slightly sunken.  
Likewise, damage can occur as runs or streaks down the cheek or as scattered marks around the stem and shoulder, resulting from sap exposure \citep{PaullMangoSapburn}.

\subsubsection{Skin Browning}
It may take two forms. The first form is abrasion while the second form is sap browning.  
Abrasion is recognized as fine brown scratches or rub marks, while sap-related browning appears as light to dark brown flecking, spots, blotches, smears, or rings.  
These types of browning are generally limited to the skin and do not penetrate deeply \citep{PaullMangoSapburn}.

\subsubsection{Lenticel Spot}
They are another common defect, appearing as round or star-shaped brown spots scattered across the skin surface.  
Furthermore, these defects are usually cosmetic in nature and do not significantly affect the flesh \citep{NguyenLenticel2015}.

\subsection{Size Determination} \label{sec:sizeDetermination}

To get the size of the mango, computer vision techniques such as Gaussian Blur and Thresholding are used to get the length and width of the mangoes.
Refer to Figure~\ref{fig:lw_mango} for the location of the length and width of mango.
% will add that the theoretical length and width would be compared to the actual length and width
% also that the predicted size and actual size classification would be shown

\begin{figure}[!htbp]
	\centering
	\includegraphics[width=0.5\textwidth]{/sorting/mango_size}
  \caption{Length and Width of Mango \citep{doa2006}}
	\label{fig:lw_mango}
\end{figure}

\subsubsection{Real World Dimensions}
For the real world dimensions method of getting the length and width of the mango a foreground masking is generated by getting absolute difference
between the foreground, that is the mango, and the background. 
Furthermore, image augmentation techniques such as Gaussian blur, grayscale, and Canny edge detection are used. 
After that, the largest contour on the foreground masking image is used. 
Once the largest contour is found then the length and width is calculated using Equation~\ref{eq:objectSizeCalculation2}.

\subsubsection{Object Detection}
For the object detection method, an annotated Carabao mango dataset containing 488 images were used. 
Likewise, the pretrained \acr{Faster R-CNN} model used is the MobileNetV3.

\begin{figure}[!htbp]
	\centering
	\includegraphics[width=0.9\textwidth]{annotated-imgs}
	\caption{Annotated Mango Image Dataset}
	\label{fig:annotated-mangoes}
\end{figure}

\section{Mango Formula with User Priority} \label{sec:formula}
The linear equation used to calculate the Carabao mango grade is shown below. Likewise, the variables \gls{not:bprio}, \gls{not:rprio}, and \gls{not:sprio} represent the user-defined priority weightings for bruising, ripeness, and size characteristics in the \gls{User Priority-Based Grading} system.
Additionally, \gls{not:bpred}, \gls{not:rpred}, and \gls{not:spred} correspond to the machine learning model's predicted values for the bruising, ripeness, and size attributes of the Carabao mango.

\begin{equation}
	\label{eq:userPriorityInput}
	\text{Mango Grade} = \ensuremath{b \left( P \right)  B\left( P \right) + r \left( P \right) R\left( P \right) + s \left( P \right) S\left( P \right)}
\end{equation}

\noindent The machine learning predictions are assigned the following numerical values:

\noindent \textbf{Ripeness Scores:}
\begin{align}
r(\text{yellow}) &= 1.0 \\
r(\text{yellow green}) &= 2.0 \\
r(\text{green}) &= 3.0
\end{align}

\noindent \textbf{Bruises Scores:}
\begin{align}
b(\text{bruised}) &= 1.0 \\
b(\text{not bruised}) &= 2.0
\end{align}

\noindent \textbf{Size Scores:}
\begin{align}
s(\text{small}) &= 1.0 \\
s(\text{medium}) &= 2.0 \\
s(\text{large}) &= 3.0
\end{align}

Note that the scores value for each respective classification cannot be changed
by the user without changing the code itself. This means that only the weight of either the
ripeness, bruises, and size
can be changed to either low, high, or remove it by setting it to zero. Furthermore,
only real numbers are allowed to be inputted as a weight. This means that 
negative and imaginary numbers are not considered in Equation \ref{eq:userPriorityInput}.

% \section{Ethical Considerations}
% Ethical considerations ensure that the system is operated safely and responsibly. Data privacy is ensured by securely storing and anonymizing extracted images and classification data so that unauthorized access becomes impossible. The system is also eco-friendly through non-destructive testing, saving mangoes while also ensuring that they are of good quality. Safety in operations is also ensured by protecting moving parts to prevent mechanical harm and incorporating fail-safes to securely stop operation in case of malfunction. Addressing these concerns, the system is not only accurate and efficient but also secure, eco-friendly, and safe for operators, thus a sustainable solution to automated mango sorting and grading.

\section{Summary} \label{sec:summary_of_methods}

This chapter details the methodology for developing an automated Carabao mango grading
and sorting system integrating machine learning and computer vision. The research employed
an experimental approach managed via Scrum agile methodology to iteratively develop and
test the hardware and software components. The hardware design features a conveyor belt
system, an image acquisition setup with controlled lighting, and a \acr{RPi}
microcontroller coordinating DC motors and sorting actuators. The software, built with
Python and PyTorch, utilizes a custom-trained CNN for
classification. The core machine learning pipeline involved extensive comparative testing
of architectures, including EfficientNet, VGGNet, and ResNet, with EfficientNetV2-B3
ultimately selected for its optimal balance of accuracy and efficiency.

A significant focus was placed on data collection and optimization. A custom
dataset of Carabao mangoes was created by capturing video of individual fruits
and extracting frames, which were then sorted into categories for ripeness (green,
yellow-green, yellow) and bruises (bruised, not bruised). The dataset was split
70-15-15 for training, validation, and testing, with aggressive data augmentation
(rotation, flipping, blur) applied only to the training set to improve
model generalization. The training process incorporated several advanced optimizations:
the AdamW optimizer for better generalization, mixed-precision training to accelerate
computation, data loading and transfer optimizations to prevent bottlenecks, and
regularization techniques like dropout and label smoothing to combat overfitting. A
cosine annealing learning rate scheduler and early stopping were also implemented to
ensure stable convergence.

For system evaluation, the methodology defined specific testing protocols for each
attribute. Ripeness was classified into three visually distinct stages, while bruise
detection was trained to identify defects like stem end rot and anthracnose
based on a standard defect guide. Two methods for size determination were developed
and compared: a traditional computer vision approach using foreground masking and
thresholding, and a more robust object detection method using a Faster R-CNN model
trained on 488 annotated mango images. A key innovation is the user-priority
formula, a weighted equation that allows users to customize the importance of
ripeness, bruises, and size in the final grade (A, B, or
C). 
\\
