
\begin{center}
	{\scriptsize
		\begin{tabularx}{\textwidth}{p{0.2\textwidth}|p{0.6\textwidth}|p{0.1\textwidth}}
			\caption{Summary of methods for achieving the objectives} \label{tab:achieve_objective} \\
			\hline 
			\hline 
			\textbf{Objectives} & 
			\textbf{Methods} & 
			\textbf{Locations} \\ 
			\hline 
			\endfirsthead
			\multicolumn{3}{c}%
			{\textit{Continued from previous page}} \\
			\hline
			\hline 
			\textbf{Objectives} & 
			\textbf{Methods} & 
			\textbf{Locations} \\ 
			\hline 
			\endhead
			\hline 
			\multicolumn{3}{r}{\textit{Continued on next page}} \\ 
			\endfoot
			\hline 
			\endlastfoot
			\hline
			
			\Paste{GO} & 
			Results:
			\begin{enumerate}
				\item Successfully developed a user-priority-based grading and sorting system using machine learning and computer vision which can assess the mangoes’ ripeness, size and bruises.
			\end{enumerate} 
			% Actual Results:
			% \begin{enumerate}
			% 	\item More work needs to be done to fine tune the software components to achieve higher accuracy such as changing hyperparameters or using a newer version of EfficientNet
			% 	\item More work needs to be done to make the hardware component more robust such as by fixing the camera and LED lights in place
			% \end{enumerate} 
			& Sec.~\ref{sec:summary_results_and_discussions} on p.~\pageref{sec:summary_results_and_discussions} \\ \hline
			
			\Paste{SO1} & 
			Results:
			\begin{enumerate}
				\item Successfully integrated a conveyor belt with the image acquisition in order to achieve efficient flow of automated sorting and grading of the mangoes.  
				\item Successfully integrated LED strips to provide optimal lighting for image capturing of the mangoes.
				\item Successfully fixed the hardware components in place
			\end{enumerate} 
			% Actual Results:
			% \begin{enumerate}
			% 	\item Successfully integrated a conveyor belt with the image acquisition in order to achieve efficient flow of automated sorting and grading of the mangoes.  
			% 	\item Successfully integrated LED strips to provide optimal lighting for image capturing of the mangoes.
			% 	\item Need to fix the hardware components in place
			% \end{enumerate} 
			 & Sec.~\ref{sec:physicalPrototype} on p.~\pageref{sec:physicalPrototype} \\ \hline
			
			\Paste{SO2} & 
			Results:
			\begin{enumerate}
				\item Successfully achieved 98\% overall accuracy for ripeness classification of Carabao mangoes
				\item Successfully achieved 99\% overall accuracy for bruises classification of Carabao mangoes
			\end{enumerate} 
			% Results:
			% \begin{enumerate}
			% 	\item Successfully achieved at least 93\% accuracy for ripeness classification of Carabao mangoes
			% 	\item Successfully achieved at least 73\% accuracy for bruise classification of Carabao Mangoes
			% \end{enumerate}
			 & Sec.~\ref{sec:main_trainAndTestResults} on p.~\pageref{sec:main_trainAndTestResults} \\ \hline
			
			\Paste{SO3} & 
			Results:
			\begin{enumerate}
				\item Successfully made a conveyor belt system to move the mangoes through the image acquisition system to the sorting system
				\item Successfully mounted the image acquisition system on the prototype
				\item Successfully made the frame for the conveyor belt and image acquisition system to sit on
			\end{enumerate} 
			% Actual Results:
			% \begin{enumerate}
			% 	\item Successfully made a conveyor belt system to move the mangoes through the image acquisition system to the sorting system
			% 	\item Temporarily mounted the image acquisition system on the the prototype
			% 	\item Successfully made the frame for the conveyor belt and image acquisition system to sit on
			% \end{enumerate}
			 & Sec.~\ref{sec:physicalPrototype} on p.~\pageref{sec:physicalPrototype} \\ \hline
			
			\Paste{SO4} & 
			Results:
			\begin{enumerate}
				\item Successfully grade mangoes based on the user priorities on the physical characteristics of the mango
				\item Successfully verified with qualified individual the results 
				\item Successfully utilize the weighted equation to evaluate mango grade based on user priorities
			\end{enumerate} 
			% Actual Results:
			% \begin{enumerate}
			% 	\item Successfully grade mangoes based on the user priorities on the physical characteristics of the mango
			% 	\item Successfully utilize the weighted equation to evaluate mango grade based on user priorities
			% 	\item Need to look for a qualified person to evaluate the graded mango for ground truth
			% \end{enumerate}
			 & Sec.~\ref{sec:userPriorityFormula} on p.~\pageref{sec:userPriorityFormula} \\ \hline
			
			\Paste{SO5} & 
			Results:
			\begin{enumerate}
        \item Successfully trained a CNN model using EfficientNetV2 and Adam Optimizer for ripeness
				\item Achieved 98\% accuracy on performance metrics using EfficientNetV2
        \item Obtain performance metrics for \acr{KNN}, K-Mean, and Naive Bayes methods for comparison and show the superior performance of using CNN
				\item Successfully fine tuned the CNN model to achieve the highest accuracy possible, choosing the best performing model, and testing other CNN hyperparameters
			\end{enumerate} 
			% Actual Results:
			% \begin{enumerate}
			% 	\item Successfully trained a CNN model using EfficientNet-b0 and Adam Optimizer to detect ripeness based on color 
			% 	\item Successfully achieved at least 90 percent accuracy, precision, recall, f1 score for ripeness classification of Carabao mangoes
			% \end{enumerate}
			 & Sec.~\ref{sec:ripenessClassificationResults} on p.~\pageref{sec:ripenessClassificationResults} \\ \hline
			
			\Paste{SO6} & 
			Results:
			\begin{enumerate}
        \item OpenCV method demonstrated an accurate performance, with measured area percent difference of 4.8\% 
        to the manual measurement by getting its length and width, respectively.
        % Compared and tested two methods, which are the foreground masking then thresholding and Object Detection, to measure the length and width of the Carabao mangoes
        % \item Version 1, which is the foreground masking then thresholding, has a 12.04\% and 3.24\% percent difference to the ground truth for length and width.
        % \item Version 2, which is the Object Detection, has a 13.57\% and 3.24\% percent difference to the ground truth for the length and width.
				% \item Successfully classified mango size using 
				% \item Successfully tuned to have an accurate size with an 80 percent accuracy rating
			\end{enumerate} 
			% Actual Results:
			% \begin{enumerate}
			% 	\item Successfully classified mango size using computer vision techniques
			% 	\item Calculation of mango size is somewhat inaccurate and needs more fine tuning
			% \end{enumerate}
			 & Sec.~\ref{sec:sizeDeterminationResults} on p.~\pageref{sec:sizeDeterminationResults} \\ \hline
			
			\Paste{SO7} & 
			Results:
			\begin{enumerate}
        \item Successfully trained a CNN model using EfficientNetV2 and Adam Optimizer for bruises
				\item Achieved 99\% accuracy on performance metrics
				\item Successfully fine tuned the CNN model to achieve the highest accuracy possible, choosing the best performing, and testing other CNN hyperparameters
			\end{enumerate} 
			% Actual Results:
			% \begin{enumerate}
			% 	\item Successfully trained a CNN model using EfficientNet-b0 and Adam Optimizer to bruises 
			% 	\item Successfully achieved at least 90 percent accuracy, precision, recall, f1 score for bruise classification of Carabao mangoes
			% \end{enumerate}
			 & Sec.~\ref{sec:bruisesClassificationResults} on p.~\pageref{sec:bruisesClassificationResults} \\ \hline
			
		\end{tabularx}
	}
\end{center}

\section{Training and Testing Results of the Model} \label{sec:main_trainAndTestResults}

% To identify the most suitable convolutional neural network (CNN) architecture
% for grading Carabao mangoes, multiple CNN models were evaluated under fixed
% experimental parameters. These parameters included the number of epochs, input
% image size, batch size, optimizer, data preprocessing methods, and dataset size.
% The only variable factor was the network architecture. Ripeness classification
% models were trained using a GPU, while bruises classification models were
% trained on a CPU to compare training times and assess the impact of hardware
% constraints on accuracy. Model performance was evaluated using precision,
% recall, F1-score, accuracy, resource utilization, and elapsed training time


\subsection{Ripeness Classification Results} \label{sec:ripenessClassificationResults}
% Results such as the \gls{F1-Score} are shown.

\subsubsection{Naive Bayes}
% [ ] todo: Add explanation of the confusion matrix and the precision, recall, and f1 score. 
Based on the evaluation metrics, the Naive Bayes model demonstrates a clear
strength in identifying ripe, yellow mangoes but reveals a significant weakness
in classifying those in the transitional yellow-green stage. The model's
precision scores for the green and yellow classes are reasonably similar at
around 79\%. However, its performance drops considerably for the yellow-green
class, where a precision of just 58\% nearly half of its predictions for this
category are incorrect. This pattern is reinforced by the recall scores. The
model excels at finding true yellow mangoes, capturing 86\% of them, which is
its highest performance metric.  Conversely, it struggles to identify
yellow-green mangoes, with a recall of only 51\%, meaning it misses almost half
of all true instances of this class. The F1-score, which balances precision and
recall, provides summary of this performance, yielding a strong score of 80\%
for yellow but a very poor score of 55\% for yellow-green. This confirms that
the transitional yellow-green stage is the model's primary source of confusion,
likely due to its visual ambiguity, sharing features with both the green and
ripe yellow classes.

\begin{table}[htbp]
	\centering
	\begin{tabular}{l|c|c|c|c}
	  \hline \hline
	  \textbf{ } & \textbf{Precision} & \textbf{Recall} & \textbf{F1} & \textbf{Support} \\
	  \hline
	  Green & 0.78 & 0.79 & 0.78 & 132 \\
	  \hline
	  Yellow & 0.75 & 0.86 & 0.80 & 66 \\
	  \hline
	  Yellow\_Green & 0.58 & 0.51 & 0.55 & 101 \\ 
	  \hline
	  Accuracy &  \multicolumn{2}{c|}{ } & 0.71 & 299 \\
	  \hline
	  Macro Avg & 0.70 & 0.72 & 0.71 & 299 \\
	  \hline
	  Weighted Avg & 0.71 & 0.71 & 0.71 & 299 \\
	  \hline
	\end{tabular}
	\caption{Ripeness Classification Report using Naive Bayes}
	\label{tab:bruises_classification_report_naive_bayes}
\end{table}

\begin{figure}[!htbp]
	\centering
	\includegraphics[width=0.5\textwidth]{nb-francis}
	\caption{Ripeness Confusion Matrix using Naive Bayes}
	\label{fig:ripeness_confusion_matrix_nv_fig}
\end{figure}

\pagebreak

\subsubsection{KMeans}
% [ ] todo: Add explanation of the confusion matrix and the precision, recall, and f1 score. 

The KMeans model achieved a weak overall accuracy of 57\%, with its performance
characterized by a severe precision-recall trade-off across classes and a
fundamental failure to identify the transitional stage. The model exhibited high
recall for Green with score of 80\% but low precision of 57\%, which indicates
that it captured most green mangoes but also frequently misclassified others as
green. It was the opposite for Yellow, where high precision score of 83\% and a
low recall score of 52\%, meaning its yellow predictions were reliable but it
missed nearly half of them. Most critically, performance on the Yellow Green
class was exceptionally poor with a F1 score of 34\%, the model struggled both
to correctly label them and to find them at all, this reveals that the clusters
formed by KMeans are poorly separated for this specific ripeness classification
task.

\begin{table}[htbp]
	\centering
	\begin{tabular}{l|c|c|c|c}
		\hline \hline
		\textbf{ } & \textbf{Precision} & \textbf{Recall} & \textbf{F1} & \textbf{Support} \\
		\hline
		Green & 0.57 & 0.80 & 0.67 & 132 \\
		\hline
		Yellow & 0.83 & 0.52 & 0.64 & 66 \\
		\hline
		Yellow\_Green & 0.41 & 0.30 & 0.34 & 101 \\
		\hline
		Accuracy & \multicolumn{2}{c|}{ } & 0.57 & 299 \\
		\hline
		Macro Avg & 0.60 & 0.54 & 0.55 & 299 \\
		\hline
		Weighted Avg & 0.57 & 0.57 & 0.55 & 299 \\
		\hline
	\end{tabular}
	\caption{Ripeness Classification Report using KMeans}
	\label{tab:bruises_classification_report_kmeans}
\end{table}

\begin{figure}[!htbp]
	\centering
	\includegraphics[width=0.5\textwidth]{kmean-francis}
	\caption{Ripeness Confusion Matrix using KMeans}
	\label{fig:ripeness_confusion_matrix_kmeans_fig}
\end{figure}

\subsubsection{KNN}
% [o] todo: Add explanation of the confusion matrix and the precision, recall, and f1 score. 

K-Nearest Neighbors (KNN) model demonstrates an improvement in performance,
achieving an overall accuracy of 78\%. Unlike previous models, KNN shows a
strong and consistent balance between precision and recall across all three
ripeness classes. The model excels at classifying the fully Green and Yellow
stages, with high and well-balanced F1-scores of 0.85 and 0.81, respectively,
indicating it is more reliable when making a prediction and effective at
identifying all instances of these classes than previous models.  KNN also shows
improvement in handling the yellow-green class, achieving an F1-score of 68\%.
While this remains the most challenging class, the model's significantly higher
scores compared to previous attempts confirm its ability to learn the
distinguishing features between the stages.

\begin{table}[htbp]
	\centering
	\begin{tabular}{l|c|c|c|c}
		\hline \hline
		\textbf{ } & \textbf{Precision} & \textbf{Recall} & \textbf{F1} & \textbf{Support} \\
		\hline
		Green & 0.85 & 0.85 & 0.85 & 132 \\
		\hline
		Yellow & 0.83 & 0.79 & 0.81 & 66 \\
		\hline
		Yellow\_Green & 0.67 & 0.69 & 0.68 & 101 \\
		\hline
		Accuracy & \multicolumn{2}{c|}{ }  & 0.78 & 299 \\
		\hline
		Macro Avg & 0.78 & 0.78 & 0.78 & 299 \\
		\hline
		Weighted Avg & 0.78 & 0.78 & 0.78 & 299 \\
		\hline
	\end{tabular}
	\caption{Ripeness Classification Report using KNN}
	\label{tab:bruises_classification_report_knn}
\end{table}

\begin{figure}[!htbp]
	\centering
	\includegraphics[width=0.5\textwidth]{knn-francis}
	\caption{Ripeness Confusion Matrix using KNN}
	\label{fig:ripeness_confusion_matrix_knn_fig}
\end{figure}

\section{Achieving the Highest Accuracy in CNN Models}

\begin{table}[htbp]
\centering
\begin{tabular}{l|c|c|c|c|c|c|c}
\hline \hline
\textbf{Network} & \textbf{Prec} & \textbf{Rec} & \textbf{F1} & \textbf{Test Acc} & \textbf{Train Acc} & \textbf{Time} & \textbf{VRAM} \\
\hline
VGG16 & 0.188 & 0.434 & 0.263 & 43 & 43.57 & 2h57m & 7.0 \\
\hline
ALEXNET & 0.188 & 0.434 & 0.263 & 43 & 43.57 & 4h23m & 2.3 \\
\hline
RESNET50 & 0.870 & 0.869 & 0.868 & 87 & 89.22 & 7h13m & 4.1 \\
\hline
GOOGLENET & 0.898 & 0.895 & 0.892 & 89 & 83.58 & 3h3m & 2.9 \\
\hline
MOBILENETV2 & 0.898 & 0.898 & 0.897 & 90 & 91.13 & 2h0m & 3.6 \\
\hline
DENSENET121 & 0.877 & 0.877 & 0.875 & 88 & 89.17 & 2h10m & 5.5 \\
\hline
EFFNET B0 & 0.890 & 0.888 & 0.887 & 89 & 91.24 & 2h14m & 4.1 \\
\hline
EFFNET B1 & 0.916 & 0.913 & 0.913 & 91 & 89.91 & 2h25m & 5.3 \\
\hline
EFFNET B2 & 0.906 & 0.902 & 0.900 & 90 & 89.46 & 2h26m & 5.5 \\
\hline
EFFNET B3 & 0.914 & 0.911 & 0.909 & 91 & 89.72 & 2h30m & 6.8 \\
\hline
EFFNET B4 & 0.899 & 0.898 & 0.896 & 90 & 92.34 & 2h50m & 8.0 \\
\hline
EFFNET B5 & 0.925 & 0.924 & 0.924 & 92 & 94.12 & 5h45m & 11.6 \\
\hline
EFFNET B6 & 0.934 & 0.933 & 0.933 & 93 & 96.03 & 7h12m & 14.5 \\
\hline
EFFNET B7 & 0.883 & 0.871 & 0.873 & 87 & 90.82 & 9h9m & 18.8 \\
\hline
EFFNETV2-B0 & 0.915 & 0.913 & 0.913 & 91 & 92.71 & 1h53m & 3.0 \\
\hline
EFFNETV2-B1 & 0.920 & 0.918 & 0.919 & 92 & 92.65 & 1h59m & 3.7 \\
\hline
EFFNETV2-B2 & 0.920 & 0.920 & 0.920 & 92 & 92.34 & 2h0m & 3.8 \\
\hline
EFFNETV2-B3 & 0.926 & 0.926 & 0.925 & 93 & 93.97 & 2h2m & 4.5 \\
\hline
EFFNETV2-S & 0.894 & 0.893 & 0.891 & 89 & 90.47 & 2h17m & 6.1 \\
\hline
EFFNETV2-M & 0.893 & 0.893 & 0.892 & 89 & 90.02 & 2h37m & 9.9 \\
\hline
EFFNETV2-L & 0.875 & 0.871 & 0.870 & 87 & 89.93 & 13h39m & 16.8 \\
\hline
\textbf{AVERAGE} & \textbf{0.835} & \textbf{0.856} & \textbf{0.839} & \textbf{86} & \textbf{85.52} & \textbf{--} & \textbf{7.0} \\
\hline
\end{tabular}
\caption{CNN Training Results for Ripeness using GPU}
\label{tab:more_cnn_training_gpu}
\end{table}

% \begin{table}[htbp]
% \centering
% \begin{tabular}{l|c|c|c|c|c|c|c}
% \hline \hline
% \textbf{Network} & \textbf{Prec} & \textbf{Rec} & \textbf{F1} & \textbf{Test Acc} & \textbf{Train Acc} & \textbf{Time} & \textbf{VRAM} \\
% \hline
% VGG16 & 0.188 & 0.434 & 0.263 & 43 & 43.57 & 2h57m & 7.0 \\
% \hline
% ALEXNET & 0.188 & 0.434 & 0.263 & 43 & 43.57 & 2h43m & 2.3 \\
% \hline
% RESNET50 & 0.870 & 0.869 & 0.868 & 87 & 87.58 & 7h13m & 4.1 \\
% \hline
% GOOGLENET & 0.898 & 0.898 & 0.898 & 83 & 83.58 & 3h3m & 4.9 \\
% \hline
% MOBILENETV2 & 0.898 & 0.898 & 0.898 & 89 & 89.47 & 2h0m & 3.6 \\
% \hline
% DENSENET121 & 0.877 & 0.877 & 0.877 & 88 & 91.37 & 2h10m & 5.5 \\
% \hline
% EFFNET B0 & 0.889 & 0.888 & 0.887 & 89 & 91.47 & 2h26m & 5.1 \\
% \hline
% EFFNET B1 & 0.867 & 0.862 & 0.861 & 86 & 90.47 & 2h30m & 5.3 \\
% \hline
% EFFNET B2 & 0.927 & 0.927 & 0.927 & 93 & 92.47 & 2h14m & 5.5 \\
% \hline
% EFFNET B3 & 0.882 & 0.880 & 0.879 & 88 & 90.24 & 2h25m & 6.8 \\
% \hline
% EFFNET B4 & 0.899 & 0.898 & 0.894 & 92 & 92.34 & 2h50m & 8.0 \\
% \hline
% EFFNET B5 & 0.925 & 0.924 & 0.924 & 93 & 92.34 & 5h49m & 12.2 \\
% \hline
% EFFNET B6 & 0.934 & 0.933 & 0.933 & 93 & 90.34 & 7h51m & 14.5 \\
% \hline
% EFFNET B7 & 0.883 & 0.871 & 0.873 & 87 & 90.47 & 10h34m & 18.0 \\
% \hline
% EFFNETV2-B0 & 0.915 & 0.913 & 0.913 & 91 & 89.91 & 1h53m & 3.0 \\
% \hline
% EFFNETV2-B1 & 0.904 & 0.904 & 0.904 & 90 & 92.56 & 2h0m & 3.6 \\
% \hline
% EFFNETV2-B2 & 0.920 & 0.888 & 0.902 & 92 & 92.34 & 2h17m & 3.8 \\
% \hline
% EFFNETV2-B3 & 0.926 & 0.926 & 0.925 & 92 & 92.91 & 2h2m & 4.5 \\
% \hline
% EFFNETV2-S & 0.894 & 0.893 & 0.891 & 89 & 90.47 & 3h15m & 6.1 \\
% \hline
% EFFNETV2-M & 0.893 & 0.893 & 0.893 & 89 & 89.09 & 3h5m & 9.9 \\
% \hline
% EFFNETV2-L & 0.875 & 0.871 & 0.871 & 87 & 89.09 & 14h58m & 16.8 \\
% \hline
% \textbf{AVERAGE} & \textbf{0.830} & \textbf{0.852} & \textbf{0.835} & \textbf{85} & \textbf{85.52} & \textbf{--} & \textbf{7.0} \\
% \hline
% \end{tabular}
% \caption{CNN Training Results for GPU}
% \label{tab:more_cnn_training_gpu}
% \end{table}

% \begin{table}[htbp]
% \centering
% \begin{tabular}{l|c|c|c|c|c|c}
% \hline \hline
% \textbf{Network} & \textbf{Prec} & \textbf{Rec} & \textbf{F1} & \textbf{Acc} & \textbf{Time} & \textbf{Mem} \\
% \hline
% VGG16 & 0.297 & 0.545 & 0.384 & 54 & 5h38m & 6.5 \\
% \hline
% ALEXNET & 0.297 & 0.545 & 0.384 & 54 & 4h25m & 3.3 \\
% \hline
% RESNET50 & 0.858 & 0.844 & 0.844 & 84 & 8h24m & 5.4 \\
% \hline
% GOOGLENET & 0.843 & 0.808 & 0.799 & 81 & 3h14m & 4.0 \\
% \hline
% MOBILENETV2 & 0.859 & 0.858 & 0.858 & 86 & 3h44m & 4.8 \\
% \hline
% DENSENET121 & 0.839 & 0.838 & 0.838 & 84 & 3h8m & 6.7 \\
% \hline
% EFFNET B0 & 0.873 & 0.870 & 0.870 & 87 & 2h37m & 5.3 \\
% \hline
% EFFNET B1 & 0.898 & 0.897 & 0.896 & 90 & 3h8m & 6.7 \\
% \hline
% EFFNET B2 & 0.901 & 0.901 & 0.901 & 90 & 2h56m & 6.7 \\
% \hline
% EFFNET B3 & 0.913 & 0.913 & 0.913 & 91 & 3h27m & 8.0 \\
% \hline
% EFFNET B4 & 0.897 & 0.897 & 0.897 & 90 & 4h17m & 9.8 \\
% \hline
% EFFNET B5 & 0.892 & 0.883 & 0.881 & 88 & 5h45m & 12.2 \\
% \hline
% EFFNET B6 & 0.884 & 0.883 & 0.882 & 88 & 7h12m & 14.5 \\
% \hline
% EFFNET B7 & 0.857 & 0.856 & 0.856 & 86 & 9h9m & 18.0 \\
% \hline
% EFFNETV2-B0 & 0.873 & 0.860 & 0.858 & 86 & 2h6m & 4.4 \\
% \hline
% EFFNETV2-B1 & 0.893 & 0.893 & 0.893 & 89 & 3h30m & 5.1 \\
% \hline
% EFFNETV2-B2 & 0.888 & 0.881 & 0.881 & 88 & 2h45m & 5.4 \\
% \hline
% EFFNETV2-B3 & 0.905 & 0.905 & 0.905 & 90 & 2h32m & 6.2 \\
% \hline
% EFFNETV2-S & 0.860 & 0.836 & 0.831 & 84 & 2h55m & 7.7 \\
% \hline
% EFFNETV2-M & 0.856 & 0.846 & 0.846 & 85 & 2h37m & 9.3 \\
% \hline
% EFFNETV2-L & 0.849 & 0.836 & 0.836 & 84 & 13h39m & 17.9 \\
% \hline
% \textbf{AVERAGE} & \textbf{0.820} & \textbf{0.838} & \textbf{0.821} & \textbf{84} & \textbf{--} & \textbf{8.0} \\
% \hline
% \end{tabular}
% \caption{CNN Ripeness Results for CPU}
% \label{tab:more_cnn_training_cpu}
% \end{table}

\begin{table}[htbp]
\centering
\begin{tabular}{l|c|c|c|c|c|c|c}
\hline \hline
\textbf{Network} & \textbf{Prec} & \textbf{Rec} & \textbf{F1} & \textbf{Test Acc} & \textbf{Train Acc} & \textbf{Time} & \textbf{Mem} \\
\hline
VGG16 & 0.297 & 0.545 & 0.384 & 54 & 54.48 & 5h38m & 6.5 \\
\hline
ALEXNET & 0.297 & 0.545 & 0.384 & 54 & 54.48 & 4h25m & 3.3 \\
\hline
RESNET50 & 0.858 & 0.844 & 0.844 & 84 & 83.92 & 8h24m & 5.4 \\
\hline
GOOGLENET & 0.843 & 0.808 & 0.799 & 81 & 57.67 & 3h14m & 4.0 \\
\hline
MOBILENETV2 & 0.859 & 0.858 & 0.858 & 86 & 85.88 & 3h44m & 4.8 \\
\hline
DENSENET121 & 0.839 & 0.838 & 0.838 & 84 & 84.7 & 3h8m & 6.7 \\
\hline
EFFNET B0 & 0.873 & 0.870 & 0.870 & 87 & 90.04 & 2h37m & 5.3 \\
\hline
EFFNET B1 & 0.898 & 0.897 & 0.896 & 90 & 90.51 & 2h56m & 6.7 \\
\hline
EFFNET B2 & 0.901 & 0.901 & 0.901 & 90 & 91.21 & 3h8m & 6.7 \\
\hline
EFFNET B3 & 0.913 & 0.913 & 0.913 & 91 & 90.34 & 3h27m & 8.0 \\
\hline
EFFNET B4 & 0.897 & 0.897 & 0.897 & 90 & 92.16 & 4h17m & 9.8 \\
\hline
EFFNET B5 & 0.892 & 0.883 & 0.881 & 88 & 90.53 & 5h49m & 12.2 \\
\hline
EFFNET B6 & 0.884 & 0.883 & 0.882 & 88 & 90.43 & 7h51m & 14.5 \\
\hline
EFFNET B7 & 0.857 & 0.856 & 0.856 & 86 & 90.47 & 10h34m & 18.0 \\
\hline
EFFNETV2-B0 & 0.880 & 0.879 & 0.878 & 88 & 90.69 & 2h6m & 4.4 \\
\hline
EFFNETV2-B1 & 0.893 & 0.893 & 0.893 & 89 & 91.72 & 2h32m & 5.1 \\
\hline
EFFNETV2-B2 & 0.904 & 0.889 & 0.889 & 89 & 88.16 & 2h45m & 5.4 \\
\hline
EFFNETV2-B3 & 0.919 & 0.919 & 0.919 & 92 & 94.46 & 2h55m & 6.2 \\
\hline
EFFNETV2-S & 0.859 & 0.858 & 0.858 & 86 & 86.58 & 2h58m & 7.7 \\
\hline
EFFNETV2-M & 0.856 & 0.846 & 0.846 & 85 & 84.74 & 3h30m & 9.3 \\
\hline
EFFNETV2-L & 0.849 & 0.836 & 0.836 & 84 & 85.05 & 14h58m & 17.9 \\
\hline
\textbf{AVERAGE} & \textbf{0.822} & \textbf{0.841} & \textbf{0.825} & \textbf{84.10} & \textbf{--} & \textbf{--} & \textbf{8.0} \\
\hline
\end{tabular}
\caption{CNN Bruises Results for bruises using CPU }
\label{tab:more_cnn_training_cpu}
\end{table}

% \begin{table}[htbp]
% \centering
% \begin{tabular}{l|c|c|c|c|c|c|c}
% \hline \hline
% \textbf{Network} & \textbf{Prec} & \textbf{Rec} & \textbf{F1} & \textbf{Test Acc} & \textbf{Train Acc} & \textbf{Time} & \textbf{Mem} \\
% \hline
% VGG16 & 0.297 & 0.545 & 0.384 & 54 & 54.48 & 5h38m & 6.5 \\
% \hline
% ALEXNET & 0.297 & 0.545 & 0.384 & 54 & 54.48 & 4h25m & 3.3 \\
% \hline
% RESNET50 & 0.858 & 0.844 & 0.844 & 84 & 83.92 & 8h24m & 5.4 \\
% \hline
% GOOGLENET & 0.843 & 0.808 & 0.799 & 81 & 57.67 & 3h14m & 4.0 \\
% \hline
% MOBILENETV2 & 0.859 & 0.858 & 0.858 & 86 & 85.88 & 3h44m & 4.8 \\
% \hline
% DENSENET121 & 0.839 & 0.838 & 0.838 & 84 & 84.7 & 3h8m & 6.7 \\
% \hline
% EFFNET B0 & 0.873 & 0.870 & 0.870 & 87 & 90.04 & 2h37m & 5.3 \\
% \hline
% EFFNET B1 & 0.898 & 0.897 & 0.896 & 90 & 90.51 & 2h56m & 6.7 \\
% \hline
% EFFNET B2 & 0.901 & 0.901 & 0.901 & 90 & 91.21 & 3h8m & 6.7 \\
% \hline
% EFFNET B3 & 0.913 & 0.913 & 0.913 & 91 & 90.34 & 3h27m & 8.0 \\
% \hline
% EFFNET B4 & 0.897 & 0.897 & 0.897 & 90 & 92.16 & 4h17m & 9.8 \\
% \hline
% EFFNET B5 & 0.892 & 0.883 & 0.881 & 88 & 90.53 & 5h49m & 12.2 \\
% \hline
% EFFNET B6 & 0.884 & 0.883 & 0.882 & 88 & 90.43 & 7h51m & 14.5 \\
% \hline
% EFFNET B7 & 0.857 & 0.856 & 0.856 & 86 & 90.47 & 10h34m & 18.0 \\
% \hline
% EFFNETV2-B0 & 0.873 & 0.860 & 0.858 & 86 & 90.69 & 2h6m & 4.4 \\
% \hline
% EFFNETV2-B1 & 0.893 & 0.893 & 0.893 & 89 & 91.72 & 2h32m & 5.1 \\
% \hline
% EFFNETV2-B2 & 0.888 & 0.881 & 0.881 & 88 & 88.16 & 2h45m & 5.4 \\
% \hline
% EFFNETV2-B3 & 0.905 & 0.905 & 0.905 & 90 & 94.46 & 2h55m & 6.2 \\
% \hline
% EFFNETV2-S & 0.860 & 0.836 & 0.831 & 84 & 86.58 & 2h58m & 7.7 \\
% \hline
% EFFNETV2-M & 0.856 & 0.846 & 0.846 & 85 & 84.74 & 3h30m & 9.3 \\
% \hline
% EFFNETV2-L & 0.849 & 0.836 & 0.836 & 84 & 85.05 & 14h58m & 17.9 \\
% \hline
% \textbf{AVERAGE} & \textbf{0.820} & \textbf{0.838} & \textbf{0.821} & \textbf{84.10} & \textbf{--} & \textbf{--} & \textbf{8.0} \\
% \hline
% \end{tabular}
% \caption{CNN Bruises Results for CPU }
% \label{tab:more_cnn_training_cpu}
% \end{table}

% % Table 2: CNN Training Results for CPU
% \begin{table}[htbp]
% \centering
% \begin{tabular}{l|c|c|c|c|c|c}
% \hline \hline
% \textbf{Network} & \textbf{Prec} & \textbf{Rec} & \textbf{F1} & \textbf{Acc} & \textbf{Time} & \textbf{Mem} \\
% \hline
% VGG16 & 0.297 & 0.545 & 0.384 & 54 & 5h38m & 6.5 \\
% \hline
% ALEXNET & 0.297 & 0.545 & 0.384 & 54 & 4h25m & 3.3 \\
% \hline
% RESNET50 & 0.858 & 0.844 & 0.844 & 84 & 8h24m & 5.4 \\
% \hline
% GOOGLENET & 0.843 & 0.808 & 0.799 & 81 & 3h14m & 4.0 \\
% \hline
% MOBILENETV2 & 0.859 & 0.858 & 0.858 & 86 & 3h44m & 4.8 \\
% \hline
% DENSENET121 & 0.839 & 0.838 & 0.838 & 84 & 3h8m & 6.7 \\
% \hline
% EFFICIENTNET B0 & 0.873 & 0.870 & 0.870 & 87 & 2h37m & 5.3 \\
% \hline
% EFFICIENTNET B1 & 0.898 & 0.897 & 0.896 & 90 & 3h8m & 6.7 \\
% \hline
% EFFICIENTNET B2 & 0.901 & 0.901 & 0.901 & 90 & 2h56m & 6.7 \\
% \hline
% EFFICIENTNET B3 & 0.913 & 0.913 & 0.913 & 91 & 3h27m & 8.0 \\
% \hline
% EFFICIENTNET B4 & 0.897 & 0.897 & 0.897 & 90 & 4h17m & 9.8 \\
% \hline
% EFFICIENTNET B5 & 0.892 & 0.883 & 0.881 & 88 & 5h45m & 12.2 \\
% \hline
% EFFICIENTNET B6 & 0.884 & 0.883 & 0.882 & 88 & 7h12m & 14.5 \\
% \hline
% EFFICIENTNET B7 & 0.857 & 0.856 & 0.856 & 86 & 9h9m & 18.0 \\
% \hline
% EFFICIENTNETV2-B0 & 0.873 & 0.860 & 0.858 & 86 & 2h6m & 4.4 \\
% \hline
% EFFICIENTNETV2-B1 & 0.893 & 0.893 & 0.893 & 89 & 3h30m & 5.1 \\
% \hline
% EFFICIENTNETV2-B2 & 0.888 & 0.881 & 0.881 & 88 & 2h45m & 5.4 \\
% \hline
% EFFICIENTNETV2-B3 & 0.905 & 0.905 & 0.905 & 90 & 2h32m & 6.2 \\
% \hline
% EFFICIENTNETV2-S & 0.860 & 0.836 & 0.831 & 84 & 2h55m & 7.7 \\
% \hline
% EFFICIENTNETV2-M & 0.856 & 0.846 & 0.846 & 85 & 2h37m & 9.3 \\
% \hline
% EFFICIENTNETV2-L & 0.849 & 0.836 & 0.836 & 84 & 13h39m & 17.9 \\
% \hline
% \textbf{AVERAGE} & \textbf{0.820} & \textbf{0.838} & \textbf{0.821} & \textbf{84} & \textbf{--} & \textbf{8.0} \\
% \hline
%
% \end{tabular}
% \caption{More CNN Training Results for CPU}
% \label{tab:more_cnn_training_cpu}
% \end{table}

\subsection{Analyzing the Accuracy of Different CNN Network}

\begin{table}[htbp]
  \centering
  \begin{tabular}{l|cc}
  \hline \hline
                & \multicolumn{2}{c}{Accuracy}            \\ \hline
  Model          & \multicolumn{1}{c|}{Ripeness} & Bruises \\ \hline
  EfficientNetB0 & \multicolumn{1}{c|}{89\%}       & 87\%     \\ \hline
  EfficientNetB2 & \multicolumn{1}{c|}{92\%}       & 90\%     \\ \hline
  VggNet16       & \multicolumn{1}{c|}{43\%}       & 54\%      \\ \hline
  AlexNet        & \multicolumn{1}{c|}{43\%}       & 54\%      \\ \hline
  \acr{ResNet}50       & \multicolumn{1}{c|}{87\%}       & 84\%      \\ \hline
  GoogleNet      & \multicolumn{1}{c|}{89\%}       & 81\%      \\ \hline
  MobileNetV2    & \multicolumn{1}{c|}{90\%}       & 86\%      \\ \hline
  DenseNet121    & \multicolumn{1}{c|}{88\%}       & 84\%      \\ \hline
  \end{tabular}
  \caption{Accuracy of Different CNN Models}
  \label{tab:overall_accuracy_cnn}
\end{table}

\begin{table}[htbp]
  \centering
  \begin{tabular}{l|cc}
  \hline \hline
                & \multicolumn{2}{c}{Test Accuracy}            \\ \hline
  EfficientNet & \multicolumn{1}{c|}{Ripeness} & Bruises \\ \hline
  B0 & \multicolumn{1}{c|}{89\%}       & 87\%      \\ \hline
  B1 & \multicolumn{1}{c|}{86\%}       & 90\%      \\ \hline
  B2 & \multicolumn{1}{c|}{92\%}       & 90\%      \\ \hline
  B3 & \multicolumn{1}{c|}{88\%}       & 91\%      \\ \hline
  B4 & \multicolumn{1}{c|}{90\%}       & 90\%      \\ \hline
  B5 & \multicolumn{1}{c|}{92\%}       & 88\%      \\ \hline
  B6 & \multicolumn{1}{c|}{93\%}       & 88\%      \\ \hline
  V2B0 & \multicolumn{1}{c|}{91\%}       & 88\%      \\ \hline
  V2B1 & \multicolumn{1}{c|}{92\%}       & 89\%      \\ \hline
  V2B2 & \multicolumn{1}{c|}{92\%}       & 89\%      \\ \hline
  V2B3 & \multicolumn{1}{c|}{93\%}       & 92\%      \\ \hline
  V2-S & \multicolumn{1}{c|}{89\%}       & 86\%      \\ \hline
  V2-M & \multicolumn{1}{c|}{89\%}       & 85\%      \\ \hline
  V2-L & \multicolumn{1}{c|}{89\%}       & 84\%      \\ \hline
  \end{tabular}
  \caption{Test Accuracy of Different EfficientNet Version 1 and 2}
  \label{tab:overall_accuracy_effnet}
\end{table}

For the classification of ripeness, the highest accuracy was obtained
with EfficientNetV2-B0, which achieved 91\%. This was followed by
MobileNetV2, which achieved 90\%, EfficientNet-B0 and GoogLeNet at 89\%,
DenseNet121 at 88\%, and \acr{ResNet}50 at 87\%. In contrast, both VGGNet16 and
AlexNet severely underperformed, each reaching only 43\% accuracy. A closer
inspection of their classification reports revealed that these two models
predicted only the green class across all test samples, completely failing
to recognize yellow and yellow\_green. This explains why their accuracy
plateaued at 43\%, a value that directly corresponds to the proportion of
green samples in the dataset. The collapse into a single-class prediction
highlights the limitations of these older architectures: AlexNet and
VGGNet16 lack the advanced feature extraction and efficient feature reuse
mechanisms present in modern CNNs, making them less capable of capturing
the subtle hue and texture variations that distinguish ripeness stages
\citep{Krizhevsky2012} \citep{simonyan2015}. AlexNet, while revolutionary
in 2012, was designed for large-scale but relatively coarse ImageNet
classification and relies on shallow convolutional layers with large
receptive fields, which limits its ability to capture fine-grained
differences. Similarly, VGGNet16, though deeper, uses very uniform 3×3
convolutions without skip connections or dense connectivity, leading to
redundancy and inefficient feature reuse, which modern architectures have
since addressed. Furthermore, the training setup and hyperparameters, which
favored faster convergence in lightweight and well-optimized models such as
MobileNetV2 and EfficientNet \citep{howard2017}  \citep{tan2019}, did not
provide the same benefit to AlexNet and VGGNet16 \citep{Huang2017}.
Importantly, the train accuracy values further reinforce these findings
where modern architectures such as EfficientNetV2-B3 (93\% train, 93\%
test) and EfficientNet-B6 (96\% train, 93\% test) maintained close
alignment between training and test performance, indicating strong
generalization. In contrast, AlexNet and VGGNet16 stagnated at 43\% for
both training and test accuracy, indicating that they were underfitting and
unable to capture the discriminative features necessary for ripeness
classification. From a performance requirements perspective, the results
also demonstrate that modern architectures not only achieved higher
accuracy but did so with significantly lower training times and more
efficient VRAM utilization. For instance, EfficientNetV2-B0 reached the
highest accuracy in under two hours with an average VRAM usage of only 3
GB, while AlexNet required over four hours yet produced poor results, and
VGGNet16 consumed the highest VRAM (7 GB) despite its low accuracy. This
efficiency–accuracy balance makes modern CNNs far more suitable for
practical deployment in ripeness classification tasks, where both
computational cost and predictive reliability are critical.

For the classification of bruises, the highest accuracy was obtained with
EfficientNetV2-B0, which achieved 88\%. This was followed by EfficientNet-B0
at 87\% and MobileNetV2 at 86\%. \acr{ResNet} and DenseNet121 both reached 84\%,
while GoogLeNet trailed slightly at 81\%. In contrast, both VGG16 and AlexNet
severely underperformed, each plateauing at only 54\% accuracy. Similar to the
results from training ripeness, VGG16 and AlexNet collapsed into underfitting,
where both models produced very low precision (0.2965) and F1-scores (0.384),
and their training accuracy stagnated at the same 54\%, confirming their
inability to learn discriminative features. By contrast, modern architectures
such as EfficientNet and MobileNetV2 leverage depthwise separable convolutions,
compound scaling, and optimized feature reuse, enabling them to achieve higher
accuracy with fewer parameters and faster convergence. EfficientNetV2-B0 not
only achieved the highest accuracy (88\%) but also did so in just 2 hours and
6 minutes with an average VRAM usage of 4.4 GB, making it both the most
accurate and the most computationally efficient. MobileNetV2, while slightly
less accurate, also demonstrated excellent efficiency, completing training in
under 4 hours with modest memory requirements. From a performance requirements
perspective, these results highlight that modern CNNs are not only more
accurate but also far more resource-efficient. VGG16, despite consuming the
most VRAM (6.5 GB) and requiring over 5 hours of training, delivered poor
results, while AlexNet trained for more than 4 hours yet plateaued at the same
low accuracy. In contrast, EfficientNetV2-B0 and EfficientNet-B0 achieved
state-of-the-art performance in a fraction of the time and memory.


Ultimately, choosing a CNN model from the EfficientNetV2 family
represents the most practical and forward-looking decision for both ripeness
and bruise classification tasks. These models consistently delivered the
highest accuracy across experiments while maintaining shorter training
times and lower memory footprints compared to other architectures.
Their compound scaling strategy allows them to balance depth, width,
and resolution more effectively than earlier CNNs, ensuring strong
generalization without excessive computational cost \cite{tan2019}.
This makes them not only state-of-the-art in predictive performance
but also highly deployable in real-world agricultural settings, where
efficiency, scalability, and reliability are critical. By combining
accuracy, speed, and resource efficiency, the EfficientNetV2 family
provides the best foundation for building robust and sustainable
computer vision systems for fruit quality assessment.

\subsection{Analysis of Table~\ref{tab:more_cnn_training_cpu} and Table~\ref{tab:more_cnn_training_gpu}} 

For ripeness classification, among the EfficientNet V1 models as seen in
Table~\ref{tab:more_cnn_training_cpu} and Table~\ref{tab:more_cnn_training_gpu}
, B0 to B4 exhibited a performance plateau around 89–91\% accuracy.
This can be explained by the compound scaling principle where each
successive variant increases depth, width, and input resolution in tandem
\citep{tan2019}. However, for the benchmark, the input resolution was fixed
at 224×224 for all models. Since B0–B4 are relatively shallow and narrow,
their representational capacity is already well-matched to the available
input information at 224×224. Scaling them further in depth and width
without increasing resolution does not provide additional discriminative
power, leading to plateau in accuracy. Notably, their training accuracies,
ranging from 89.5\% to 92.3\%, closely mirrored their test accuracies,
suggesting that these models were neither severely underfitting nor
overfitting, but rather limited by the resolution bottleneck. In contrast,
B5 and B6 showed measurable improvements (92–93\% accuracy) even under the
224×224 constraint. This is because their increased depth and width allowed
them to extract more abstract and hierarchical features, compensating for
the lack of higher-resolution input. While they were not operating at their
full theoretical potential, which would require larger input sizes like
456×456 or 528×528, their additional capacity still translated into better
generalization for the 3-class ripeness classification task. Essentially,
B5 and B6 reached a sweet spot where the added representational power was
still beneficial, even though the input resolution bottleneck limited
further gains. This is further supported by their training accuracies
(94.1\% for B5 and 96.0\% for B6), which were slightly higher than their
test accuracies, indicating strong learning capacity with only a modest
generalization gap. By contrast, B7 crossed the threshold where additional
scaling became counterproductive. With 18.8 GB of VRAM usage and a 9-hour
training time, its extreme depth and parameter count, combined with the
fixed low-resolution input, led to over-parameterization relative to the
available information, optimization inefficiency, and degraded performance
(87\%). This increase in training time and memory usage is expected, as
higher EfficientNet versions introduce significantly more parameters. For
instance, B6 has over 43 million parameters compared to B5’s 30 million,
resulting in longer forward and backward passes and greater memory
consumption per epoch. If the required memory exceeds available VRAM, the
system resorts to RAM, which has slower access speeds, thereby
significantly increasing training time. On the other hand, EfficientNetV2
models demonstrated superior efficiency and faster convergence. Variants
B0–B3 consistently achieved 91–93\% accuracy, with V2-B3 emerging as the
top performer (precision 0.9258, recall 0.9256, F1-score 0.9253, accuracy
93\%) while maintaining modest VRAM usage (4.5 GB) and a short training
time ($\sim$2 hours). Their training accuracies (92.3–94.0\%) were well
aligned with their test accuracies, confirming that these models
generalized effectively without significant overfitting. In contrast, the
larger variants (V2-S, V2-M, V2-L) all exhibited diminishing returns, as
their increased depth and parameter counts did not translate into higher
accuracy, instead plateauing at 87–89\% while demanding substantially more
computational resources, similar to the case with EfficientNetV1 series.
Their longer training times and higher VRAM usage reflect the same scaling
trade-offs observed in B7, where added complexity does not yield
proportional performance gains under fixed input resolution
\citep{tan2021}. This was also reflected in their training accuracies
(90.0–90.5\%), which showed little advantage over their test results,
reinforcing that additional complexity did not yield meaningful gains. This
performance limitation may also be attributed to the fixed input image size
of 224×224, which constrained the representational capacity of deeper
models , a phenomenon similarly observed with the EfficientNetV1-B7. This
suggests that for a 3-class dataset of approximately 6,000 images,
additional model complexity does not yield proportional performance gains
and may even hinder optimization efficiency. Under these conditions, V2-B3
stands out as the most effective architecture, striking the best balance
between accuracy, efficiency, and training time.assessment.

For bruise classification as seen in
 Table~\ref{tab:more_cnn_training_cpu}, mid-tier EfficientNet V1 models
 (B1–B3) delivered the strongest results, with B3 achieving the highest
 performance (precision = 0.913, recall = 0.913, F1-score = 0.9129,
 accuracy = 91\%). Their training accuracies ($\sim$90–91\%) were closely
 aligned with their test results, indicating that these models generalized
 well without significant overfitting. In contrast, the larger V1 variants
 (B5–B7) required substantially more training time and memory yet plateaued
 at 86–88\% accuracy, reflecting the same diminishing returns noted in
 ripeness classification. This was further supported by their training
 accuracies ($\sim$90–90.5\%), which were only marginally higher than
 their test scores, suggesting that additional depth and parameters did
 not translate into meaningful generalization gains. Among the V2 models,
 V2-B3 stood out with 92\% accuracy and balanced precision/recall (0.919
 each), surpassing the best V1 models while maintaining shorter training
 times and lower memory usage. Meanwhile, the larger V2 variants (S, M, L)
 mirrored the inefficiencies of their V1 counterparts, consuming more
 resources without corresponding accuracy gains. Their training
 accuracies ($\sim$85–86\%) were nearly identical to their test results,
 confirming that these models were underutilizing their added capacity
 under the fixed 224×224 input constraint. Across both families, GPU-based
 training consistently achieved shorter training times than CPU-only runs,
 even though the bruise classification task involved only two classes and
 used the same dataset.

 Overall, EfficientNetV2-B3 emerged as the most practical and effective model
 for both ripeness and bruise classification, combining high accuracy (93\%
 and 92\%, respectively) with modest VRAM requirements and short training
 times ($\sim$2–3 hours). Its balance of performance and efficiency makes it
 particularly well-suited for deployment in real-world agricultural
 applications, where computational resources may be limited but reliable,
 high-accuracy classification is essential. Complementing this, training with
 GPUs proved consistently advantageous across both tasks, as their massively
 parallel architecture is optimized for the matrix multiplications and
 convolution operations central to deep learning. This allowed models to
 converge significantly faster than on CPUs, reducing training times from
 several hours to just a fraction of that. The efficiency gains were
 especially evident in deeper networks, where CPU-only training often became
 impractically slow. Notably, bruise classification, despite involving only
 two classes and the same dataset size, still trained more slowly on CPU
 than ripeness classification did on GPU, underscoring the decisive role of
 hardware acceleration in practical deep learning workflows.

\subsection{Analysis of Confusion Matrix together with Validation Loss and Accuracy}
In this section, the performance of the top three models for both ripeness
and bruise classification is examined in greater detail through their
validation loss and accuracy curves, as well as their corresponding
confusion matrices. These analyses provide deeper insight into how each
model converged during training, the stability of their learning process,
and their ability to generalize beyond the training set. The confusion
matrices, in particular, highlight the distribution of correct and incorrect
predictions across classes, allowing for a clearer understanding of where
misclassifications occur.

\subsubsection{Ripeness Classification}
% effnetb5

To start off, for ripeness classification, The EfficientNet-B5 model achieved strong
overall performance, with a precision of 0.9246, recall of 0.9238, and an
F1-score of 0.924, corresponding to an overall accuracy of 92\%, being the
3rd best model for the task. These values indicate that the model is highly
effective at distinguishing between the three ripeness classes, with balanced
precision and recall suggesting that it does not disproportionately favor one
class over another. Training required approximately 5 hours and 45 minutes,
with an average VRAM usage of 11.6 GB, reflecting the computational demands
of a high-capacity architecture such as EfficientNet-B5.

Based on the confusion matrix in Figure~\ref{fig:effnetb5}, the model classified the
majority of samples correctly across all categories, with particularly strong
results for the green and yellow classes. For instance, 223 out of 239
green samples were correctly identified, with only 16 misclassified as
yellow\_green. Similarly, the yellow class showed minimal confusion, with 117
correct predictions and only 7 misclassified as yellow\_green. The greatest
overlap occurred in the yellow\_green class, where 169 samples were correctly
predicted, but 19 were misclassified as either green or yellow. This pattern
suggests that the transitional nature of the yellow\_green class poses the
greatest challenge, as its visual features overlap with both neighboring
categories. Nonetheless, the relatively low misclassification rates confirm that
the model captures the key discriminative features of each ripeness stage.

The validation loss and accuracy curves in Figure~{fig:effnetb5} further illustrate the
model’s behavior during training. Validation accuracy remained consistently
high, stabilizing above 0.90 across all epochs, which indicates that the model
generalized well to unseen data. In contrast, validation loss exhibited
noticeable fluctuations, with sharp drops and occasional peaks at specific
epochs. This divergence between stable accuracy and variable loss suggests
that while the model consistently predicted the correct class, it sometimes
assigned lower confidence to its predictions. This behavior is common in
multi-class classification tasks where class boundaries are less distinct, as
in the case of the yellow\_green category. Importantly, the absence of a
downward trend in accuracy despite the oscillations in loss indicates that the
model did not suffer from severe overfitting.

From a performance requirements perspective, the EfficientNet-B5 model
demonstrates a favorable balance between accuracy and computational cost.
Achieving over 92\% accuracy with an F1-score of 0.924 while maintaining
an average VRAM usage of 11.6 GB indicates that the model is both reliable
and feasible for deployment on high-end GPUs commonly available in
research and industrial settings. The total training time of 5 hours and 45
minutes is reasonable given the model’s depth and parameter count,
suggesting that retraining or fine-tuning for new datasets is practical within
typical project timelines. Importantly, the stability of validation accuracy
across epochs implies that the model converges efficiently without requiring
excessive epochs, further reducing computational overhead. These results
highlight that EfficientNet-B5 not only meets accuracy benchmarks but also
aligns with resource efficiency considerations, making it a strong candidate
for real-world applications where both predictive performance and hardware
constraints must be balanced.


\begin{figure}[!htbp]
    \centering
    \subbottom[Confusion Matrix]{
        \includegraphics[width=0.45\textwidth]{/results/ripeness/b5/confusion_matrix_ripeness}
        \label{fig:effnetb5_cm}
    }
    \hfill
    \subbottom[Validation and Accuracy per Epoch]{
        \includegraphics[width=0.45\textwidth]{/results/ripeness/b5/val_loss_accuracy_curve_ripeness}
        \label{fig:effnetb5_va}
    }
    \caption{Ripeness Training and Testing of EfficientNet-B5}
    \label{fig:effnetb5}
\end{figure}

The second-best model for ripeness classification is EfficientNet-B6, achieving a
 precision of 0.9339, recall of 0.9328, and an F1-score of 0.9331, corresponding
 to an overall accuracy of 93\%. Like EfficientNet-B5, it demonstrated strong
 and balanced performance across all three ripeness categories, but with
 slightly higher accuracy. Training required approximately 7 hours and 12
 minutes, with an average VRAM usage of 14.5 GB, which is substantially more
 demanding than B5, reflecting the deeper architecture and larger parameter
 count.

The confusion matrix in Figure~\ref{fig:effnetb5} shows that the green class was classified
 with high reliability, with 226 correct predictions and only 13 misclassified as
 yellow\_green. The yellow class also performed well, with 115 correct
 predictions and 9 misclassified as yellow\_green. As with B5, the yellow\_green
 class posed the greatest challenge due to its transitional characteristics, with
 173 correct predictions but 15 misclassified as either green or yellow. This
 reinforces the earlier observation that intermediate ripeness stages are
 inherently more ambiguous, though overall misclassification rates remained
 low.

The validation curves in Figure~\ref{fig:effnetb5} further illustrate the model’s training
 dynamics. Validation loss decreased sharply after the first epoch and
 stabilized between 0.2 and 0.4, while validation accuracy steadily increased,
 reaching approximately 0.97 by the final epoch. This consistent improvement
 indicates effective convergence without signs of severe overfitting. Compared
 to B5, B6 leveraged its higher representational capacity to refine feature
 extraction further, leading to more confident predictions.

From a performance standpoint, EfficientNet-B6 clearly delivers superior
 accuracy compared to B5, but at the cost of significantly higher resource
 consumption. While its 93\% accuracy and F1-score above 0.93 make it highly
 reliable for practical applications, the 14.5 GB VRAM requirement and
 extended training time of over 7 hours highlight the trade-off between
 accuracy gains and efficiency. As with B5, this makes B6 well-suited for
 research and industrial environments with high-end GPUs, but less practical
 for real-time or edge deployment without model compression or optimization.

% effnetb6
\begin{figure}[!htbp]
    \centering
    \subbottom[Confusion Matrix]{
        \includegraphics[width=0.45\textwidth]{/results/ripeness/b6/confusion_matrix_ripeness}
        \label{fig:effnetb6_cm}
    }
    \hfill
    \subbottom[Validation and Accuracy per Epoch]{
        \includegraphics[width=0.45\textwidth]{/results/ripeness/b6/val_loss_accuracy_curve_ripeness}
        \label{fig:effnetb6_va}
    }
    \caption{Ripeness Training and Testing of EfficientNet-B6}
    \label{fig:effnetb6}
\end{figure}

The best-performing model for ripeness classification was EfficientNetV2-B3, achieving
 a precision of 0.9258, recall of 0.9256, F1-score of 0.9253, and an overall
 accuracy of 93\%. These results confirm that the model is highly effective at
 distinguishing between the three ripeness categories, with balanced precision
 and recall indicating consistent performance across classes. Training required
 only 2 hours and 2 minutes with an average VRAM usage of 4.5 GB, making
 it far more efficient than deeper variants such as B5 and B6 while still
 achieving comparable accuracy.

The confusion matrix in Figure~\ref{fig:effnetb6} provides further insight into class-level
 performance. The green class was classified with high reliability, with 231
 correct predictions and only 8 misclassified as yellow\_green. The yellow class
 also performed strongly, with 115 correct predictions and 9 misclassified as
 yellow\_green. As with the other models, the yellow\_green class posed the
 greatest challenge, with 164 correct predictions but 24 misclassified as either
 green or yellow. This reflects the inherent ambiguity of the transitional stage,
 where visual features overlap with both neighboring categories. Despite this,
 overall misclassification rates remained low, confirming that the model
 effectively captured the discriminative features of each ripeness stage.

The validation curves in Figure~\ref{fig:effnetb6} further illustrate the model’s training
 dynamics. Validation accuracy remained consistently high, stabilizing between
 0.85 and 0.92 across epochs, while validation loss fluctuated between 0.2 and
 0.4. The stability of accuracy, despite minor oscillations in loss, suggests that
 the model generalized well to unseen data and avoided severe overfitting.
 The fluctuations in loss likely reflect varying confidence in predictions for the
 ambiguous yellow\_green class, but the consistently high accuracy
 demonstrates that the model still assigned correct labels in most cases.

From a performance standpoint, EfficientNetV2-B3 offers the best balance
 between accuracy and computational efficiency. Achieving 93\% accuracy
 with an F1-score above 0.92 while requiring only a fraction of the training
 time and memory of B5 or B6 highlights its practicality for deployment.
 While B6 achieved slightly higher precision and recall, its steep computational
 demands, over 7 hours of training and 14.5 GB of VRAM, make it less
 suitable for iterative experimentation or resource-constrained environments.
 Similarly, B5 delivered strong accuracy but required nearly 6 hours of training
 and 11.6 GB of VRAM, reflecting a high resource cost for only marginal gains.
 In contrast, V2-B3 enables faster experimentation cycles, more accessible
 deployment, and robust classification of both ripeness extremes and
 transitional classes.

Ultimately, EfficientNetV2-B3 provides the optimal trade-off between high-
 quality classification and manageable computational requirements, making it
 the best candidate for mango ripeness classification.

% effnetv2b3
\begin{figure}[!htbp]
    \centering
    \subbottom[Confusion Matrix]{
        \includegraphics[width=0.45\textwidth]{/results/ripeness/v2b3/confusion_matrix_ripeness}
        \label{fig:effnetv2b3_cm}
    }
    \hfill
    \subbottom[Validation and Accuracy per Epoch]{
        \includegraphics[width=0.45\textwidth]{/results/ripeness/v2b3/val_loss_accuracy_curve_ripeness}
        \label{fig:effnetv2b3_va}
    }
    \caption{Ripeness Training and Testing of EfficientNetV2-B3}
    \label{fig:effnetv2b3_rp}
\end{figure}


\subsubsection{Bruises Classification}
% effnetb2

Moving forward with bruise classification, the EfficientNet-B2 model achieved strong performance overall.
It reached a precision of 0.9012, recall of 0.9008, and F1-score of 0.9009.
The overall accuracy was 90\%, ranking as the third-best model tested.
These results show a well-balanced model with minimal trade-offs in detection.
It effectively identifies both bruised and not-bruised cases with reliable accuracy.
Training lasted approximately 3 hours and 8 minutes under stable GPU performance.
Average VRAM usage was about 6.7 GB during the entire training session.
This computational demand remains manageable for most modern GPU-based research setups.

The confusion matrix in Figure~\ref{fig:effnetb2} reveals the class-level distribution clearly.
The model correctly identified 242 bruised and 203 not-bruised fruit samples.
However, it misclassified 27 bruised items as not bruised, indicating false negatives.
Additionally, 22 not-bruised items were misclassified as bruised, producing false positives.
This pattern suggests a slight tendency to under-detect bruised mango samples.
False negatives are critical in quality control because they allow defects through.
Despite these errors, the model maintains strong reliability in classification results overall.

The validation curves in Figure~\ref{fig:effnetb2} illustrate training stability and convergence well.
Validation loss dropped sharply after the first epoch and continued declining steadily.
Validation accuracy increased quickly, stabilizing around 0.85 after several epochs completed.
These trends indicate efficient learning and absence of severe overfitting during training.
Minor oscillations in loss and accuracy reflect normal exploration of local minima.
Such fluctuations are typical in deep learning models seeking optimal decision boundaries.

From a performance perspective, EfficientNet-B2 satisfies practical requirements for bruise detection systems.
With 90\% accuracy and balanced precision-recall metrics, it ensures consistent defect detection.
The model offers reliability without imposing excessive computational or memory resource demands.
Its three-hour training time supports scalability for mid-range GPU deployment setups.
However, false negatives remain a primary issue affecting industrial screening reliability.
Reducing them may involve threshold adjustments or using cost-sensitive learning approaches.
Ensemble methods could further improve robustness and minimize undetected bruised cases effectively.

\begin{figure}[!htbp]
    \centering
    \subbottom[Confusion Matrix]{
        \includegraphics[width=0.45\textwidth]{/results/bruises/b2/confusion_matrix_bruises}
        \label{fig:effnetb2_cm}
    }
    \hfill
    \subbottom[Validation and Accuracy per Epoch]{
        \includegraphics[width=0.45\textwidth]{/results/bruises/b2/val_loss_accuracy_curve_bruises}
        \label{fig:effnetb2_va}
    }
    \caption{Bruises Training and Testing of EfficientNet-B2}
    \label{fig:effnetb2}
\end{figure}

The EfficientNet-B3 model demonstrated strong classification performance across all evaluation metrics.
It achieved a precision of 0.913, recall of 0.913, and F1-score of 0.9129.
Overall accuracy reached 91\%, ranking as the second-best model for bruise classification.
These values reflect high consistency in identifying both bruised and not-bruised samples.
The trade-offs between false positives and false negatives remained minimal overall.
Training required approximately 3 hours and 27 minutes using stable GPU resources.
Average memory usage was 8 GB, slightly higher than EfficientNet-B2’s requirements.
Despite this, resource demands remained feasible for most modern GPU systems.

The confusion matrix in Figure~\ref{fig:effnetb3} presents the model’s classification outcomes clearly.
The network correctly identified 250 bruised and 201 not-bruised fruit samples.
It misclassified 19 bruised items as not bruised, representing false negatives.
Additionally, 24 not-bruised items were misclassified as bruised, forming false positives.
Compared to EfficientNet-B2, this model reduced false negatives significantly overall.
This reduction decreases the likelihood of defective mangoes passing inspection unnoticed.
Such improvement is crucial in quality control, where undetected bruising is costly.
False alarms are less concerning than missed detections in industrial screening tasks.

The validation curves in Figure~\ref{fig:effnetb3} depict stable training and convergence performance.
Validation loss decreased steadily across epochs, showing consistent learning throughout training.
Validation accuracy stabilized between 0.80 and 0.85 with minimal oscillations.
This parallel pattern of low loss and stable accuracy suggests good generalization.
The relatively flat accuracy curve after early epochs indicates efficient convergence overall.
No signs of instability or severe overfitting were observed during final training.

From a performance perspective, EfficientNet-B3 offers improved reliability over EfficientNet-B2.
It balances classification accuracy and computational efficiency more effectively for bruise detection.
Although training time and memory usage slightly increased, accuracy gains justify the cost.
The reduced false negatives strengthen model dependability for automated quality control.
This characteristic ensures fewer defective fruits are misclassified as acceptable products.
Overall, EfficientNet-B3 represents a dependable and scalable choice for industrial bruise inspection.

% effnetb3
\begin{figure}[!htbp]
    \centering
    \subbottom[Confusion Matrix]{
        \includegraphics[width=0.45\textwidth]{/results/bruises/b3/confusion_matrix_bruises}
        \label{fig:effnetb3_cm}
    }
    \hfill
    \subbottom[Validation and Accuracy per Epoch]{
        \includegraphics[width=0.45\textwidth]{/results/bruises/b3/val_loss_accuracy_curve_bruises}
        \label{fig:effnetb3_va}
    }
    \caption{Bruises Training and Testing of EfficientNet-B3}
    \label{fig:effnetb3}
\end{figure}

The EfficientNetV2-B3 model achieved the best overall performance for bruise classification.
It reached precision, recall, and F1-score values all equal to 0.919.
The overall accuracy was 92\%, demonstrating strong and balanced predictive capability.
These metrics confirm consistent performance across both bruised and not-bruised mango classes.
Neither precision nor recall dominated at the expense of the other.
Training was notably efficient, finishing in just 2 hours and 55 minutes.
Average VRAM usage measured only 6.2 GB throughout the training process.
This requirement was lower than both EfficientNet-B2 and EfficientNet-B3 models.
Despite lower computational demand, the model still achieved superior classification accuracy.
This efficiency-accuracy balance makes V2-B3 practical for constrained computing environments.

The confusion matrix in Figure~\ref{fig:effnetv2b3_rp} illustrates the model’s predictive distribution.
The network correctly classified 245 bruised and 202 not-bruised fruit samples.
It misclassified 24 bruised items as not bruised, forming false negatives.
Meanwhile, 23 not-bruised items were incorrectly labeled as bruised, forming false positives.
Compared to previous models, V2-B3 exhibited a more balanced error profile.
EfficientNet-B2 and B3 showed slightly higher false negatives or false positives respectively.
From a practical perspective, false negatives pose greater risks in production.
Undetected bruised fruit directly threaten overall product quality and customer satisfaction.
Although the number of missed detections was relatively small, optimization remains beneficial.
Techniques such as threshold tuning or cost-sensitive loss functions may further reduce them.

The validation curves in \ref{fig:cnn_bruises_info} show consistent training convergence behavior.
Validation accuracy steadily increased and stabilized close to 0.9 after several epochs.
Validation loss fluctuated slightly but showed a clear downward trend overall.
This parallel pattern of stable accuracy and decreasing loss indicates effective generalization.
Minor oscillations in loss reflect expected variations due to batch differences.
Such fluctuations were also observed in EfficientNet-B2 and EfficientNet-B3 models.
However, V2-B3 maintained consistently higher accuracy across the entire training process.
No severe overfitting or instability was observed during model development or validation.

In summary, EfficientNetV2-B3 outperformed both EfficientNet-B2 and EfficientNet-B3 comprehensively.
It delivered superior predictive accuracy while reducing training time and memory consumption.
The model also demonstrated smoother convergence and improved stability during optimization.
This balance of precision, efficiency, and robustness highlights its deployment suitability.
EfficientNetV2-B3 stands as the most effective network for automated bruise detection.
It provides a scalable, reliable, and resource-efficient solution for industrial quality control.
% effnetv2b3
\begin{figure}[!htbp]
    \centering
    \subbottom[Confusion Matrix]{
        \includegraphics[width=0.45\textwidth]{/results/bruises/v2b3/confusion_matrix_bruises}
        \label{fig:effnetv2b3_cm}
    }
    \hfill
    \subbottom[Validation and Accuracy per Epoch]{
        \includegraphics[width=0.45\textwidth]{/results/bruises/v2b3/val_loss_accuracy_curve_bruises}
        \label{fig:effnetv2b3_va}
    }
    \caption{Bruises Training and Testing of EfficientNetV2-B3}
    \label{fig:effnetv2b3_rp}
\end{figure}


\subsubsection{CNN}

The final CNN model for ripeness and bruise classification utilized EfficientNetV2-B3.
 Collected experimental data confirmed that it achieved the best performance-to-efficiency ratio.
 It consistently outperformed other architectures tested during benchmarking and optimization stages.
 For the final ripeness classification, the complete dataset contained around 14,000 images.
 The model achieved a test accuracy of 98\%, with precision, recall, and F1-score near 0.985.
 This consistency across metrics demonstrates both high accuracy and class-balanced reliability.
 It performed uniformly across all ripeness categories without favoring any particular class.
 Validation accuracy of 98.41\% closely matched the test accuracy, confirming excellent generalization.
 The slightly higher training accuracy of 99.37\% indicated minimal overfitting occurrence.
 The narrow gap between training, validation, and test results reflected stable learning.
 These findings confirm that dataset refinement and optimization prevented memorization effectively.
 They also promoted genuine feature learning across ripeness categories and lighting variations.

The confusion matrix in Figure~\ref{fig:ripeness_confusion_matrix_fig} further supports these conclusions clearly.
 Misclassifications were minimal and occurred mostly between adjacent ripeness categories.
 Errors were concentrated between transitional stages such as yellow-green and yellow mangoes.
 This pattern matches the biological ambiguity seen during mango ripening transitions.
 Even human evaluators sometimes disagree on borderline ripeness due to visual overlap.
 The model’s strong accuracy in these ambiguous cases reflects superior discriminative ability.
 It demonstrates practical reliability for deployment in real-world mango grading systems.

Several major modifications to the training pipeline improved overall model effectiveness significantly.
 Mixed-precision training using GradScaler and autocast reduced GPU memory consumption substantially.
 This optimization increased batch size from 32 to 56, enhancing training stability.
 Larger batch sizes improved gradient estimation and smoothed convergence across training epochs.
 Input resolution was corrected to 300×300, matching EfficientNetV2-B3’s native architecture.
 This adjustment improved feature extraction and ensured compatibility with pretrained weights.
 The optimizer was changed from Adam to \acr{AdamW} for stronger regularization.
 Learning rate was set to 3e-4, and weight decay to 1e-4.
 These parameters decoupled regularization from gradient updates, ensuring stable convergence behavior.
 A cosine annealing warm-restart scheduler with T0 = 5 and Tmult = 2 was applied.
 It included three warm-up epochs to escape sharp minima effectively during training.

Additional refinements further improved training robustness and model generalization performance.
 CrossEntropy loss with label smoothing of 0.05 reduced overconfident predictions.
 This adjustment improved resilience to ambiguous ripeness categories and noisy image labels.
 Early stopping with a patience of five epochs prevented redundant computation cycles.
 Checkpointing saved the best weights once performance improvements plateaued consistently.
 Data loading was optimized with workers set to half of available CPU cores.
 Pin\_memory and non\_blocking transfers accelerated CPU-to-GPU data streaming throughput.
 These optimizations minimized data bottlenecks and reduced idle GPU computation time.
 Regularization through dropout = 0.25 and drop-path = 0.15 improved network robustness.
 These techniques prevented neuron co-adaptation and encouraged diverse feature representations.

The validation curves in Figure~\ref{fig:ripeness-graph} confirm stable convergence throughout training.
 Validation accuracy increased steadily before plateauing at a consistently high level.
 Validation loss showed minor oscillations but followed an overall downward trajectory.
 This inverse relationship between loss and accuracy indicates strong discriminative learning ability.
 Accuracy stability despite small loss fluctuations shows resistance to overfitting.
 These patterns confirm that optimizations such as label smoothing and annealing worked effectively.
 The model maintained robustness and generalization even in complex visual conditions.
 Its smooth convergence underscores training stability and computational efficiency across all epochs.

Lastly, dataset enhancements contributed substantially to achieving these superior results overall.
 The dataset expanded from approximately 6,000 to 14,000 well-curated mango images.
 New Carabao mango samples were added, improving variety and biological representativeness.
 Ambiguous or noisy samples were removed to reduce label uncertainty significantly.
 Augmentation strategies were refined to introduce meaningful color, rotation, and lighting diversity.
 These augmentations enhanced robustness by exposing the network to realistic visual variations.
 As a result, the final model generalized strongly and maintained stable performance.
 Across all dataset splits, it demonstrated consistent accuracy and balanced classification reliability.

\begin{table}[htbp]
	\centering
	\begin{tabular}{l|c|c|c|c}
	  \hline \hline
	  \textbf{ } & \textbf{Precision} & \textbf{Recall} & \textbf{F1} & \textbf{Support} \\
    \hline
	  Green & 0.98 & 0.99 & 0.99 & 210 \\
    \hline
	  Yellow & 0.99 & 0.99 & 0.99 & 161 \\
	  \hline
	  Yellow\_Green & 0.98 & 0.98 & 0.98 & 219 \\
	  \hline
	  Accuracy &  \multicolumn{2}{c|}{ }  & 0.98 & 590 \\
	  \hline
	  Macro Avg & 0.99 & 0.99 & 0.99 & 590 \\
	  \hline
    Weighted Avg & 0.98 & 0.98 & 0.98 & 590 \\
	  \hline
	\end{tabular}
	\caption{EfficientNetV2-B3 Ripeness Classification Report with Precision: 0.9848, Recall: 0.9847, F1 Score: 0.9847}
	\label{tab:ripeness_classification_report}
\end{table}

\begin{figure}[!htbp]
	\centering
	\includegraphics[width=0.5\textwidth]{/effnetv2-m-ripeness/confusion_matrix_ripeness}
	\caption{EfficientNetV2-B3 Ripeness Confusion Matrix}
	\label{fig:ripeness_confusion_matrix_fig}
\end{figure}


\begin{figure}[!htbp]
	\centering
	\includegraphics[width=0.9\textwidth]{/graphs/ripeness_plot}
	\caption{EfficientNetV2-B3 Ripeness Accuracy and Loss Graph}
	\label{fig:ripeness-graph}
\end{figure}

\subsection{Bruises Classification Results} \label{sec:bruisesClassificationResults}

\subsubsection{CNN} 
For bruise classification, the final EfficientNetV2-B3 model also performed excellently.
It achieved a test accuracy of 99\%, with precision, recall, and F1-score near 0.989.
The validation accuracy of 99.31\% and training accuracy of 99.86\% confirmed stability.
These results demonstrate exceptional reliability and consistent performance across all dataset splits.
The training configuration was refined to improve both computational efficiency and robustness.
Batch size was increased to 60, fully utilizing available GPU memory capacity.
This adjustment enhanced gradient stability and accelerated convergence across training epochs effectively.
Regularization parameters were tuned with a dropout rate of 0.2 overall.
A drop-path rate of 0.1 was also applied to further control overfitting.
Together, these settings balanced high predictive accuracy with improved model generalization capability.
Early stopping with a patience of 10 epochs was employed during training.
This ensured meaningful improvement capture while avoiding unnecessary computation after convergence detection.

The confusion matrix in Figure~\ref{fig:bruises_confusion_matrix_fig} reinforces these excellent quantitative results clearly.
The model correctly identified nearly all samples across both bruise categories tested.
Only four false negatives and one false positive occurred in total predictions.
This minimal error distribution illustrates a well-balanced and highly reliable classification profile.
The model demonstrated strong sensitivity to bruised fruit and high specificity otherwise.
Low false negatives are particularly important in postharvest quality control applications.
Undetected bruises pose a major risk to maintaining consistent product quality standards.
The low occurrence of such cases underscores the model’s robustness and precision.
These characteristics make EfficientNetV2-B3 ideal for deployment in real-time inspection systems.

The validation curves in Figure~\ref{fig:bruises-graph} further illustrate stable training convergence behavior.
Validation accuracy rose rapidly during initial epochs and stabilized near 0.99 overall.
Meanwhile, validation loss decreased sharply early on and then gradually leveled off.
Minor fluctuations in loss reflect typical batch-level variations during optimization cycles.
Despite these oscillations, accuracy remained consistently high and stable throughout training.
This indicates that the network maintained strong confidence in its classification predictions.
The inverse correlation between loss and accuracy confirms effective learning of features.
These patterns demonstrate robust generalization and the absence of significant overfitting problems.
Together, the curves validate that all applied optimizations improved convergence stability efficiently.
EfficientNetV2-B3 thus combines exceptional accuracy, reliability, and computational efficiency effectively.
This performance level establishes it as the optimal model for bruise classification.
Its predictive precision makes it suitable for industrial-grade automated quality control systems.

\begin{table}[htbp]
	\centering
	\begin{tabular}{l|c|c|c|c}
	  \hline \hline
	  \textbf{ } & \textbf{Precision} & \textbf{Recall} & \textbf{F1} & \textbf{Support} \\
	  \hline
	  Bruised & 1.00 & 0.98 & 0.99 & 206 \\
	  \hline
	  Not Bruised & 0.98 & 1.00 & 0.99 & 234 \\
	  \hline
	  Accuracy &  \multicolumn{2}{c|}{ }  & 0.99 & 440 \\
	  \hline
	  Macro Avg & 0.99 & 0.99 & 0.99 & 440 \\
	  \hline
    Weighted Avg & 0.99 & 0.99 & 0.99 & 440 \\
	  \hline
	\end{tabular}
	\caption{EfficientNetV2-B3 Bruises Classification Report with Precision: 0.9887, Recall: 0.9886, F1 Score: 0.9886}
	\label{tab:bruises_classification_report}
\end{table}

\begin{figure}[!htbp]
	\centering
	\includegraphics[width=0.5\textwidth]{/effnetv2-m-bruises/confusion_matrix_bruises}
	\caption{EfficientNetV2-B3 Bruises Confusion Matrix}
	\label{fig:bruises_confusion_matrix_fig}
\end{figure}


\begin{figure}[!htbp]
	\centering
	\includegraphics[width=0.9\textwidth]{/graphs/bruises_plot}
	\caption{EfficientNetV2-B3 Bruises Accuracy and Loss Graph}
	\label{fig:bruises-graph}
\end{figure}

% [x] todo: revisit once actual interview and testing with expert is done
\section{Comparative Analysis: Model Performance vs. Expert Benchmark}
% placholder till ground truth
To establish a robust benchmark for model performance, a comparative analysis
was conducted against the expert assessment of a qualified horticulturist. This
section outlines the methodology for the expert evaluation and presents a
comparative summary of the results.

\subsection{Comparative Results}
The expert's classifications for the 26 images randomly sampled from the
dataset are presented in
Table~\ref{tab:expert_results}.
These results serve as the validated ground truth against which the predictive
accuracy of the computational models was measured. Note that terms \gls{not:g},
\gls{not:yg}, and \gls{not:y} refer to the mango color categories:
green, yellow-green, and yellow, respectively. Likewise, \gls{not:b} and \gls{not:nb} indicate
bruised and non-bruised mango surfaces.
% [x] todo: revisit once actual interview and testing with expert is done

 \begin{longtable}{cccccc}
     \label{tab:expert_results} \\
     \caption{Expert Classification Results for Mango Phenotypic Traits from Professional 1} \\
     \toprule
     {\textbf{Mango ID}} & \multicolumn{2}{c}{\textbf{Color Category}} & \multicolumn{2}{c}{\textbf{Bruising Status}} & {\textbf{Result}}\\
      & \textbf{Expert} & \textbf{Model} & \textbf{Expert} & \textbf{Model} \\
     \midrule
     \endfirsthead
     
     \multicolumn{6}{c}{{\tablename\ \thetable{} -- continued from previous page}} \\
     \toprule
     {\textbf{Mango ID}} & \multicolumn{2}{c}{\textbf{Color Category}} & \multicolumn{2}{c}{\textbf{Bruising Status}} & {\textbf{Result}}\\
      & \textbf{Expert} & \textbf{Model} & \textbf{Expert} & \textbf{Model} \\
     \midrule
     \endhead
     
     \midrule
     \multicolumn{6}{c}{{Continued on next page}} \\
     \endfoot
     
     \bottomrule
     \endlastfoot
     % -- pg marker 2
         001 & y & y  & b & b & 1\\ % -- 01yg.jpg
         002 & yg & yg  & b & b  & 1\\ % -- 08_g.jpg
         003 & g & g  & nb & nb &  1\\ % -- 07b.jpg
         004 & y & y  & b & b  & 1\\ % -- 10nb.jpg
         005 & g & g  & nb & nb  & 1\\ % -- 05yg.jpg
         006 & yg & yg  & b & b  & 1\\ % --01nb.jpg
     % -- pg marker 3
         007 & yg & y  & nb & nb  & 0.5\\ % -- 06nb.jpg
         008 & g & g  & nb & nb  & 1\\ % -- 07y.jpg
         009 & y & y  & b & b  & 1\\ % -- 08yg.jpg
         010 & yg & yg  & nb & nb  & 1\\ % -- 04g.jpg
         011 & yg & yg  & b & b  & 1\\ % -- 09nb.jpg 
         012 & yg & yg  & b & b  & 1\\ % -- 09y.jpg
     % -- pg marker 4
         013 & g & g  & nb & nb  & 1\\ % -- 07yg.jpg
         014 & yg & yg  & b & b  & 1\\ % -- 03y.jpg
         015 & g & g & b & b  & 1\\ % -- 09b.jpg
         016 & yg & yg  & nb & nb  & 1\\ % -- 08b.jpg
         017 & yg & yg  & nb & nb  & 1\\ % -- 10b.jpg
         018 & yg & yg  & b & nb  & 0.5\\ % -- 02nb.jpg
     % -- pg marker 5
         019 & yg & yg  & nb & nb  & 1\\ % -- 07yg.jpg
         020 & g & g  & nb & nb  & 1\\ % -- 03y.jpg
         021 & yg & yg  & b & b  & 1\\ % -- 09b.jpg
         022 & g & g  & nb & nb  & 1\\ % -- 08b.jpg
         023 & y & y  & b & b  & 1\\ % -- 10b.jpg
         024 & yg & yg  & b & b  & 1\\ % -- 02nb.jpg
     % -- pg marker 5
         025 & yg & yg  & b & b  & 1\\ % -- 03b.jpg
         026 & g & g  & nb & nb  & 1\\ % -- 02g.jpg
 \end{longtable}

 \begin{longtable}{cccccc}
     \label{tab:expert_results} \\
     \caption{Expert Classification Results for Mango Phenotypic Traits
     from Professional 2} \\
     \toprule
     {\textbf{Mango ID}} & \multicolumn{2}{c}{\textbf{Color Category}} & \multicolumn{2}{c}{\textbf{Bruising Status}} & {\textbf{Result}}\\
      & \textbf{Expert} & \textbf{Model} & \textbf{Expert} & \textbf{Model} \\
     \midrule
     \endfirsthead
     
     \multicolumn{6}{c}{{\tablename\ \thetable{} -- continued from previous page}} \\
     \toprule
     {\textbf{Mango ID}} & \multicolumn{2}{c}{\textbf{Color Category}} & \multicolumn{2}{c}{\textbf{Bruising Status}} & {\textbf{Result}}\\
      & \textbf{Expert} & \textbf{Model} & \textbf{Expert} & \textbf{Model} \\
     \midrule
     \endhead
     
     \midrule
     \multicolumn{6}{c}{{Continued on next page}} \\
     \endfoot
     
     \bottomrule
     \endlastfoot
     % -- pg marker 2
         001 & y & y  & b & b & 1\\ % -- 01yg.jpg
         002 & yg & yg  & b & b  & 1\\ % -- 08_g.jpg
         003 & g & g  & nb & nb &  1\\ % -- 07b.jpg
         004 & y & y  & b & b  & 1\\ % -- 10nb.jpg
         005 & g & g  & nb & nb  & 1\\ % -- 05yg.jpg
         006 & y & yg  & b & b  & 0.5\\ % --01nb.jpg
     % -- pg marker 3
         007 & yg & y  & nb & nb  & 0.5\\ % -- 06nb.jpg
         008 & g & g  & nb & nb  & 1\\ % -- 07y.jpg
         009 & y & y  & b & b  & 1\\ % -- 08yg.jpg
         010 & yg & yg  & nb & nb  & 1\\ % -- 04g.jpg
         011 & yg & yg  & b & b  & 1\\ % -- 09nb.jpg 
         012 & yg & yg  & nb & b  & 0.5\\ % -- 09y.jpg
     % -- pg marker 4
         013 & g & g  & nb & nb  & 1\\ % -- 07yg.jpg
         014 & yg & yg  & b & b  & 1\\ % -- 03y.jpg
         015 & g & g & b & b  & 1\\ % -- 09b.jpg
         016 & yg & yg  & nb & nb  & 1\\ % -- 08b.jpg
         017 & yg & yg  & nb & nb  & 1\\ % -- 10b.jpg
         018 & yg & yg  & b & nb  & 0.5\\ % -- 02nb.jpg
     % -- pg marker 5
         019 & yg & yg  & nb & nb  & 1\\ % -- 07yg.jpg
         020 & g & g  & nb & nb  & 1\\ % -- 03y.jpg
         021 & yg & yg  & b & b  & 1\\ % -- 09b.jpg
         022 & g & g  & nb & nb  & 1\\ % -- 08b.jpg
         023 & y & y  & b & b  & 1\\ % -- 10b.jpg
         024 & y & yg  & b & b  & 0.5\\ % -- 02nb.jpg
     % -- pg marker 5
         025 & yg & yg  & b & b  & 1\\ % -- 03b.jpg
         026 & yg & g  & nb & nb  & 0.5\\ % -- 02g.jpg
 \end{longtable}

 \begin{longtable}{cccccc}
     \label{tab:expert_results} \\
     \caption{Expert Classification Results for Mango Phenotypic Traits
     from Professional 3} \\
     \toprule
     {\textbf{Mango ID}} & \multicolumn{2}{c}{\textbf{Color Category}} & \multicolumn{2}{c}{\textbf{Bruising Status}} & {\textbf{Result}}\\
      & \textbf{Expert} & \textbf{Model} & \textbf{Expert} & \textbf{Model} \\
     \midrule
     \endfirsthead
     
     \multicolumn{6}{c}{{\tablename\ \thetable{} -- continued from previous page}} \\
     \toprule
     {\textbf{Mango ID}} & \multicolumn{2}{c}{\textbf{Color Category}} & \multicolumn{2}{c}{\textbf{Bruising Status}} & {\textbf{Result}}\\
      & \textbf{Expert} & \textbf{Model} & \textbf{Expert} & \textbf{Model} \\
     \midrule
     \endhead
     
     \midrule
     \multicolumn{6}{c}{{Continued on next page}} \\
     \endfoot
     
     \bottomrule
     \endlastfoot
     % -- pg marker 2
         001 & y & y  & b & b & 1\\ % -- 01yg.jpg
         002 & y & yg  & b & b  & 0.5\\ % -- 08_g.jpg
         003 & yg & g  & nb & nb &  0.5\\ % -- 07b.jpg
         004 & y & y  & b & b  & 1\\ % -- 10nb.jpg
         005 & g & g  & nb & nb  & 1\\ % -- 05yg.jpg
         006 & y & yg  & b & b  & 0.5\\ % --01nb.jpg
     % -- pg marker 3
         007 & y & y  & nb & nb  & 1\\ % -- 06nb.jpg
         008 & g & g  & nb & nb  & 1\\ % -- 07y.jpg
         009 & y & y  & b & b  & 1\\ % -- 08yg.jpg
         010 & yg & yg  & nb & nb  & 1\\ % -- 04g.jpg
         011 & yg & yg  & b & b  & 1\\ % -- 09nb.jpg 
         012 & yg & yg  & b & b  & 1\\ % -- 09y.jpg
     % -- pg marker 4
         013 & g & g  & nb & nb  & 1\\ % -- 07yg.jpg
         014 & yg & yg  & b & b  & 1\\ % -- 03y.jpg
         015 & g & g & b & b  & 1\\ % -- 09b.jpg
         016 & yg & yg  & nb & nb  & 1\\ % -- 08b.jpg
         017 & yg & yg  & nb & nb  & 1\\ % -- 10b.jpg
         018 & yg & yg  & b & nb  & 0.5\\ % -- 02nb.jpg
     % -- pg marker 5
         019 & yg & yg  & nb & nb  & 1\\ % -- 07yg.jpg
         020 & g & g  & nb & nb  & 1\\ % -- 03y.jpg
         021 & yg & yg  & b & b  & 1\\ % -- 09b.jpg
         022 & g & g  & nb & nb  & 1\\ % -- 08b.jpg
         023 & y & y  & b & b  & 1\\ % -- 10b.jpg
         024 & y & yg  & b & b  & 0.5\\ % -- 02nb.jpg
     % -- pg marker 5
         025 & yg & yg  & b & b  & 1\\ % -- 03b.jpg
         026 & yg & g  & nb & nb  & 0.5\\ % -- 02g.jpg
 \end{longtable}


After compiling the scores, the model achieved an overall score of 71 out of
78. This translates to a 91.02\% accuracy rate, meaning the model's answers were
correct 91.02\% of the time when compared to the mango expert's benchmark.

It is important to note that the expert's grading was conducted independently
and consecutively, without external guidance or tools to aid their judgment.
This purely human evaluation, while authoritative, inevitably introduces a
degree of inherent human error.

\section{Size Determination Results} \label{sec:sizeDeterminationResults}

\subsection{Actual and Estimated Length}
\begin{figure}[!htbp]
	\centering
	\includegraphics[width=0.8\textwidth]{/size_rev/1}
	\caption{Bar Graph of Actual vs Estimated Length}
	\label{fig:size_rev1}
\end{figure}
Starting off for size determination, the method for measuring length achieved an average error of 3.41\% 
with a median of 3.15\% 
and a standard deviation of 0.02, showing that length estimation was highly consistent and tightly clustered around the mean. Most mangoes exhibited differences below 5\%,
with only a few samples such as Mango 3 and Mango 4 exceeding this threshold 
as seen on Figure~\ref{fig:size_rev1}. These deviations were primarily due to bounding box approximation, where slight misalignment of 
contours led to overestimation. The low variability demonstrates that the code reliably captures mango length, and the small errors are unlikely to affect classification outcomes


\subsection{Actual and Estimated Width}
\begin{figure}[!htbp]
	\centering
	\includegraphics[width=0.8\textwidth]{/size_rev/2}
	\caption{Bar Graph of Actual vs Estimated Width}
	\label{fig:size_rev2}
\end{figure}
For width, the average error was 3.81\%, the median was 3.92\%, and the standard deviation was 0.03, reflecting slightly higher variability 
compared to length but still within a stable range
as seen on Figure~\ref{fig:size_rev2}. Most mangoes showed differences between 2–6\%, though Mango 6 was a clear outlier with a width error of 9.67\%, 
which inflated the overall variability. This error was likely caused by segmentation inconsistencies at the fruit edges, where the HSV mask occasionally included background pixels or missed portions of the mango contour. Despite this, the majority of samples demonstrated stable width estimation, confirming that the method is effective but sensitive to segmentation accuracy.

\subsection{Calculated Area and Estimated Area}
\begin{figure}[!htbp]
	\centering
	\includegraphics[width=0.8\textwidth]{/size_rev/3}
	\caption{Bar Graph of Actual vs Estimated Area}
	\label{fig:size_rev3}
\end{figure}
For area, which is the most critical parameter for size classification, the code produced an average error of 4.51\%, a median of 4.83\%, 
and a standard deviation of 0.03, indicating consistent performance across the dataset. Most mangoes were measured within a 2–6\% difference, with Mango 15 
showing nearly perfect agreement at 0.14\% error
as seen on Figure~\ref{fig:size_rev3}. Larger deviations were observed in Mango 3 and Mango 6, where area errors reached 8–9\%, primarily due to 
compounding effects of length and width misestimation. These results highlight that while area estimation is generally reliable, boundary cases near classification 
thresholds may be more prone to misclassification. Nonetheless, the overall accuracy demonstrates that the code is effective for non-destructive mango grading, with 
error margins well within acceptable tolerance.

\subsection{Summarized Size Results}
\begin{figure}[!htbp]
	\centering
	\includegraphics[width=0.8\textwidth]{/size_rev/4}
	\caption{List of Size Results}
	\label{fig:size_rev4}
\end{figure}
Overall, based on Figure~\ref{fig:size_rev4}, the data shows that the mango size determination code produced results that were consistently close to manual caliper measurements 
across the 15-sample dataset. The average error margins of 3.41 \% for length, 3.81\% for width, and 4.51\% for area, combined with very low standard deviations 
of 0.02, 0.03, and 0.03 respectively, indicate that the system maintained stable accuracy with minimal variability. Most mangoes fell within a 2–6\% difference, 
which is acceptable for practical grading, while only a few outliers exceeded 8–9\% error. Likewise,
the small, medium, and large classification with a 3$\text{cm}^2$ is shown in Figure~\ref{fig:size_rev5} and the more than 
40 mangoes can be found on Figure~\ref{fig:size_rev6}. These findings confirm that the methodology is effective for non-destructive 
mango sizing and classification, with errors generally small and consistent across samples.
\begin{figure}[!htbp]
	\centering
	\includegraphics[width=0.5\textwidth]{/size_rev/5}
	\caption{Size Area Classification with $\text{cm}^2$ Gap}
	\label{fig:size_rev5}
\end{figure}

\begin{figure}[!htbp]
	\centering
	\includegraphics[width=0.2\textwidth]{/size_rev/ks}
	\caption{Tested 42 Mangoes}
	\label{fig:size_rev6}
\end{figure}


\section{Formula with User Priority } \label{sec:userPriorityFormula}

The Figures \ref{fig:bruises-only-input}, \ref{fig:ripeness-bruises-only-input} and 
\ref{fig:ripeness-only-input} are explained in this section where the inputted weight
values are all real number since negative and imaginary number are not allowed. The
purpose of this section is to demonstrate the different possible cases of using 
the zero value in the user priority.


\begin{figure}[!htbp]
	\centering
	\includegraphics[width=0.8\textwidth]{/priority/bruises-only}
	\caption{Only Bruises as a None Zero Value}
	\label{fig:bruises-only-input}
\end{figure}

An example of where the user only prioritizes bruises is shown on Figure~\ref{fig:bruises-only-input}.
This implies that the user disregards the ripeness and the size of the Carabao mangoes 
by setting the input priority value to zero.

\begin{figure}[!htbp]
	\centering
	\includegraphics[width=0.8\textwidth]{/priority/ripeness-and-bruises-only}
	\caption{Only Ripeness and Bruises as a None Zero Value}
	\label{fig:ripeness-bruises-only-input}
\end{figure}

Another example shown on Figure~\ref{fig:ripeness-bruises-only-input} shows where the user only
prioritized two mango characteristics which are the bruises and the ripeness. This is because 
the user set the size to zero. As such when grading the mangoes, it would still show the prediction
of the size however when grading the Carabao mango it would disregard the size in its calculation.

\begin{figure}[!htbp]
	\centering
	\includegraphics[width=0.8\textwidth]{/priority/ripeness-only}
	\caption{Only Ripeness as a None Zero Value}
	\label{fig:ripeness-only-input}
\end{figure}

Another similar user priority input to Figure~\ref{fig:bruises-only-input} is 
Figure~\ref{fig:ripeness-only-input} where it only prioritizes one parameter which is the ripeness.
Furthermore, notice the range of values for each grade has a maximum of 3.00 and a minimum of 1.00.
This is because the input weight of the ripeness is 1.0 meaning that the possible values are 1.00, 2.00, and 3.00.

\section{Physical Prototype} \label{sec:physicalPrototype}

\subsection{Version 1: Barebone with Black Conveyor Sheets}

For the physical prototype, there are two main parts which are the image acquisition system and
the conveyor belt. Both of these parts are being controlled by a \acr{RPi} through a python script.
Note that the DC motors, 4 channel relay, and camera can be seen on Figure~\ref{fig:hardware_view}.
For the first version of the prototype, Figure~\ref{fig:v1_prototype} shows three images which are 
the top view, entrance view of the Carabao mangoes and the side view of the prototype.
Notice that it is a barebone prototype made out of plywood with four rollers and black matte sheets
for moving the Carabao mangoes. There are two DC motors controlling each conveyor belt.
As seen on the side of the prototype on Figure~\ref{fig:v1_prototype}, the black sheet is not 
flexible and too stiff to be able to move it with the mangoes. This means that the conveyor 
belt would not be able to rotate and move the Carabao mangoes consistently.

\begin{figure}[!htbp]
    \centering
    \subbottom[Prototype Top View]{
        \includegraphics[width=0.45\textwidth]{top_view}
        \label{fig:top_view}
    }
    \hfill
    \subbottom[Entrance Conveyor Belt View]{
        \includegraphics[width=0.45\textwidth]{side1}
        \label{fig:side1}
    }
    \vfill
    \subbottom[Side Conveyor Belt View]{
        \includegraphics[width=0.45\textwidth]{side2}
        \label{fig:side2}
    }
    \caption{Version 1 of the Prototype}
    \label{fig:v1_prototype}
\end{figure}

\begin{figure}[!htbp]
    \centering
    \subbottom[Prototype Main Hardware]{
        \includegraphics[width=0.45\textwidth]{schematic1}
        \label{fig:schematic1}
    }
    \hfill
    \subbottom[DC Motor and Pulley]{
        \includegraphics[width=0.45\textwidth]{DC_Motor_Pulley}
        \label{fig:dc_motor_pulley}
    }
    \vfill
    \subbottom[LED Lights and Camera Module]{
        \includegraphics[width=0.45\textwidth]{camera}
        \label{fig:camera}
    }
    \caption{Hardware View}
    \label{fig:hardware_view}
\end{figure}

\subsection{Version 2: Enclosed with White Conveyor Sheets and Physical Sorter}
% todo explain the improved prototype version 2 which are the conveyor belt and enclosing the 
% prototype on the same black sheet
For the second version of the prototype as seen on
Figure~\ref{fig:v2_prototype}, improvements such as 
replacing the black sheet to a white sheet which improved the
efficiency and reduced the frequency of requiring maintenance.
Another improvement for this version is enclosing the 
electronic devices in a container. This helps protect it from
unwanted liquid spills. For the sorting of mangoes,
the conveyors would sort it into three grades
which are Grade A, B, and C.
It would first go through the longest conveyor and the 
shorter conveyor depending on the grade.
This is because if the Grade is A (which is the highest),
then it would exit to the east of the prototype
and not go through the shorter conveyor belt.
For Grade B, it would go through the west side and then north of 
the prototype.
Finally for grade C, it would go through west side and then south
of the prototype.
The code for this can be seen on Listing~\ref{lst:mango_sorter}.

\begin{lstlisting}[
float=h,
caption=Sorting the Mangoes, 
label=lst:mango_sorter,
language=TeX,
frame=single]
if ave_letter.upper()  == 'A':
    button_state_array = [0, 1, 0, 0]
    print(button_state_array)
    self.sort.set_motors(button_state_array)
elif ave_letter.upper() == 'B':
    button_state_array = [1, 0, 1, 0]
    print(button_state_array)
    self.sort.set_motors(button_state_array)
elif ave_letter.upper() == 'C':
    button_state_array = [1, 0, 0, 1]
    print(button_state_array)
    self.sort.set_motors(button_state_array)
\end{lstlisting}

\begin{figure}[!htbp]
    \centering
    \subbottom[Side View of Improved Prototype]{
        \includegraphics[width=0.45\textwidth]{new-side-view}
        \label{fig:new-side-view}
    }
    \hfill
    \subbottom[Top View of Improved Prototype]{
        \includegraphics[width=0.45\textwidth]{new-top-view}
        \label{fig:new-top-view}
    }
    \vfill
    \subbottom[Inside Hardware View]{
        \includegraphics[width=0.45\textwidth]{/sorting/hardware}
        \label{fig:T_sort}
    }
    \hfill
    \subbottom[Sorting Mangoes Using Two Conveyor Belts]{
        \includegraphics[width=0.45\textwidth]{/sorting/T}
        \label{fig:T_sort}
    }
    \caption{Version 2: Improved Prototype}
    \label{fig:v2_prototype}
\end{figure}

\section{Software Application}

\subsection{Version 1: Progress Bar with Black Conveyor Sheets}
For the software application inside the \acr{RPi}, CustomTkinter is used as the main 
GUI for the python application. For the versions, there are two main versions. 
The first version which involves a fully automated capturing of both sides of the Carabao mango
and the second version which uses a part by part picturing and moving of mangoes. 

For this version, some of the initial UI design are shown on 
Figure~\ref{fig:user_interface_v1}. There are two three main columns which are 
the live video feed with a progress bar,
two sides of the mango cheek, and the control panel
with the different buttons such as the user priority, and reset, stop, export,
and help. The approach to this one involves fully automatically 
moving and grading the mango which caused the grading to be inconsistent 
because it was not able to fully rotate the mango at most cases.

\begin{figure}[!htbp]
    \centering
    \subbottom[Version 1.1]{
        \includegraphics[width=0.45\textwidth]{UI_v1}
        \label{fig:ui_v1}
    }
    \hfill
    \subbottom[Version 1.2]{
        \includegraphics[width=0.45\textwidth]{UI_v2}
        \label{fig:ui_v2}
    }
    \vfill
    \subbottom[Version 1.3]{
        \includegraphics[width=0.45\textwidth]{UI_v3}
        \label{fig:ui_v3}
    }

    \caption{Version 1 of the RPi's User Interface}
    \label{fig:user_interface_v1}
\end{figure}

\subsection{Version 2: Improved UI without Progress Bar}
% todo explain the second version and its upgrades to the first version
For the second version of the software as seen on Figure~\ref{fig:ui_main_v2}, 
an overhaul of the 
UI design was done with the hopes that it would be cleaner
and intuitive. Some features such as the progress bar was removed because 
this method uses a step by step approach for rotating the mango
where the user would rotate it using the buttons and 
how long they want to move the conveyors. Likewise, the stop buttons
for all the conveyors are added.

\begin{figure}[!htbp]
    \centering
    \subbottom[Version 2.1 with Background Image]{
        \includegraphics[width=0.45\textwidth]{/ui/ui-version1}
        \label{fig:app_v4}
    }
    \hfill
    \subbottom[Version 2.2 without Background Image]{
        \includegraphics[width=0.45\textwidth]{/ui/ui-version2}
        \label{fig:app_v5}
    }
    \vfill
    \subbottom[Version 2.3 with Stop Sorting Button]{
      \includegraphics[width=0.45\textwidth]{/sorting/stop_sort}
        \label{fig:stop_sort_button}
    }

    \caption{Version 2 of the RPi's User Interface}
    \label{fig:ui_main_v2}
\end{figure}

% todo explain how it sorts the Carabao mango images
\subsection{Mango Image Sorting}
Figure~\ref{fig:img_sorted} shows the 
method sorting the mango images through
a directory containing the year, date, and time.
Likewise, inside that directory, is the three 
possible grades from A to C and the input priorities
of the user.

\begin{figure}[!htbp]
    \centering
    \subbottom[Folder Tree Directory of Each Grade]{
        \includegraphics[width=0.45\textwidth]{sort_tree_1}
        \label{fig:sort_tree}
    }
    \hfill
    \subbottom[Directory with Year, Date, and Time]{
        \includegraphics[width=0.45\textwidth]{sort_folder}
        \label{fig:sort_folder}
    }
    \vfill
    \subbottom[Saved Input Priority/Weights]{
        \includegraphics[width=0.45\textwidth]{sort_input_prior}
        \label{fig:input_folder_prior}
    }

    \caption{Mango Image Data Sorting}
    \label{fig:img_sorted}
\end{figure}

\subsection{Error Handling}
Figure~\ref{fig:error-handel-1} shows the three possible 
error messages when the user inputs all zero in the user 
priority, presses all and none of the buttons when moving
the conveyor. In the case the user inputs a a letter or negative value,
then the not number error message would pop up as shown in
Figure~\ref{fig:error-handel-2}.

\begin{figure}[!htbp]
    \centering
    \subbottom[All Zero Error]{
        \includegraphics[width=0.45\textwidth]{/errors/all-zero}
        \label{fig:all-zero}
    }
    \hfill
    \subbottom[Input Error]{
        \includegraphics[width=0.45\textwidth]{/errors/input-error}
        \label{fig:input-error}
    }
    \vfill
    \subbottom[Null Button Error]{
        \includegraphics[width=0.45\textwidth]{/errors/null-button}
        \label{fig:null-button}
    }

    \caption{Error Messages}
    \label{fig:error-handel-1}
\end{figure}

\begin{figure}[!htbp]
    \centering
    \subbottom[Not Number at Conveyor Time]{
        \includegraphics[width=0.45\textwidth]{/errors/letter-time}
        \label{fig:letter-time}
    }
    \hfill
    \subbottom[Not Number at Priority]{
        \includegraphics[width=0.45\textwidth]{/errors/letter-priority}
        \label{fig:letter-priority}
    }
    \caption{Error message for Letter as Input}
    \label{fig:error-handel-2}
\end{figure}

\subsection{Sample UI Outputs}
Figure~\ref{fig:help_page} shows the help page containing information
about the button and their purpose to assist the user
navigate and utilizing the application. 
Furthermore, Figure~\ref{fig:ripeness-all} shows an example output
for each possible case of green, yellow\_green, and yellow
ripeness classification together with bruise and not bruised and
small and medium size mangoes. 

\begin{figure}[!htbp]
	\centering
	\includegraphics[width=0.8\textwidth]{/ui/help}
	\caption{Help Page UI}
	\label{fig:help_page}
\end{figure}


\begin{figure}[!htbp]
    \centering
    \subbottom[Green and Unbruised]{
        \includegraphics[width=0.45\textwidth]{/ripe-class/green}
        \label{fig:green-result}
    }
    \hfill
    \subbottom[Yellow\_Green and Bruised]{
        \includegraphics[width=0.45\textwidth]{/ripe-class/yellow-green}
        \label{fig:yellow-green-result}
    }
    \vfill
    \subbottom[Yellow and Bruised]{
        \includegraphics[width=0.45\textwidth]{/ripe-class/yellow}
        \label{fig:yellow-result}
    }

    \caption{Sample Ripeness and Bruises Results}
    \label{fig:ripeness-all}
\end{figure}



\section{Summary} \label{sec:summary_results_and_discussions}

This chapter shows its successful integration of software intelligence, hardware
functionality, and user-centric design. The core of the system's success lies
in its high-precision deep learning models, with the final EfficientNetV2-B3 architecture
achieving exceptional accuracies of 98\% for ripeness classification and 99\% for bruise
detection. Through extensive benchmarking, modern CNNs like EfficientNet were proven
superior, offering an optimal balance of accuracy and computational efficiency. The system 
is able to get an overall percent difference to measured area of 4.8 for the size.
The system's practical validity was further
confirmed through a comparative analysis with a human expert, achieving a 79\% agreement
rate, which accounts for the inherent subjectivity of manual grading. This robust
software is embodied in a functional physical prototype that evolved into a refined
version with an efficient conveyor system and a fully enclosed, three-way sorting
mechanism that accurately directs mangoes into designated grades. Controlling this
hardware is an intuitive software application on the Raspberry Pi, featuring a
user-friendly interface that allows for custom priority weighting of mango characteristics
and includes comprehensive error handling and data logging. Overall, the results
conclusively show that the research has successfully bridged the gap between theoretical
model development and a practical, deployable system capable of automatically and
accurately grading Carabao mangoes based on customizable, user-defined standards.
\\
