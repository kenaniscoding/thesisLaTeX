\section{Concluding Remarks}

In this \documentType, the prototype is successful in grading and sorting Carabao mangoes based on 
the user priority and machine learning algorithm. More specifically, the prototype is successful in classifying Carabao mangoes based on ripeness (Green, Green Yellow, and Yellow), 
size (Large, Medium, Small), and bruises (bruised and not bruised).

% Likewise, the researchers were successful in getting a training and testing accuracy of at least 90\% for ripeness and bruises classification.

\subsection{Objectives Achieved}
\subsubsection{\Paste{GO}}
For GO, the study successfully developed a user-priority-based grading and
sorting system for Carabao mangoes by integrating machine learning and computer vision
techniques to assess ripeness, size, and bruises. 
The system achieved high accuracy and reliability while maintaining a non-destructive 
process through its hardware and software integration using a Raspberry Pi platform.


\subsubsection{\Paste{SO1}}
For SO1, the researchers designed and implemented an automated image acquisition
system consisting of a Raspberry Pi 4, camera module, LED lighting, and a conveyor belt,
which ensured consistent lighting and image alignment necessary for precise visual analysis
and classification.

\subsubsection{\Paste{SO2}}
For SO2, multiple models were trained and evaluated, with EfficientNetV2 achieving
precision, recall, and F1 scores of approximately 0.98 and accuracy above 98\%, which surpassed
the target performance threshold and validating the effectiveness of the selected machine
learning architecture.

\subsubsection{\Paste{SO3}}
For SO3, a microcontroller-driven setup using the Raspberry Pi was developed to synchronize
conveyor movement, image capture, and data processing, demonstrating a fully automated and 
self-contained embedded system capable of real-time classification.

\subsubsection{\Paste{SO4}}
For SO4, the grading module incorporated a linear weighting formula that allowed users to assign
priority values to ripeness, bruises, and size, effectively producing customizable grading outcomes
that reflected user-defined criteria and market standards.

\subsubsection{\Paste{SO5}}
For SO5, various algorithms were implemented and tested, with CNN-based EfficientNetV2 outperforming
traditional classifiers, achieving 98\% accuracy in categorizing mango ripeness into green, yellow-green,
and yellow stages based on color and texture features.

\subsubsection{\Paste{SO6}}
For SO6, the system utilized OpenCV with an average percent difference of 4.8\% in area measurement.

\subsubsection{\Paste{SO7}}
For SO7, the implemented CNN models effectively detected and classified visible surface bruises, 
achieving a 99\% accuracy rate and demonstrating robustness in identifying varying bruise intensities 
under controlled lighting conditions.

\section{Contributions}

The contributions of each group member are as follows:
\begin{itemize}
  \item BANAL Kenan A.: Scrum Master (Project manager in charge of the hardware and software integration, assisted in mango size determination, incharge of dataset collection and data augmentation) 
  \item BAUTISTA Francis Robert Miguel F.: Front End Engineer (UI/UX Designer in charge of software interface 
  and hardware assistant of the Scrum Master, assisted in dataset splitting, categorization and collection) 
  \item HERMOSURA Don Humphrey L. : Back End Engineer (in charge of mango size determination, assisted in machine learning algorithm)
  \item SALAZAR Daniel G.: Product Engineer (Software Engineer in charge of training and testing of 
  the machine learning algorithm, assisted in dataset collection and data augmentation)
\end{itemize}


\section{Recommendations}

The researchers recommend that the prototype be improved in the optimization of the machine learning algorithm
and the hardware design. The researchers also recommend that the prototype be tested in the 
actual grading and sorting of Carabao mangoes in the market. 

\section{Future Prospects}

Future researchers may consider the following recommendations for future work:
\begin{enumerate}
	\item  User testing of the prototype in the actual grading and sorting of Carabao mangoes in the Philippine market.
	\item  Additional of weight measurement to the prototype to improve the grading and sorting of Carabao mangoes.
	\item  Integration of a custom PCB to improve the hardware design of the prototype.
\end{enumerate}

