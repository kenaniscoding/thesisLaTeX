

\section{Existing Work}

The research paper written by Adam et. al. (2022) developed a ripeness grader for
Carabao mangoes. The Carabao mango ripeness grade calculated based on object and
color detection which were written in microcontroller. These are the systems designed
by the researchers that consists of Raspberry Pi 4, Arduino Uno, camera, touch screen
LCD, MQ3 gas sensor, ventilation system. The proposed system was able to ascertain an
overall reliability of 95\%: therefore, the specified objective of ascertaining the ripeness
level of the mangoes was met with success. However, accuracy and reliability of the
software system are there since the hardware design does not seem to be workable when
one must deal with the scores of mangoes \cite{adam-non-destructive-2022}. In addition, the design of the hardware does
not integrate any form of physical automating, say like the conveyor belt. Besides, the
hardware system only works efficiently when deciding the ripeness grade of mangoes
separately.

A study done by Samaniego et. al. (2023) is another research paper that supports and has
relevant information concerning the topic. The researchers proposed a fully-perovskite
photonic system which has the capability to identify and sort or grade mango based on
features such as color, weight and, conversely, signs of damages \cite{school-of-engineering-asia-pacific-college-philippines-carabao-2023}. Some of the techniques
in image processing that the researchers used included image enhancement, image
deblurring, edge detection using MATLAB and Arduino as well as color image
segmentation. By carrying out the multiple trials on the device they achieved a
classification speed of 8.132 seconds and an accuracy of 91.2\%. The proponents’
metrics used for the ratings were speed wherein the results were rated “excellent” while
the accuracy rating given was “good”. One of the limitations of the paper is that the
researchers were only limited to the color, texture, and size of the Carabao mango

Furthermore, the research paper presented by Guillergan et. al. (2024) designed an
Automated Carabao mango classifier, in which the mango image database is used to
extract the features like weight, size, area along with the ratio of the spots for grading
using Naïve Bayes Model. Concerning the quantitative test design, one had to control
and experiment with various methods of image processing that would improve the
likelihood of improved classification. The paper methodology entailed sample collection
from 300 Carabao mangoes, picture taking using a DSLR camera, and feature
deconstruction for categorization \cite{guillergan-naive-2024}. The system prototype and
the software were designed with the programming language C\# with integration of
Aforge. NET routines. The performance of this model was checked with the help of the
dataset containing 250 images, precision, recall, F-score key indicators were used. The
investigation discovered that the Naïve Bayes’ model recognized large and rejected
mangoes with 95\% accuracy and the large and small/medium difference with a 7\% error,
suggesting an application for quality differentiation and sorting in the mango business
industry. The limitations in the researchers’ paper include the researchers were able to
achieve high accuracy after using a high quality DSLR camera and the fact that the
researchers were not able to incorporate the use of microcontrollers.

\begin{table}[h]
	\centering
	\caption{Comparison of Existing Studies}
	\label{tab:comparison_existing_studies}
	\renewcommand{\arraystretch}{1.3}
	\begin{tabular}{p{0.2\textwidth}|p{0.6\textwidth}|p{0.1\textwidth}}
		\hline
		\textbf{Existing Study} & \textbf{Limitations} & \textbf{Accuracy Rating} \\
		\hline
		\cite{adam-non-destructive-2022} & No physical automation, not suitable for large amounts of mangoes, only classifies ripeness and only a sample size of 10 mangoes. & 95\% \\
		\hline
		\cite{school-of-engineering-asia-pacific-college-philippines-carabao-2023} & Focuses only on color and size. & 91.2\% \\
		\hline
		\cite{guillergan-naive-2024} & Relies on high-quality DSLR cameras, and limited automation due to not integrating microcontrollers. & 95\% \\
		\hline
		\cite{supekar-multi-parameter-2020} & No physical automation implemented. Ripeness, size, and shape-based classification achieved 100\%, 98.19\%, and 99.20\% accuracy respectively on their own. However, errors occurred when taking into account all these aspects together for grading mangoes, causing an accuracy rating deduction. & 88.88\% \\
		\hline
	\end{tabular}
\end{table}

Previous studies on mango grading have achieved an accuracy rating of up to 95\%, as
shown in Table~\ref{tab:comparison_existing_studies}. However, these studies either relied on a small sample size, which
limits statistical significance, or utilized expensive equipment, which may be
impractical. In light of this, the researchers have set a target accuracy rating of greater than or equal to 90\%.
This target ensures that the system being developed is comparable to, or better than,
existing studies that used larger sample sizes or assessed multiple mango traits at the
same time. Furthermore, this research aims to distinguish itself by not only maintaining
or exceeding the 90\% accuracy rating but also incorporating a graphical user interface
(GUI) for selective priority-based mango classification. The system will integrate both
software and hardware components, and it will evaluate a greater number of mango
traits for grading purposes.

\subsection{Sorting Algorithms}
In previous studies, researchers have implemented various artificial intelligence
algorithms in order to determine the optimal and most effective method for sorting
mangoes. One of the algorithms that was used in the classification of mangoes was the
CNN or Convolutional Neural Networks. A study done by Zheng and Huang (2021)
explored the effectiveness of CNN, specifically in classifying mangoes through image
processing. The system that the researchers developed graded mangoes into four groups
which was based on the Chinese National Standard \cite{zheng-mango-2021}. These mangoes were examined by
their shape, color uniformity, and external defects. The system that was developed had
an impressive accuracy of 97.37\% in correctly classifying the mangoes into these
grading categories
Support Vector Machine was also one of the classification algorithms that was
implemented to detect flaws in mangoes. In that study by Veling (2019), SVM
was used in the classification of diseases from mangoes. The study used 4 different
diseases/defects for testing \cite{veling-mango-2019}. The diseases were Anthracnose, Powdery Mildew, Black
Banded, and Red Rust. and provided 90\% accuracy for both the leaves and the fruit

In the study done by Schulze et. al. (2015), Simple
Linear Regression, Multiple Linear Regression, and Artificial Neural Network models
were all studied and compared for the purpose of size-mass estimation for mango fruits.
The researchers found that the Artificial Neural Network yielded a high accuracy rating
for mass estimation and for mango classification based on size with a success rate of
96.7\% \cite{schulze-development-2015}. This is attributed to the Artificial Neural Network model’s ability to learn both
linear and nonlinear relationships between the inputs and the outputs. However, a
problem can occur with the use of the model, which is overfitting. This issue occurs
when the model is overtrained with the data set such that it will start to recognize
unnecessary details such as image noise which results in poor generalization when fed
with new data. With this in mind, additional steps will be necessary to mitigate the issue.
Another research article written by Alejandro et. al. (2018)
implements a method for sorting and grading Carabao mangoes. This research focuses
on the use of Probabilistic Neural Network, which is another algorithm that is used for
pattern recognition and classification of objects. For this study, the researchers focused
on the area, color, and the black spots of the mango for their Probabilistic Neural
Network model \cite{alejandro-grading-2018}. Their research using the model yielded an accuracy rating of 87.5\% for
classification of the mangoes which means it is quite accurate for classifying mangoes
within the predefined categories. However, problems were encountered with the use of
the model when trying to identify mangoes that did not fit the predefined size categories
of small, medium, and large. This means that the PNN model may become challenged
when presented with a mango with outlying traits or traits that were very different from
the data set.

\begin{table}[h]
	\centering
	\caption{Comparison of Sorting Algorithm Models}
	\label{tab:sorting_algorithm_models}
	\renewcommand{\arraystretch}{1.3} % Adjust row spacing
	\begin{tabular}{p{0.3\textwidth}|p{0.1\textwidth}|p{0.2\textwidth}|p{0.3\textwidth}}
		\hline
		\textbf{Sorting Algorithm Model} & \textbf{Accuracy Rating} & \textbf{Criteria} & \textbf{Problems Encountered} \\
		\hline
		Convolution Neural Network & 97.37\% & shape, color, defects & Minor blemishes affected the accuracy. \\
		\hline
		Support Vector Machine & 90\% & mango defects and diseases & The model is sensitive to noise, which requires intensive image preprocessing. \\
		\hline
		Artificial Neural Network & 96.7\% & for mango size and mass & Overfitting \\
		\hline
		Probabilistic Neural Network & 87.5\% & for mango area, color, and black spots & Difficulty in identifying mangoes that have outlying features or did not fit the predefined 		categories \\
		\hline
	\end{tabular}
\end{table}

\section{Lacking in the Approaches}
%todo put the citations correctly
The majority of past researchers such as Amna et. al. (2023) and Guillergan et. al.
(2024) were able to implement a fruit and mango sorter together with an accurate AI
algorithm to detect the ripeness defects. This means that none of the previous research
papers were able to integrate an interchangeable user-priority-based grading together
with size, ripeness, and bruises using machine learning for Carabao mango sorter and
grader. Our research however would implement an automated Carabao mango sorter in
terms of size, ripeness, and bruises with its own UI, conveyor belt, stepper motors, and
bins for collecting the different ripeness and defect grade of the Carabao mango.

\section{Summary}
%todo put the citations correctly
To reiterate, there is an innovative gap that needs to be filled with regards to the process
of sorting and grading Carabao mangoes. The traditional methods for conducting this
process manually by hand, by a porous ruler, by a sugar meter, and by a color palette can
be prone to human error and expensive costs due to the number of laborers required to
do the task. On the other hand, although researchers have already taken steps to
automate the process of mango sorting and grading, there is still a need for an
implementation that takes into account size, ripeness, and bruises altogether whilst being
non-destructive and having its own embedded system. The research articles shown
above show the different computer vision and CNN approaches for sorting and
classifying mangoes. For example, a system created by Adam et al. (2022) was more
focused on ripeness detection. Samaniego et al. (2023) considered photonic systems for
grading mango fruit based on color and weight. On the other hand, Guillergan et al.
(2024) implemented the Naïve Bayes classification model on mangoes with high
accuracy, which thereby did not include any microcontroller. There was an attempt to
study each of those parameters separately and that is why the multifactorial approach
was not used. With this in mind, the system being proposed does exactly what was
mentioned, to implement a non-destructive and automated sorting and grading system
for Carabao mangoes that takes into account size, ripeness, and bruises altogether using
machine learning, as well as having its own embedded system. This system will be
mainly composed of a conveyor belt, servo motors, a camera, microcontrollers, and an
LCD display for the user interface. By doing so, the system should be able to improve
the efficiency and productivity of mango sorting and grading, remove the effect of
human error and reduce time consumption. The studies also provided critical insights
regarding the effective algorithms that can be used in classification stages in image
processing. The use of CNN had the most accuracy with manageable potential
challenges. Lastly, by scaling the implementation, the overall export quality of the
Carabao mangoes can be improved.




