\chapter*{Abstract}
% Current machine learning systems for Carabao mango sorting and grading primarily classify mangoes based on individual physical characteristics such as size, bruises, and ripeness. However, limited research exists on grading and sorting systems that can prioritize these characteristics according to user preferences, with customizable weighting for each attribute. This research developed a Carabao mango grader and sorter that evaluates ripeness, size, and bruises through non-destructive methods and machine learning, featuring an interchangeable priority system that allows users to adjust the importance of each mango attribute based on their specific requirements.
% The researchers tested different machine learning methods for classifying ripeness and bruises separately. The image dataset comprised online available images and the researchers' own Carabao mango images with a data split of 70-15-15 for training, testing, and validation respectively. Results showed that Convolutional Neural Networks (CNN), specifically EfficientNetV2, achieved optimal performance for classifying ripeness and bruises with accuracy scores of 0.97 and 0.98 respectively.
% For size classification, two methods were evaluated to determine the accuracy in measuring the length and width of Carabao mangoes. Results demonstrated that the Faster R-CNN with object detection method achieved the best overall size classification performance, with length and width measurements showing percent differences from ground truth of 13.57\% and 3.24\% respectively.
% Finally, the image acquisition system, consisting of a Raspberry Pi 4B with camera module and conveyor belt, successfully demonstrated the grading and sorting of Carabao mangoes using the developed linear grading formula.
% todo check if the abstract is good
% Current machine learning systems for \gls{Carabao mango} sorting and grading classify mangoes based on individual physical characteristics 
% such as size, \gls{bruises}, and \gls{ripeness}. However, limited research exists on systems that prioritize these characteristics according to 
% user preferences with customizable weighting. This research introduces a flexible Carabao mango grading and sorting system that 
% integrates machine learning with a user-defined weighting mechanism, enabling dynamic prioritization/elimination of ripeness, size, and bruises.
% Different machine learning methods were tested for classifying ripeness and bruises separately.
% The dataset comprised of online available images and researchers' own Carabao mango images with a data split of 70-15-15.
% \acr{CNN}, specifically EfficientNetV2, achieved optimal performance for ripeness and bruises classification 
% with accuracy scores of 98\% and 99\% respectively. To validate these results, a comparative analysis between the best model and mango expert
% was done where the overall accuracy is 79\%. For size classification, two methods (Computer Vision and Object Detection) were evaluated 
% for measuring mango length and width. \acr{Faster R-CNN} with object detection achieved best overall performance, with length and width measurement 
% showing 13.57\% and 3.24\% differences from ground truth respectively. Finally, the image acquisition system consisting of \acr{RPi}
% with camera module and conveyor belts successfully demonstrated grading and sorting using the developed linear grading formula.
Current machine learning systems for \gls{Carabao mango} sorting and grading primarily classify mangoes based on individual physical characteristics 
such as size, \gls{bruises}, and \gls{ripeness}. However, limited research has explored systems that can prioritize these characteristics 
according to user-defined preferences with customizable weighting. This study introduces a flexible Carabao mango grading and sorting system 
that integrates machine learning with a user-adjustable weighting mechanism, enabling dynamic prioritization or exclusion of ripeness, size, 
and bruises based on specific requirements.
Different machine learning methods were evaluated for classifying ripeness and bruises separately. 
The dataset consisted of both publicly available images and researchers’ own Carabao mango images, 
with a data split of 70-15-15 for training, validation, and testing, respectively. 
\acr{CNN} models, particularly EfficientNetV2, achieved optimal performance for ripeness and bruise classification 
with accuracy scores of 98\% and 99\%, respectively. To validate these results, a comparative analysis between the best-performing model 
and expert evaluations was conducted, yielding an overall agreement accuracy of 79\%.
For size classification, OpenCV method demonstrated an accurate performance, with measured area percent difference of 4.8\% 
to the manual measurement by getting its length and width, respectively. Finally, the image acquisition system, consisting of an \acr{RPi} with a camera module 
and conveyor belt setup, successfully demonstrated the proposed grading and sorting process using the developed linear grading formula.


