\chapter*{Abstract}
Carabao Mangoes are one of the sweetest mangoes in the world and one
 of the major producers of this is the Philippines.
  However, mangoes go through many screening processes, one of them being 
  sorting and grading during post harvesting which is labor intensive, prone
   to human error, and can be inefficient if done manually. Previous researchers
    have taken steps to automate the process, however, their works often focus on 
    only specific traits, and do not try to encapsulate all the physical traits
     of the mangoes altogether. Furthermore, previous researchers made the grading 
     system static or unchangeable to the user. In this study, the researchers
      will develop an automated Carabao mango grader and sorter based on ripeness, 
      size, and bruises with an interchangeable mango attribute priority through 
      non-destructive means. Using machine vision, image processing, Machine Learning,
       microcontrollers and sensors the mangoes will be physically sorted into designated
        bins via a conveyor belt system which can be controlled and monitored via a graphical
         user interface. The approach will streamline the post-harvest process and cut down on 
         human errors and labor costs, helping maintain the high quality of Carabao mango exports. 
