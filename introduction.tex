\section{Background of the Study}

Mangoes, also known as the Mangifera indica, are a member of the cashew family. 
This fruit can often be seen being farmed by countries such as Myanmar, the Philippines, 
and India as they have a tropical dry season. Being in a tropical country is an important 
aspect for mango cultivation as it ensures proper growth for mangoes. 
If aspects such as temperature and rainfall are not ideal, it may affect the quality 
of the mango \citep{noauthor-mango-2024}. 
Carabao mangoes is a variety of a mango that is found and cultivated in the
Philippines. It is known for its sweet signature taste that was recognized
sweetest in the world in the Guinness Book of World Records in 1995. The mango
was named after the national animal of the Philippines, a native breed of
buffalo. On average, it is 12.5 cm in length and 8.5 cm in diameter, having a
bright yellow color when ripe as seen in Figure \ref{fig:img1}. It is often cultivated
during late May to early July \citep{DBpediaCarabao}.

\begin{figure}[!htbp]
	\centering
	\includegraphics[width=0.5\textwidth]{fig1}
	\caption{Carabao Mangoes at Different Ripeness Stages \citep{guillermo-determining-2019}}
	\label{fig:img1}
\end{figure}

Likewise, the Philippines produced an estimated 596.34 thousand metric tons of mangoes 
during the April-June 2023 quarter, marking an 11.4 percent 
increase from the 535.43 thousand metric tons harvested in the same three-month period of 
2022. Of this total output, the Carabao mango variety accounted for the vast 
majority at 495.06 thousand metric tons, or 83.0 percent of the nation's entire mango production \citep{psa2023major}.

This shows that mangoes are a highly valued fruit in the Philippines as
it is not only the country’s national fruit but also amongst the leading
agricultural exports of the country, ranking only third below bananas and
pineapples. This gives the country the 9th slot amongst the leading exporters of
Mangoes across the world. Attributed to this ranking is the country's export of
both fresh and dried mangoes, as well as low tariff rates. This allows the
country to export a large quantity of the fruit in countries such as Singapore,
Japan, and the USA as they can enter duty free markets provided by the World
Trade Organization and Japan. Due to this, the mangoes have become a major
source of income to an estimated 2.5 million farmers in the country
\citep{centino-current-nodate}.

Before mangoes are sold in markets, they first undergo multiple post-harvest
processes. This is to ensure that the mangoes that arrive in markets are utmost
quality before being sold to consumers. Moreover, it ensures that mangoes are
contained and preserved properly such that they do not incur damages and/or get
spoiled on its transportation to the market. Processing of the mango involves
pre-cooling, cleaning, waxing, classification, grading, ripening, packaging,
preservation, storage, packing, and transportation \citep{patel-novel-2019}
\citep{rizwan-iqbal-classification-2022}.

Among the processes that mangoes undergo, classification and grading is
important as it allows the manufacturer to separate mangoes with good qualities
versus mangoes with poor qualities. According to a study by
\citep{lacap-bruise-2021}, size, length, width, volume, density, indention, and
grooves are aspects that determine the maturity of mangoes. These traits are
being checked along with the ripeness of the mango, sightings of bruise injury,
and cracks on the fruit \citep{lacap-bruise-2021} as these aspects affect the
sellability of the fruit as well as the chances of it getting spoiled sooner.

Previous studies have been made to automate the sortation process of the
mangoes. Among these is a research done by \citet{abbas-mango-2018}, which
focuses on classification of mangoes using their texture and shape features.
They do this by, first, acquiring an image of the mango using a digital camera.
Then, these images are fed to the MaZda package, which is a software originally
developed for magnetic resonance imaging. Within the MaZda package is the B11
program, which uses Principal Component Analysis, Linear Discriminant Analysis,
Nonlinear Discriminant Analysis, and texture classification to extract features
from the mango, which in this case are the length, width, and texture. This data
is then compared to a database in order to classify any given mango
\citep{abbas-mango-2018}.  

Another study is done by \citet{rizwan-iqbal-classification-2022}, which
classifies mangoes based on their color, volume, size, and shape This is done by
making use of Charge Coupled Devices, Complementary Metal-Oxide Semiconductor
sensors, and 3-layer \gls{Convolutional Neural Network}. To classify the mangoes,
images are first captured and preprocessed to be used as a data set
\citep{rizwan-iqbal-classification-2022}. This data set is then augmented to be
used as a model for the 3-layer \gls{Convolutional Neural Network}. After extracting
the features of the mango, the 3-layer \gls{Convolutional Neural Network} is used as a
method for their classification as it can mimic the human brain in pattern
recognition, and process data for decision making. This is important as some
mangoes have very subtle differences which make it difficult to differentiate
them.



\section{Prior Studies}

A paper written by \citet{amna-et-al-machine-2023}, designed an automated fruit
sorting machine based on the quality through 
an image acquisition system and \acr{CNN}. Furthermore, the results of the paper show
that the image processing detection score was 89\% while that of the tomatoes
was 92\% while the CNN model had higher validity of 95\% for mangoes and 93\% 
for tomatoes. 15\%, while the percentage of distinction between the two groups
was reported to be 5\% respectively 
\citep{amna-et-al-machine-2023}. Despite the high \gls{accuracy score}  in
detecting mango defects, the fruit sorting system only sorts based 
on the mango defects and not on ripeness, and weight.

Furthermore, the research paper presented by \citet{guillergan-naive-2024}
designed an Automated Carabao mango classifier, in which the mango image
database is used to extract the features like size, area along with the ratio of
the spots for grading using Naïve Bayes Model. For the results, the Naïve Bayes’
model recognized large and rejected mangoes with 95\% accuracy and the large and
small/medium difference with a 7\% error, suggesting an application for quality
differentiation and sorting in the mango business industry. Despite the high
accuracy of classifying Carabao mangoes, the researchers used a high quality
DSLR camera for the image acquisition system without any \gls{microcontroller}
to control the mangoes \citep{guillergan-naive-2024}. 


\section{Problem Statement}
As mangoes are among the top exports of the Philippines
\citep{centino-current-nodate}, assessing the physical deformities is a
necessity. The physical deformities of the Carabao mango can determine the
global competitiveness of the country. Having higher quality exports can often
lead to gaining competitive edge, increase in demand, increase export revenues,
and becoming less susceptible to low-wage competition
\citep{dadamo-determinants-2018}. In order to increase the quality of mango
fruit exports, a key post-harvest process is done, which is sorting and grading.
Mango sorting and grading then becomes important to determine which batches are
of high quality and can be sold for a higher price, and which batches are of low
quality and can only be sold for a low price
\citep{zhengzhou-first-industry-co-ltd-what-nodate}. Traditionally, fruit
sorting and grading is inefficient as it is done manually by hand. Some tools
are used such as porous ruler to determine fruit size and color palette for
color grading \citep{zhengzhou-first-industry-co-ltd-what-nodate}. However,
among the problems encountered in the process of manually sorting and grading
mangoes are susceptibility to human error and requiring a number of laborers to
do the task. 

With the current advancements in technology, some researchers have already taken
steps to automate the process of sorting and grading mangoes. However, these
attempts would often only consider some of the aspects pertaining to size,
ripeness, and \gls{bruises} but not dynamically change the method of sorting and grading.
Furthermore, most of the research papers have a fix static method in grading and sorting the mangoes.
This means that it doesn't take into consideration the user's priority when grading and
sorting the mangoes. Lastly, not
all research approaches were able to implement a hardware for their algorithm,
limiting their output to only a software implementation and not an embedded
system. As such the proposed system would assess the quality of the
Carabao mango based on all the mentioned mango traits, namely size,
\gls{bruises}, and ripeness while also taking into consideration being
non-destructive and the user's priority when grading and sorting the mangoes.
These aspects are important because, as was previously
mentioned, there is a need to develop a user priority based Carabao mango sorter that 
takes into
account all these aspects at the same time while being non-destructive.



\section{Objectives and Deliverables}

\subsection{General Objective (GO)}
\begin{itemize}
	\item \Copy{GO}{GO: To develop a user-priority-based grading and sorting system for Carabao mangoes, 
		using \gls{machine learning} and \gls{computer vision} techniques to assess ripeness, size, and bruises. };
\end{itemize}

\subsection{Specific Objectives (SOs)}

\begin{itemize}
	\item \Copy{SO1}{SO1: To make an image acquisition system with a conveyor belt for 
		automatic sorting and grading mangoes. };
	
	\item \Copy{SO2}{SO2: To get the precision, recall, F1 score, confusion matrix,
		and train and test accuracy metrics for classifying the ripeness and bruises with an accuracy score of at least 90\%.};
	
	\item \Copy{SO3}{SO3: To create a microcontroller-based system to operate the image acquisition system, 
		control the conveyor belt, and process the mango images through machine learning. };
	
	\item \Copy{SO4}{SO4: To grade mangoes based on user priorities for size, ripeness, and bruises.  };
	
	\item \Copy{SO5}{SO5: To classify mango ripeness based on image data using machine learning algorithms
		such as kNN, k-mean, and Naïve Bayes.  };
	
	\item \Copy{SO6}{SO6: To classify mango size based on image data by getting its length and width using OpenCV, 
		geometry, and image processing techniques. };
	
	\item \Copy{SO7}{SO7: To classify mango bruises based on image data by employing machine learning algorithms.}
\end{itemize}



\subsection{Expected Deliverables}

Table~\ref{tab:expected_deliverables} shows the outputs, 
products, results, achievements, gains, realizations, and/or
yields of the \documentType. 

\begin{center}
	{\scriptsize
		\begin{longtable}{p{0.3\textwidth}|p{0.6\textwidth}}
			\caption{Expected Deliverables per Objective} \label{tab:expected_deliverables} \\
			\hline 
			\hline 
			\textbf{Objectives} & 
			\textbf{Expected Deliverables} \\ 
			\hline 
			\endhead
			\hline 
			\multicolumn{2}{r}{\textit{Continued on next page}} \\ 
			\endfoot
			\hline 
			\endlastfoot
			\hline
			\Paste{GO} & \begin{itemize}
				\item To develop a Carabao mango grading and sorting system.
				\item To grade Carabao mangoes into three categories based on ripeness, size, and 
				bruises using machine learning.
				\item To integrate sensors and actuators to control the conveyor belt and image acquisition system.
			\end{itemize} \\ \hline
			
			\Paste{SO1} & \begin{itemize}
				\item To make an image acquisition system with a camera and LED light source.
				\item To build a flat belt conveyor for moving the mangoes.
			\end{itemize} \\ \hline
			
			\Paste{SO2} & \begin{itemize}
				\item To use a publicly available dataset of at least 10,000 mango images
				for classification of ripeness and bruises.
			\end{itemize} \\ \hline
			
			\Paste{SO3} & \begin{itemize}
				\item To develop an intuitive UI where users can start and stop the system.
				\item To implement a priority-based grading system with sliders for ripeness, bruises, and size.
			\end{itemize} \\ \hline
			
			\Paste{SO4} & \begin{itemize}
				\item To utilize a linear combination formula as the overall mango score, where each classification level 
				contributes a grade, weighted by the priority assigned to the three properties.
				\item To assign score values for each classification level of the mango.
			\end{itemize} \\ \hline
			
			\Paste{SO5} & \begin{itemize}
				\item To train a machine learning model such as kNN, k-means, or Naïve Bayes capable
				of classifying mango ripeness based on the image color.
				\item To gather a dataset of annotated images with ripeness labels.
				\item To obtain an evaluation report of performance metrics of the model.
			\end{itemize} \\ \hline
			
			\Paste{SO6} & \begin{itemize}
				\item To develop an image processing algorithm capable of determining mango 
				size using OpenCV, NumPy, and imutils.
				\item To classify mangoes based on size into small, medium, and large based on measurements.
			\end{itemize} \\ \hline
			
			\Paste{SO7} & \begin{itemize}
				\item To train a machine learning model such as 
				CNN capable of distinguishing bruised and non-bruised mangoes.
				\item To train a machine learning model such as kNN, k-means, and Naïve Bayes 
				capable of assessing the extent of bruising on the mangoes if it is significant or partial.
				\item To gather a dataset of annotated images based on bruises.
				\item To obtain an evaluation report of performance metrics of both CNN and other machine learning models.
			\end{itemize} \\ \hline
			
		\end{longtable}
	}
\end{center}


\section{Significance of the Study}

Automating the process of sorting and grading mangoes increases efficiency and
productivity for the user which would in effect remove human error in sorting
and grading and decrease the human labor and time taken to sort and grade the
mangoes.  This is especially important for farmers with a large amount of fruit
such as mangoes and a lesser labor force. A recent study showed that their
automated citrus sorter and grader using computer vision can reduce the human
labor cost and time to sort and grade when comparing the automated citrus sorter
and grader to manual human labor \cite{chakraborty-development-2023}. 

Another benefit to automating sorting and grading mangoes is the improvement in
quality control.  This implies that compared to human labor, automating sorting
and grading mangoes can uniformly assess the quality of mangoes based on size,
color, and \gls{bruises}, ensuring that the expected grade and high-quality
mangoes reach the consumer. By accurately identifying substandard mangoes, the
system helps in reducing waste and ensuring that only marketable fruits are
processed further.

Likewise, the scalability of automating sorting and grading mangoes is simpler,
especially for lower labor force farmers with large volumes of mangoes. Because
of the possibility of large-scale operations by automating sorting and grading
mangoes, farmers can now handle large volumes of mangoes, making them suitable
for commercial farms and processing plants. Moreover, it can be adapted to
different varieties of mangoes and potentially other fruits with minor
modifications.


\subsection{Technical Benefit}

\begin{enumerate}
	\item The development of an automated \gls{Carabao mango} sorter would increase the quality control 
	of classifying \gls{Carabao mango} based on ripeness, size, and bruising.
	
	\item The accuracy in sorting Carabao mangoes will be significantly improved while
	reducing the errors due to human factors in manual sorting.
	
	\item The automated \gls{Carabao mango}  sorter carefully sorts the mangoes 
	while ensuring that they remain free from bruising or further damage during the process	
\end{enumerate}

\subsection{Social Impact}

\begin{enumerate}
	\item The reduction in manual labor creates opportunities in maintenance and
	technologies in the automated \gls{Carabao mango}  sorter.
	
	\item The automated \gls{Carabao mango}  sorter system improves Carabao mango 
	standards and enhances the satisfaction of the buyers and the customers through
	guaranteeing consistent Carabao mango grade.
	
	\item Opportunity to increase sales and profit for the farmers through consistent 
	quality and grade Carabao mangoes while reducing the physical labor to sort it.
\end{enumerate}

\subsection{Environmental Welfare}

\begin{enumerate}
	\item With the utilization of non-destruction methods of classifying Carabao mangoes together with an
	accurate sorting system, overall waste from Carabao mangoes is reduced and the likelihood
	of improperly sorted mangoes is decreased.
	
	\item Automation of sorting and grading Carabao mangoes promotes sustainable farming practices.
	
\end{enumerate}



\section{Assumptions, Scope, and Delimitations}

\subsection{Assumptions}

\begin{enumerate}
	\item The Carabao mangoes are from the same source together with the same variation
	
	\item The Carabao mangoes do not have any fruit borer and diseases
	
	\item All the components do not have any form of defects
	\item The prototype would have access to constant electricity/power source.
	\item The Carabao mangoes to be tested would be in the post-harvesting stage and in the grading stage.
	\item The image-capturing system would only capture the two sides of the mango which 
	are the two largest surface areas of the skin.	
\end{enumerate}

\subsection{Scope}
\begin{enumerate}
	\item The prototype would be specifically designed to grade and 
	sort Carabao Mangoes based on only ripeness, size, and visible skin bruises.
	
	\item The mangoes used as the subject will be solely sourced from markets in the Philippines.
	
	\item The Carabao mangoes would be graded into three levels.
	\item The prototype will be using a microcontroller-based system locally stored on 
	the device itself to handle user interaction.
	\item Computer vision algorithms to be used will include image classification.
\end{enumerate}

\subsection{Delimitations}
\begin{enumerate}
	\item The project would only be able to perform sorting and grading on one specific fruit 
	which is the Carabao mango and will not be able to sort other types of mangoes.
	
	\item Additionally, the project prototype will only be able to capture, sort, and grade one 
	mango subject at a time which means the mangoes have to be placed in the conveyor belt in
	a single file line for accurate sorting. 
	
	\item For the bruises, the system will only be able to detect external bruises and 
	may not identify the non-visible and internal bruises.
	\item The system does not load the mangoes onto the conveyor belt itself. 
	Assistance is required to put mangoes into the conveyor belt to start the sorting process
	\item The prototype will be powered using \acr{AC} power and will be plugged into 
	a wall socket which is only suitable for indoor use.
\end{enumerate}


\ifFinished
\else

\section{Estimated Work Schedule and Budget}

\begin{figure}[!htbp]
	\centering
	\includegraphics[width=1\textwidth]{ganttchart}
	\caption{Gantt Chart}
	\label{fig:img2}
\end{figure}

As seen above, Table~\ref{fig:img2} shows the Gantt Chart together with the
assigned task. For the first part of the THSCP4A, the group would primarily
revise and fine tune Chapters 1 and 2 while also preparing for the defense.
After that for THSCP4B, the yellow team which consists of two members, Hermosura
and Salazar, would start buying and collecting the materials needed for
assembling the prototype. While team yellow is doing that, team purple which
consists of Banal and Baustista would start training and validating the
\acr{CNN} model based on the Carabao mango image dataset. After that integration
of the sensors and actuators together with the integration of the \acr{CNN}
model and beginning of coding of the Application to the \acr{RPi} would be
done. Once that \acr{CNN} model is deployed and the Application works testing of
the Carabao mangoes to the prototype would be done. During THSCP4C, data
gathering would be done together with polishing and revising of the final paper.

\ifPhD
\section{Publication Plan}
\graytx{\blindtext}
\fi

\fi


\section{Overview of the \documentType}

There are seven succeeding chapters. To recall, chapter 1 involves the
introduction of the thesis topic containing the background of the study,
previous studies, objectives and deliverables, assumptions, scope, and
delimitation, significance of the study, description of the project together
with the methodology, and Gantt chart and budget. Chapter 2 involves the
existing articles, the lacking in their approaches, and the summary of chapter
2. Chapter 3 involves the theoretical considerations of the thesis topic while
chapter 4 would consist of the design consideration involving the thesis topic.
Chapter 5 would involve the research methodology containing the testing
procedure and setup.  Chapter 6 would involve the results and discussion based
on the methodology while Chapter 7 would involve the conclusion,
recommendations, and future suggestions.
