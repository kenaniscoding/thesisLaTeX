Before starting the first section, provide an overview of the purpose of this chapter and its contents, and how they are relevant to your methodology.  Discuss in this chapter the relevant theories and concepts that should support your proposed solutions.

This chapter is for providing the context to your panelist/reader.  It is actually an expanded form of the Background of the Study that you have put in Chapter~\ref{ch:intro}.

%\graytx{\Blindtext}


\section{Introduction}

Chapter 3 contains the theoretical considerations that provide the research a foundation for understanding the underlying principles guiding the development of the Carabao mango sorter and grader system. 

\section{Relevant Theories and Models}

\begin{figure}[!htbp]
	\centering
	\includegraphics[width=0.5\textwidth]{theoreticalDiagram1}
	\caption{Theoretical Framework Diagram.}
	\label{fig:theoreticalDiagram1}
\end{figure}


The theoretical framework seen in figure x revolves around the concepts that revolve around the research topic. Embedded systems include the Raspberry Pi, which is the microcontroller that will be the brain of the system, DC motors, 4 channel relays, and the conveyor belt. The machine learning portion includes a neural network model, namely the Convolutional Neural Network, which will use computer vision as a method of seeing and classifying the mangoes based on their physical traits. The image processing will include methods such as size calculation and background removal using OpenCV. Lastly, the Carabao mango will be the test subject of the system.

\section{Technical Background}

At its core, the system will be using machine learning concepts pertaining to CNN and OpenCV, and may use other algorithms such as Naive Bayes and k-Nearest Neighbors to supplement the classification tasks, particularly for assessing mango ripeness, bruise detection, and size determination. The system will be built on an embedded framework, integrating a Raspberry Pi microcontroller to control the RaspberryPi camera, actuators, LED lights, and motors. A user-friendly GUI will also be utilized to ensure users can customize the prioritization of the mango sorting system.

\section{Conceptual Framework Background}


\begin{figure}[!htbp]
	\centering
	\includegraphics[width=0.5\textwidth]{theoreticalDiagram2}
	\caption{Conceptual Framework Diagram.}
	\label{fig:theoreticalDiagram2}
\end{figure}


\section{Summary}

Overall, chapter 3 establishes key concepts and theoretical considerations that form the foundation of the Carabao mango sorter and grading system. It discusses and connects each component together, explaining how each component such as the RaspberryPi and DC motors work together to create a system that utilizes machine learning and computer vision techniques to classify mangoes based on user priority.